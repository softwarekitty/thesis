\vspace{-4pt}
\section{Related Work}
\label{sec:related}
Regular expressions have been a focus point in a variety of research objectives. From the user perspective, tools have been developed to support more robust creation~\cite{Spishak:2012:TSR:2318202.2318207} or to allow visual debugging~\cite{Beck:2014:RVD:2591062.2591111}.
Building on the perspective that regexes are difficult to create, other research has focused on removing the human from the creation process by learning regular expressions from  text~\cite{Babbar:2010:CBA:1871840.1871848, Li:2008:REL:1613715.1613719}.

Regarding applications, regular expressions have been used for test case generation~\cite{Ghosh:2013:JAT:2486788.2486925, Galler:2014:STD:2683035.2683100, Anand:2013:OSM:2503903.2503991, Tillmann:2014:TAT:2642937.2642941},  and
as specifications for string constraint solvers~\cite{Trinh:2014:SSS:2660267.2660372, hampi}.
Regexes are also employed in MySQL injection prevention~\cite{Yeole:2011:ADT:1980022.1980229} and network intrusion detection~\cite{network}, or in more diverse applications like DNA sequencing alignment~\cite{1594922} or querying RDF data~\cite{Lee:2010:PSQ:1871871.1871877, Alkhateeb:2009:ESR:1540656.1540975}.


As a query language, lightweight regular expressions are pervasive in search. For example,
some data mining frameworks use regular expressions as queries (e.g., ~\cite{Begel:2010:CDE:1806799.1806821}). Efforts have also been made to expedite the processing of regular expressions on large bodies of text~\cite{Baeza-Yates:1996:FTS:235809.235810}.

%One common misconception is that all regular expression languages are \emph{regular languages} which can be represented using deterministic finite automata (DFA), and so they are easy to model, easy to describe formally and execute in O(n) time.  In fact, many regular expression matching engines run in exponential time in order to support useful features such as lazy quantifiers, capturing groups, look-aheads and back-references~\cite{msdnmatching}.  In a recent regular expression library, the RE2 projext~\cite{re2}, Russ Cox aimed to use DFAs as much as possible (maximizing speed) while supporting as many useful features as possible.

%Thousands of research papers have focused on various other regular expression-related investigations.



%In this work, we perform a feature analysis on regular expressions used in the wild and compare that set to the features supported by four popular regular expression tools.
Research tools like Hampi~\cite{hampi}, and Rex~\cite{rex}, and commercial tools like brics\cite{brics} and RE2~\cite{re2}, all support the use of regular expressions in various ways. Hampi was developed  in academia and uses regular expressions as a specification language for a constraint solver. Rex was developed by Microsoft Research and generates strings for regular expressions that can be used in  applications such as test case generation~\cite{Anand:2013:OSM:2503903.2503991, Tillmann:2014:TAT:2642937.2642941}. Brics is an open-source package that creates automata from regular expressions for manipulation and evaluation.
RE2 is an open-source tool created by Google to power code search with an efficient regex engine.


Mining properties of open source repositories is a well-studied topic, focusing, for example, on API usage patterns~\cite{Linares-Vasquez:2014:MEA:2597073.2597085} and bug characterizations~\cite{Chen:2014:ESD:2597073.2597108}.
Exploring language feature usage by mining source code has been studied extensively for
Smalltalk~\cite{Callau:2011:DUD:1985441.1985448, Callau:2013:DUD:2589712.2589718},
JavaScript~\cite{Richards:2010:ADB:1809028.1806598},
and Java~\cite{Dyer:2014:MBA:2568225.2568295, Grechanik:2010:EIL:1852786.1852801, Parnin:2013:AUJ:2589712.2589717, Livshits:2005:RAJ:2099708.2099724},
and more specifically,
Java generics~\cite{Parnin:2013:AUJ:2589712.2589717} and
Java reflection~\cite{Livshits:2005:RAJ:2099708.2099724}.
To our knowledge, this is the first work to mine and evaluate regular expression usages from existing software repositories. Related to mining work, regular expressions have been used to form queries in mining framework~\cite{Begel:2010:CDE:1806799.1806821}, but have not been the focus of the mining activities.
Surveys have been used to measure adoption of various programming languages~\cite{Meyerovich:2013:EAP:2509136.2509515, Dattero:2004:PLG:962081.962087}, and been combined with  repository analysis~\cite{Meyerovich:2013:EAP:2509136.2509515}, but have not focused on regexes.


% \subsection{Research on Regular Expressions}
% Visual debugging of regular expressions~\cite{Beck:2014:RVD:2591062.2591111}

% %the related work section in the Spishak section is very good re: regex tools like those that represent regexes as automata or grammars
% Static analysis to reduce errors in building regular expressions by using a type system to identify errors like {\tt PatternSyntaxExceptions} and {\tt IndexOutOfBoundsExceptions} at compile time~\cite{Spishak:2012:TSR:2318202.2318207}.

% \subsection{Research on Regular Expressions}
% Visual debugging of regular expressions~\cite{Beck:2014:RVD:2591062.2591111}

% \subsection{Research that Depends on Regular Expression Usage}
% Regular expressions are used as queries in a data mining framework~\cite{Begel:2010:CDE:1806799.1806821}

