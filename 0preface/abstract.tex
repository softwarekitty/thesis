%Optional thesis abstract
\cleardoublepage \phantomsection
\specialchapt{ABSTRACT}
Though regular expressions (regex) are baked into every major language, have inspired several tools and research projects, and have been around since the first days of Unix (1960?), no one has ever formally studied how they are used in practice, or what can be done to make them easier to understand.  This thesis presents the original work of studying a sample of regex taken from Python projects pulled from Github, determining what features are used most often, defining some categories that illuminate common use cases, and identifying areas of significance for tool builders.  Furthermore, this thesis defines an equivalence class model used to explore comprehension of regex, identifying the most common and most understandable representations of semantically identical regex, suggesting several refactorings and preferred representations.  Opportunities for future work include the novel and rich field of regex refactoring, semantic search of regexes, and further fundamental research into regex usage and understandability.
\begin{abstract}
\todoNow{Smooth these Abstracts together}
Regular expressions (regex) are powerful tools employed across many tasks and platforms.
Regex can be complex, so optimizing understandability of regex is desirable for maintainers.
Because of a rich feature set, there is more than one way to compose a regex to get the same desired behavior.
We define five equivalence classes where the same behavior can be achieved with multiple representations.
With the goal of finding refactorings that improve understandability, we analyze regexes in GitHub to find community standards, and obtain understandability metrics from an empirical study with 180 participants to find out which representations are more  understandable to programmers.
We found, for example, that patterns requiring one or more of some character expressed using kleene star such as \verb!`::*'! are more understandable when expressed using the plus: \verb!`:+'!.  We identify strongly preferred transformations in three of the five equivalence classes and identify opportunities for future work in improving regex refactoring.
\end{abstract}
