\chapter{CONCLUSION}
\section{Summary of contributions}
In an effort to find refactorings that improve the understandability of regexes, we created five equivalence class models and used these models to investigate the most common representations and most comprehensible representations per class.  We found the most common representations per class by both number of patterns and number of projects to be C1, D2, T1 and S2 (L3 has the most patterns, L2 has the most projects).
We also identified three strongly preferred transformations between representations (i.e., $\overrightarrow{T4 T1}$, $\overrightarrow{D2 D3}$, and  $\overrightarrow{L2 L3}$) according to the results of our comprehension tests.  We combined the results of these two investigations using a version of Kahn's topological sorting algorithm to produce a total ordering of representations within each model.  The agreement between Community Standards and Understandability in this analysis validates our results and suggests that indeed one particular representation can be preferred over others in most cases.  We can also recommend using hex to represent invisible characters in regexes instead of octal, and to escape special characters with slashes instead of wrapping them in brackets to avoid escaping them.  Further research is needed into more granular models that treat common specific cases separately, and that address the effect of length on readability when transforming from one representation to another.


The contributions of this work are:
\begin{itemize} \setlength \itemsep{.1pt}
    \item A survey of 18 professional software developers about their experience with regular expressions,
    \item An empirical analysis of regex feature usage in nearly 14,000 regular expressions in \dbfetch{nProjScanned} open-source Python projects, mapping of those features to those supported by common regex tools and survey results showing the impact of not supporting various features,
    \item An approach for measuring behavioral similarity of regular expressions and qualitative analysis of the most common behaviorally similar clusters, and
    \item An evidence-based discussion of opportunities for future work in supporting programmers who use regular expressions, including refactoring regexes, developing regex similarity analyses, and providing migration support between languages.
\end{itemize}
