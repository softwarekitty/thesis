\section{Discussion of survey results}
\subsection{Implications}


The fact that all the surveyed developers compose regexes, and half of the developers use tools to test their regexes indicates the importance of tool development for regex.  Developers complain about regexes being hard to read and hard to write.

Common uses of regexes include locating content within a file, capturing parts of strings, and parsing user input.

\begin{table}
\caption{Results of subtracting the average task frequency of ephemeral users from the average task frequency of persistent users, ordered by difference \label{table:regexPersistingTasks}}
\begin{center}
\begin{small}
\begin{tabular}{lccc}
\toprule
\textbf{Task} & \textbf{Persistence Freq.} & \textbf{Ephemeral Freq.} & \textbf{Difference} \\  \midrule \bigstrut
Counting  substrings that match a pattern & 3 & 1.7 & 1.2\\  \midrule \bigstrut
Parsing user input & 3.6 & 2.7 & 0.9\\ \midrule \bigstrut
Capturing parts of strings & 3.8 & 3.1 & 0.7\\ \midrule \bigstrut
Parsing generated text & 2.4 & 1.9 & 0.5\\  \midrule \bigstrut
Locating content within a file or files & 3.6 & 3.2 & 0.4\\ \midrule \bigstrut
Filtering collections (lists, tables, etc.) & 2.2 & 1.9 & 0.3\\ \midrule \bigstrut
Counting lines that match a pattern & 1.8 & 2.1 & -0.3\\
\bottomrule
\end{tabular}
\end{small}
\end{center}
\vspace{-12pt}
\end{table}


Only 27\% (5)
% (L,N,P,Q,T) T counts be
of participants wrote regular expressions that persist (general purpose, scripting, etc.) more frequently than in a text editor or command line tool (where they will not persist).  This result indicates that non-persistent, or \emph{ephemeral} regexes are most frequently used type of regexes.  It also suggests that a distinction can be made between types of regular expression users: those who primarily use regexes that are used once and then forgotten (ephemeral uses), and those who primarily use regexes that are maintained as an artifact (persistent users).

The most frequently performed task according to Table~\ref{table:regextasks} is `locating content within a file or files'.  This result agrees with the idea that regexes are used more often in text editors and command line tools than in general purpose languages, since locating content is often done within a text editor.  The five participants who write persistent regexes more often also answered the task frequency questions differently.  Table~\ref{table:regexPersistingTasks} describes the tasks more frequently performed by persistent users than by ephemeral users.  The



\subsection{Opportunities for future work}

\subsubsection{Studying ephemeral regexes}
Although these ephemeral regexes are more numerous than those used in general-purpose languages, collecting them for analysis presents a unique challenge, and may require the cooperation of an institution or collective.

\subsection{Threats to validity}


% Self-identification data is available in Table~\ref{table:surveyQ01T3}, as is data for techincal~\ref{table:surveyQ04} and actvity~\ref{table:surveyQ05} usage frequency.  Five feature usage data~\ref{table:surveyQ09T13}, general usage frequency data~\ref{table:surveyQ078}, back-reference preferences data~\ref{table:surveyQ2021}, .


% The eight most common features are found in over 50\% of the projects.
% Shown in Table~\ref{table:featureStats}, the STR and END features are present in over half of the scanned projects containing utilizations.  In our survey, over half (56\%) of the respondents answered that they use endpoint anchors frequently or very frequently, and none of them claimed to never use them.

% The LZY feature  is present in over 36\% of scanned projects with utilizations, and yet was not supported by two of the four major regex projects we explored, brics and RE2.
% In our developer survey, 11\% (2) of participants use this feature frequently and 6 (33\%) use it occasionally, showing a modest impact on potential users.

% When survey participants were asked if they prefer to always use numbered (BKR) or named (BKRN) back references, 66\% (12) of survey participants said that they always use BKR, and the remaining 33\% (6) said ``it depends."  No participants preferred named capture groups.  BKR is present in 5\% of scanned projects, while BKRN is present in only 1.7\%, which corroborates our findings that numbered  are generally preferred over named capture groups.

