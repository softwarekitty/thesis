\section{Feature analysis}

Unable to get enough information about Swift's underlying NSRegularExpression to include it in the table - a strong contender for future work!  Also wanted to get Vim's features but do not have time, and it is a very alien feature set!.

\subsection{Ordinary characters}

\subsection{Comparing populations}

\subsection{Taxonomy and history}

\paragraph{Portability Guides}
Similarly in JavaScript and POSIX ERE, the pattern \verb!"a\Z"! compiles to a regex matching the string \verb!"aZ"!, because the sequence \verb!"\Z"! has no special significance and the backslash is ignored.  In Python Regular Expressions, this sequence does have significance - a feature matching the absolute end of the string (after the last newline).  However, in Java, Perl, .Net and many other variants this sequence has a slightly different meaning (absolute end or before last newline).

\subsubsection{Ephemeral regex exploration}
In some environments, such as command line or text editor, regexes are used extensively by the surveyed developers (Section\todoLast{N}), but these regular expressions do not persist. Thus, using a repository analysis for feature usage only illustrates part of how regexes are used in practice. Exploring how the feature usage differs between environments would help inform tool developers about how to best support regex usage in context, and is left for future work.

\paragraph{Alternative techniques for building a corpus}
\label{sec:alternateCorpus}
any number of different regular expression languages like Perl Regular Expressions, Java Regular Expressions or .Net Regular Expressions.  Also independent of the regular expression language, the regexes studied could come from a variety of sources like sourceforge, bitbucket, a private repository, or even from github using a different technique than the one used to build the corpus (described in
\todoMid{Mention how exploring character details like literals, hex, octal and supported escape specials like bell, vertical wsp, etc is an opportunity for future work}

\paragraph{Studying a different corpus}
A technique similar to the one described in this work could be applied to a different corpus.  Ideas about alternative ways to build a corpus of regexes can be found in Section~\ref{sec:alternateCorpus}).  The concept of a community standard would be reinforced by regexes sourced from a very specific community like only text editor projects, or only shopping cart frameworks.

