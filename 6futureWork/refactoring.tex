\section{Refactoring Regexes}
Much work remains to be done in the new field of regex refactoring.  The techniques described in this study for identifying refactorings based on community standards and understandability can be applied to many other data sets, and can be refined or extended to include new strategies for identifying preferred representations.

\subsection{Equivalence models}
\label{dis:equivalenceModels}
The refactoring studies in this work used five equivalence classes, each with three to five nodes to reason about possible refactorings.  These equivalence classes are very inclusive of regexes with very different behavior, and are defined by the features and syntax a regex uses (detailed definitions are in Appendix~\ref{app:completeNodeDescriptions}).  This is not the only way to reason about refactoring regexes.  Other possible approaches are discussed in this section.

\paragraph{Other known feature-based equivalences.}  Due to the functional variety and significant number of features to consider, this work does not provide a list of all possible feature-based refactoring groups.  However the following 5 additional equivalence classes are examples of other possible groups:
\begin{description}
\item[Single line option]  \cverb!'''(.|\n)+'''! $\equiv$ \cverb!(?s)'''(.)+'''!
\item[Multi line option]  \cverb!(?m)G\n! $\equiv$ \cverb!(?m)G$!
\item[Multi line option]  \cverb!(?i)[a-z]! $\equiv$ \cverb![A-Za-z]!
\item[Backreferences]  \cverb!(X)q\1! $\equiv$ \cverb!(?P<name>X)q(?P:<name>)!
\item[Word Boundaries]  \cverb!\bZ! $\equiv$ \cverb!((?<=\w)(?=\W)|(?<=\W)(?=\w))Z!
\end{description}

\paragraph{Community-based equivalence classes.}  More narrowly defined equivalence models, specific to particular behaviors of the most frequently observed regexes in a community, would enable discover of which of these most frequently observed regexes is preferred. For example a node could require the presence of a very specific CCC like \cverb![a-zA-Z0-9_-]! (which can alternatively be represented as \cverb![\w-]!) that is frequently observed in a particular community.  The results of a preference evaluation would necessarily be impactful because the node was designed to apply to that community.

\paragraph{Combining categorization, clustering and formal tools.}  Using formal reasoning tools like Microsoft's Automata.Z3, \emph{every} regex (using supported features) can belong to a single cluster of exactly equivalent regexes. Because no regex is excluded from analysis by not belonging to a cluster, an entire body of regexes can be covered by an understandability study.
Computational limits (when determining equivalence) may present a challenge in a thorough analysis of this type, but using the categories of regex usage defined in Section~\ref{sec:categoriesDefined}, the number of pairwise comparisons can be greatly reduced to only comparisons within a category.  Future work is needed to determine the feasibility of this approach.

\paragraph{Approximating equivalence.}  In existing refactoring work, code after a refactoring cannot behave differently than it did before it was refactored.  However, it is likely that for many common use cases, like parsing dates or emails, two non-equivalent regexes that have nearly identical behavior, where the differences never apply in practice, could be considered approximately equivalent.  Future work is needed into how approximate equivalences could be useful in the genre of regex refactoring.

\subsection{Identifying Preferred Representations}
\paragraph{Refactoring to prevent catastrophic backtracking.}  Regexes like \cverb!(a+)*b!\footurl{http://www.rexegg.com/regex-explosive-quantifiers.html} represent an avenue for attack on shared systems~\cities{Kirrage2013}.  This regex is not describing a complicated set of strings, but suffers from \emph{catastrophic backtracking} because of a common engine implementation choice.  This regex can be refactored to \cverb!a*b!, which matches the same set of strings, and will not catastrophically backtrack.  The ability to cause catastrophic backtracking provides malicious users an opportunity to perform an algorithmic complexity attack~\cities{DBLP:journals/corr/abs-1301-0849}, crippling shared machines that allow users to execute arbitrary regular expressions.  So this refactoring, applied before regexes are executed, renders bad actors harmless, providing non-malicious users with greater freedoms on shared systems.

\paragraph{Refactoring for performance.}  The representation of regexes may have a strong impact on the runtime performance of a chosen regex engine. Prior work has sought to expedite the processing of regexes over large bodies of text~\cities{Baeza-Yates:1996:FTS:235809.235810}.  Refactoring regexes for performance would complement those efforts.  Further study is needed to determine which representations are most efficient in general, and for each engine specifically.

\paragraph{Refactoring for compatibility.}  As discussed in Section~\ref{sec:featureSupport}, different language variants have different feature sets.  In the case of transforming code in one language to code in another language~\cities{7372046}, regexes also must be refactored so that regexes in the transformed code use only features supported by the target language, and must maintain expected behavior.  Besides transforming code, an organization may want to enforce standards for compatibility with a regex analysis tool like Z3, HAMPI, BRICS or REX.

\subsection{Applications for regex refactoring}

\paragraph{Regex migration libraries.}  This work identified opportunities to improve the understandability of regexes in existing code bases by looking for some of the less understandable regex representations, which can be thought of as antipatterns, and refactoring to the more common or understandable representations.  Building migration libraries to accomplish these refactorings and other yet-undiscovered regex refactorings, is a promising direction of future work to ease the manual burden of this process, similar in spirit to prior work on class library migration~\cities{Balaban:2005:RSC:1103845.1094832}.

\paragraph{Regex Programming Standards.}  Many organizations enforce coding standards in their repositories to ease understandability.  Using an equivalence class model and the node counting technique described in this chapter could help to objectively develop regular expression standards for a given development community like Mozilla or OpenBSD.
