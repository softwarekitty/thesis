\section{Composition}

\subsection{Study of composition challenges}  In Section~\ref{sec:surveyResults}\todoMid{closer reference},

\subsection{Bracket parsers}
One category of clusters, \emph{Content of Brackets and Parenthesis}, parses the contents of angle brackets, which may indicate developers are using regexes to parse HTML or XML.  As the contents of angle brackets are usually unconstrained, regexes are a poor replacement for XML or HTML parsers.  This may be a missed opportunity for the regex users to take advantage of more robust tools. More research is needed into how regex users discover best practices and how aware they are of how regexes should and should not be used.

\section{Semantic Search}

\subsection{Finding a filter set}

\subsection{Automated regex repair}
Regular expression errors are common and have produced thousands of bug reports~\cite{Spishak:2012:TSR:2318202.2318207}. This provides an opportunity to introduce automated repair techniques for regular expressions.
Recent approaches to automated program repair rely on mutation operators to make small changes to source code and then re-run the test suite (e.g., ~\cite{cacm10, genprog-tse-journal}). In regular expressions, it is likely that the broken regex is close, semantically, to the desired regex. Syntax changes through mutation operators could lead to big changes in behavior, so we hypothesize that using the semantic clusters identified in Section\todoLast{N} to identify potential repair candidates could efficiently and effectively converge on a repair candidate.
