\section{Milestones in Regular Expression History}

\subsection{Kleene's theory of regular events}
In \citeyear{McCulloch1943Logical}, a new model for how nets of nerves might `reason' to react to patterns of stimulus was proposed~\cities{McCulloch1943Logical}.  In 1951, \citeauthor{Kleene1951RAND} further developed this model with the idea of 'regular events'~\cities{Kleene1951RAND}.  In his terminology, `events' are all inputs on a set of neurons in discrete time, a \emph{definite event} is some explicit sequence of events, and a \emph{regular event} is defined using three operators: 1. logical OR \bverb!|!, 2. concatenation and 3. KLE \bverb!*! repetition which represents zero or more of some definite event.  Kleene showed that `all and only regular events can be represented by nerve nets or finite automata', and went on to show that operations on regular events are closed, and to define an algebra for simplifying regular events.  The formulas used to describe regular events were named `regular expressions' in Kleene's \citeyear{kleene56} refined paper~\cities{kleene56}.

\subsection{First regex compiler}
Many additional formalisms were built on Kleene's set of three operators, and then in 1967 Ken Thompson filed a patent~\cite{ThompsonBell1971} and published a paper~\cities{Thompson:1968:PTR:363347.363387} for his implementation of the first regular expression compiler.  This compiler was written in IBM 7090 assembly for a version of `qed'\footurl{https://www.bell-labs.com/usr/dmr/www/qed.html} (quick editor) at Bell labs.  Existing editors were only able to search and replace using whole words.  Thompson's compiler  enabled qed to search and replace using regular expressions.  As described in his paper, Thompson's compiler accepted Kleene Regular Expressions and ordinary characters as input.  Later versions of this compiler eventually supported the new features STR \bverb!^!, END \bverb!$!, ANY \bverb!.!, CCC \bverb![...]!, RNG \bverb![a-z]! and NCCC \bverb![^...]!.  Although these features provided a useful shorthand, and could be considered a new language, whatever could be expressed using these features could also be expressed using only Kleene Regular Expressions features~\cities{Hopcroft:2006:IAT:1196416}.

\subsection{Early regular expressions in Unix}
Thompson went on to create Unix in \citeyear{Ritchie:1974:UTS:361011.361061} with Dennis M. Ritchie~\cities{Ritchie:1974:UTS:361011.361061}.  Early Unix relied on `ed' - an editor with regular expression search and replace capabilities based on qed.  Unix tools grep (1973), sed (1974) and awk (1977) also leveraged regular expression concepts~\cities{UnixReader1987}.  The feature set of regular expressions evolved over time, and although it is outside the scope of this thesis to capture all details of this evolutionary process, a major milestone was the creation of egrep by Alfred Aho in 1975 which effectively defined Extended Regular Expressions~\cities{Hume:1988:TTG:55329.55333}.  This new language added the features CG \bverb!(...)!, SNG \bverb!a{1}!, DBB \bverb!a{1,3}!, LWB \bverb!a{1,}!, QST \bverb!a?!, ADD \bverb!a+! as well as 12 default character classes similar to DEC \bverb!\d!, but using syntax like \bverb![:digit:]!. This new language also introduced the `backreference' BKR \bverb!(a.b)\1! feature.  This feature, which goes `back' and  `references' the content of a capture group, is noteworthy in that it is the first feature to extend the set of languages expressible by regular expressions \emph{beyond the regular languages} described by Kleene Regular Expressions~\cities{Hopcroft:2006:IAT:1196416}. Aho also wrote fgrep, which is optimized for efficiency instead of expressiveness using the Aho–Corasick algorithm~\cities{Aho:1975:ESM:360825.360855}.

\subsection{Maturity of standards}
In \citeyear{HopcroftUllman1979}, Hopcroft and Ullman published the `Cindarella' textbook covering automata and theory supporting the syntax of grep (excluding back-references)~\cities{HopcroftUllman1979}.  Perl 2 was released in 1988 with some regular expression support~\cities{perlhist}, and included shorthand for default character classes like \bverb!\d! for DEC.  The Perl community significantly boosted the popularity and user base of regular expressions~\cities{perlTimeline}.

In \citeyear{IEEE1994POSIX2}, IEEE released the POSIX.2 standard~\cities{IEEE1994POSIX2}, detailing specifications for shells and utilities, formally specifying the Basic Regular Expressions (BRE) and Extended Regular Expressions (ERE) languages.  In 1997, the O'Reilly book `Mastering Regular Expressions'~\cities{Friedl:2006:MRE:1209014} was first published, providing tutorials on regular expression usage in plain language.  In 1999, Henry Spencer released POSIX.2-compliant {\tt regcomp}~\cities{spencerTimeline}, which is a regular expression library for C, as part of 4.4BSD Unix.  By the time Perl 5.10 was released in 2007~\cities{perl5.10Release}, many advanced features had been introduced like recursion, conditionals and subroutines.
