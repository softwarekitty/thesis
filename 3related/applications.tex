\section{Applications Of Regex}
\label{sec:applications}
A variety of applications for regular expressions is explored in this section.

\subsection{End user applications}
\paragraph{Find and replace utility}  The task for which regular expressions were first implemented is finding and replacing strings in blocks of text.  This remains a central application for regular expressions, which programmers can use to save effort.  Utilities provided by text editors such as Emacs, Notepad++, Sublime Text and Eclipse include: incremental find and replace within a single file, batch find and replace within a single file or group of files, highlighting matched text and counting the number of matching strings.  Emacs also provides utilities that delete all lines or retain all lines containing a match, and aligning columns by a regex-defined delimiter.

\paragraph{System administration}  System administrators and power users rely on command-line utilities to accomplish complex computing tasks, often dealing with file names and the contents of configuration files.  The output of one utility can be used as the input for another utility using \emph{pipes}, and the mini-programs written using piped commands often rely on regular expressions to filter or transform strings in one step or another.  For example, {\tt grep}~\cities{grepManual} allows users to find strings that match a regex, and {\tt sed}~\cities{sedManual} provides a replace functionality.  The {\tt find}~\cities{findManual} utility searches filenames based on a regex, and tools like {\tt git}~\cities{gitManual} and {\tt cron}~\cities{cronManual} use regexes written in configuration files (`.gitignore' and `crontab', respectively) to specify sets of files or sets of dates and times.

\paragraph{Searching fields in relational databases}  The popular SQL query language~\cities{Chamberlin:1974:SSE:800296.811515} uses the `LIKE' operator and an exotic regular expression syntax (\bverb!%! for zero or more of any character, equivalent to \cverb!.*!, \bverb!_! to match any character, equivalent to \cverb!.!, and typical character classes using brackets) to search fields for strings.  Modern relational database systems have expanded this syntax considerably with functions such as {\tt REGEXP\_LIKE} in Oracle~\cities{OracleRegexpLike}, {\tt REGEXP} in MySQL~\cities{MySQLRegexp}, {\tt\$regex} in MongoDB~\cities{MongoDBregex} and {\tt regexp\_replace} in PostrgreSQL~\cities{PostgreSQLregexreplace}.


\subsection{Research and industry applications}
\paragraph{Meta-programming}  Regular expressions are central to YACC and Lex, which are critical compiler tools for generating parsers used in the compilation process and lexing source files, respectively.  In the case of YACC, regex are used as a meta-programming language specifying the behavior of a parser~\cities{YACCManual}.  Similarly in Lex, regexes are used to specify the behavior of a source code lexer~\cities{LexManual}.

Regular expressions have also been used for test case generation~\cities{Ghosh:2013:JAT:2486788.2486925, Galler:2014:STD:2683035.2683100, Anand:2013:OSM:2503903.2503991, Tillmann:2014:TAT:2642937.2642941},  and as specifications for string constraint solvers~\cities{Trinh:2014:SSS:2660267.2660372, hampi}.  Some data mining frameworks use regular expressions as queries (e.g., ~\cities{Begel:2010:CDE:1806799.1806821}).

\paragraph{Network administration and security} Regular expressions are used to encode forwarding paths in software-defined networks in `Merlin'~\cities{Soule:2014:MLP:2674005.2674989}, and are used in network intrusion detection~\cities{network,Sommer:2003:EBN:948109.948145} and deep packet inspection~\cities{Kumar:2006:AAM:1151659.1159952,Yu:2006:FMR:1185347.1185360}.

Regexes are employed in MySQL injection prevention~\cities{Yeole:2011:ADT:1980022.1980229} and in more diverse applications like DNA sequencing alignment~\cities{1594922} or querying RDF data~\cities{Lee:2010:PSQ:1871871.1871877, Alkhateeb:2009:ESR:1540656.1540975}.  Efforts have also been made to expedite the processing of regular expressions on large bodies of text~\cities{Baeza-Yates:1996:FTS:235809.235810}.

\subsection{Regex composition and analysis tools}
\paragraph{Composition tools} Until this work (Chapter~\ref{sec:comprehensionRefactorings}), regular expression understandability has not been studied directly, though prior work has suggested that regexes are hard to read and understand since there are tens of thousands of bug reports related to regular expressions~\cities{Spishak:2012:TSR:2318202.2318207}.  Due in part to their common use across programming languages and how susceptible regexes are to error, many researchers and practitioners have developed tools to support more robust regex creation~\cities{Spishak:2012:TSR:2318202.2318207} or to allow visual debugging~\cities{Beck:2014:RVD:2591062.2591111}.  Other tools allow users to compose regular expressions using natural language\footurl{https://github.com/VerbalExpressions/PHPVerbalExpressions}.

Tools have also been developed to make regexes easier to understand, and many are available online. Some tools will, for example, highlight parts of regexes that match parts of strings as a tool to aid in comprehension.\footurl{https://regex101.com/}  Building on the perspective that regexes are difficult to create, other research has focused on removing the human from the creation process by learning regular expressions from  text~\cities{Babbar:2010:CBA:1871840.1871848, Li:2008:REL:1613715.1613719}.

\paragraph{Analysis tools} Research tools like Hampi~\cities{hampi}, and Rex~\cities{rex}, and commercial tools like brics~\cities{brics} all support the analysis of regular expressions in various ways. Hampi was developed  in academia and uses regular expressions as a specification language for a constraint solver. Rex was developed by Microsoft Research and generates strings for regular expressions that can be used in  applications such as test case generation~\cities{Anand:2013:OSM:2503903.2503991, Tillmann:2014:TAT:2642937.2642941}. Brics is an open-source package that creates automata from regular expressions for manipulation and evaluation. Automata.Z3\footurl{https://github.com/AutomataDotNet/Automata} is one of a suite of tools developed by Microsoft to analyze regular expressions.
