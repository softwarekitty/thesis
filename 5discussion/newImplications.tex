\section{Additional Implications}

The features of Rex in particular are of importance in this thesis, as Rex was used, indirectly, to determine a similarity score between regexes, as described in Section~\ref{sec:clusteringDesign}.  The lack of support for various features reduced the number of features that could be included by .

contrast developer preferences with refactorings from sections


% \paragraph{Use Of Backreferences}\todoLast{Think about the part of all regex that use any back-references and so are not representing regular languages (vs those that are).}



When asked if they have ever used the OPT feature (\verb!(?i)!), 78\% (14) said that they had never used it, with the rest saying they had.  This provides some context to Table 4.3 - if a feature is used only 9.4\% of projects, then many programmers may have never even used it.  But DBB was at 14.5\% and in this case, 78\% \emph{had} used it.  So there may be a cutoff around 10\%...

% The eight most common features are found in over 50\% of the projects.
% Shown in Table~\ref{table:featureStats}, the STR and END features are present in over half of the scanned projects containing utilizations.  In our survey, over half (56\%) of the respondents answered that they use endpoint anchors frequently or very frequently, and none of them claimed to never use them.

% The LZY feature  is present in over 36\% of scanned projects with utilizations, and yet was not supported by two of the four major regex projects we explored, brics and RE2.
% In our developer survey, 11\% (2) of participants use this feature frequently and 6 (33\%) use it occasionally, showing a modest impact on potential users.

% When survey participants were asked if they prefer to always use numbered (BKR) or named (BKRN) back references, 66\% (12) of survey participants said that they always use BKR, and the remaining 33\% (6) said ``it depends."  No participants preferred named capture groups.  BKR is present in 5\% of scanned projects, while BKRN is present in only 1.7\%, which corroborates our findings that numbered  are generally preferred over named capture groups.
% \begin{table*}
\begin{center}
\begin{footnotesize}
\caption{How Frequently do Features Appear in Patterns, Files and Projects?}
\label{table:featureStatsOnly}
\begin{tabular}
{lllcccc  cc}
rank & code & example & \% patterns & nPatterns & nFiles & nProjects & nTokens & maxTokens. \\
\toprule[0.16em]
1 & ADD & \begin{minipage}{0.5in}\begin{verbatim}z+\end{verbatim}\end{minipage} & 44.1 & 6,003 & 9,165 & 1,204 & 11,136 & 30 \\
\midrule
2 & CG & \begin{minipage}{0.5in}\begin{verbatim}(caught)\end{verbatim}\end{minipage} & 52.4 & 7,130 & 9,559 & 1,194 & 12,707 & 17 \\
\midrule
3 & KLE & \begin{minipage}{0.5in}\begin{verbatim}.*\end{verbatim}\end{minipage} & 44.3 & 6,017 & 8,163 & 1,099 & 11,620 & 50 \\
\midrule
4 & CCC & \begin{minipage}{0.5in}\begin{verbatim}[aeiou]\end{verbatim}\end{minipage} & 32.9 & 4,468 & 7,648 & 1,026 & 8,179 & 42 \\
\midrule
5 & ANY & \begin{minipage}{0.5in}\begin{verbatim}.\end{verbatim}\end{minipage} & 34.3 & 4,657 & 6,277 & 1,005 & 7,119 & 60 \\
\midrule
6 & RNG & \begin{minipage}{0.5in}\begin{verbatim}[a-z]\end{verbatim}\end{minipage} & 19.3 & 2,631 & 5,092 & 848 & 8,043 & 50 \\
\midrule
7 & STR & \begin{minipage}{0.5in}\begin{verbatim}^\end{verbatim}\end{minipage} & 26.2 & 3,563 & 5,458 & 846 & 3,661 & 12 \\
\midrule
8 & END & \begin{minipage}{0.5in}\begin{verbatim}$\end{verbatim}\end{minipage} & 23.3 & 3,169 & 5,393 & 827 & 3,276 & 12 \\
\midrule[0.12em]
9 & NCCC & \begin{minipage}{0.5in}\begin{verbatim}[^qwxf]\end{verbatim}\end{minipage} & 14.2 & 1,935 & 3,947 & 776 & 2,718 & 15 \\
\midrule
10 & WSP & \begin{minipage}{0.5in}\begin{verbatim}\s\end{verbatim}\end{minipage} & 20.9 & 2,846 & 4,704 & 762 & 6,128 & 32 \\
\midrule
11 & OR & \begin{minipage}{0.5in}\begin{verbatim}a|b\end{verbatim}\end{minipage} & 15.5 & 2,102 & 3,926 & 708 & 2,606 & 15 \\
\midrule
12 & DEC & \begin{minipage}{0.5in}\begin{verbatim}\d\end{verbatim}\end{minipage} & 16.9 & 2,297 & 4,198 & 692 & 4,868 & 24 \\
\midrule
13 & WRD & \begin{minipage}{0.5in}\begin{verbatim}\w\end{verbatim}\end{minipage} & 10.5 & 1,430 & 2,952 & 650 & 2,037 & 13 \\
\midrule
14 & QST & \begin{minipage}{0.5in}\begin{verbatim}z?\end{verbatim}\end{minipage} & 13.8 & 1,871 & 3,707 & 645 & 3,290 & 35 \\
\midrule
15 & LZY & \begin{minipage}{0.5in}\begin{verbatim}z+?\end{verbatim}\end{minipage} & 9.6 & 1,300 & 2,221 & 605 & 1,761 & 12 \\
\midrule
16 & NCG & \begin{minipage}{0.5in}\begin{verbatim}a(?:b)c\end{verbatim}\end{minipage} & 5.8 & 791 & 1,709 & 404 & 1,453 & 28 \\
\midrule
17 & PNG & \begin{minipage}{0.5in}\begin{verbatim}(?P<name>x)\end{verbatim}\end{minipage} & 6.7 & 915 & 1,475 & 354 & 2,399 & 16 \\
\midrule
18 & SNG & \begin{minipage}{0.5in}\begin{verbatim}z{8}\end{verbatim}\end{minipage} & 4.3 & 581 & 1,267 & 340 & 1,159 & 17 \\
\midrule
19 & NWSP & \begin{minipage}{0.5in}\begin{verbatim}\S\end{verbatim}\end{minipage} & 3.6 & 484 & 776 & 270 & 676 & 10 \\
\midrule
20 & DBB & \begin{minipage}{0.5in}\begin{verbatim}z{3,8}\end{verbatim}\end{minipage} & 2.7 & 367 & 647 & 238 & 573 & 11 \\
\midrule
21 & NLKA & \begin{minipage}{0.5in}\begin{verbatim}a(?!yz)\end{verbatim}\end{minipage} & 1 & 131 & 489 & 183 & 148 & 3 \\
\midrule
22 & WNW & \begin{minipage}{0.5in}\begin{verbatim}\b\end{verbatim}\end{minipage} & 1.8 & 248 & 438 & 166 & 408 & 36 \\
\midrule
23 & NWRD & \begin{minipage}{0.5in}\begin{verbatim}\W\end{verbatim}\end{minipage} & 0.7 & 94 & 305 & 165 & 149 & 6 \\
\midrule
24 & LWB & \begin{minipage}{0.5in}\begin{verbatim}z{15,}\end{verbatim}\end{minipage} & 0.7 & 91 & 281 & 158 & 107 & 3 \\
\midrule
25 & LKA & \begin{minipage}{0.5in}\begin{verbatim}a(?=bc)\end{verbatim}\end{minipage} & 0.8 & 112 & 358 & 158 & 133 & 4 \\
\midrule
26 & OPT & \begin{minipage}{0.5in}\begin{verbatim}(?i)CasE\end{verbatim}\end{minipage} & 1.7 & 231 & 377 & 154 & 238 & 2 \\
\midrule
27 & NLKB & \begin{minipage}{0.5in}\begin{verbatim}(?<!x)yz\end{verbatim}\end{minipage} & 0.7 & 94 & 296 & 137 & 117 & 4 \\
\midrule[0.12em]
28 & LKB & \begin{minipage}{0.5in}\begin{verbatim}(?<=a)bc\end{verbatim}\end{minipage} & 0.6 & 80 & 255 & 120 & 99 & 4 \\
\midrule
29 & ENDZ & \begin{minipage}{0.5in}\begin{verbatim}\Z\end{verbatim}\end{minipage} & 0.7 & 89 & 149 & 90 & 89 & 1 \\
\midrule
30 & BKR & \begin{minipage}{0.5in}\begin{verbatim}\1\end{verbatim}\end{minipage} & 0.4 & 60 & 129 & 84 & 73 & 4 \\
\midrule
31 & NDEC & \begin{minipage}{0.5in}\begin{verbatim}\D\end{verbatim}\end{minipage} & 0.3 & 36 & 92 & 58 & 51 & 6 \\
\midrule
32 & BKRN & \begin{minipage}{0.5in}\begin{verbatim}(P?=name)\end{verbatim}\end{minipage} & 0.1 & 17 & 44 & 28 & 19 & 2 \\
\midrule
33 & VWSP & \begin{minipage}{0.5in}\begin{verbatim}\v\end{verbatim}\end{minipage} & 0.1 & 13 & 16 & 15 & 14 & 2 \\
\midrule
34 & NWNW & \begin{minipage}{0.5in}\begin{verbatim}\B\end{verbatim}\end{minipage} & 0 & 4 & 11 & 11 & 5 & 2 \\
\bottomrule[0.13em]
\end{tabular}
\end{footnotesize}
\end{center}
\end{table*}


 It depends, but IMHO the capture group really shines in programming-language use, because captured content can be put into a variable and used later.

 Simple matching that requires the whole string to match seems less useful - unless we are validating user input.


 I use split all the time, usually splitting on a comma or tab, but this needs to be flexable, why not regex?  This qualifies as worthwhile for future work.

 \subsection{Can't see the forest for the antecdotal evidence}
