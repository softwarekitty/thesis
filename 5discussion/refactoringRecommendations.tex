In this section, the implications of this work are discussed

\section{Implications For Programmers}

\paragraph{Portability} Transferring code written in one language to another language is complicated by potential differences in the regular expression features supported.  This work provides Table~\ref{table:rankedFeatureSupport} and Table~\ref{table:unrankedFeatureSupport}, which concisely describe the feature sets of eight regular expression language variants.  Programmers can use this table to develop code standards that enable portability, or to pinpoint the issues that will arise when performing such a transfer.

\paragraph{Bracket parsing detected} In the behavioral similarity clustering study (Section~\ref{sec:categoriesDefined}), balanced delimiter parsing by regexes was detected in 15\% of all projects for which clusters were formed.  This type of regex is prone to causing bugs\footurl{http://stackoverflow.com/questions/1732348} because regular expressions do not express the rules of balanced delimiters well.  The results of this study provide an indication of the pervasiveness of this problem, which helps programmers to know what to expect when considering potential bugs in code they maintain.

\subsection{Refactoring recommendations for programmers}
In a survey of 18 developers, 39\% (7) indicated that a pain point when dealing with regular expressions is that they are hard to read.  In this section, the \emph{recommended} refactorings for understandability and conformance to a community standard determined by this work are presented with programmers in mind.  Shorthand for representation styles used in this section, like T1 or C5, refers to the labeled nodes of the graph in Figure~\ref{fig:refactoringTree}.

\paragraph{For representing characters}  For understandability, use ordinary characters (T1) whenever possible.  This is the most frequently used representation (in Table~\ref{table:nodeCount}) and by far the most understandable (E1 in Table~\ref{table:testedEdgesTable}).  When invisible characters need to be represented (T1 is not possible), use T2, not T4.  This will conform to community standards (Table~\ref{table:nodeCount}) and is suggested by a (non-statistically-significant) trend in understandability scores (E8 in Table~\ref{table:testedEdgesTable}).  Characters wrapped in their own CCC, like \cverb![(]! which wraps the \verb!`('!, are not recommended by community standards, which instead recommend using the unwrapped version (Table~\ref{table:nodeCount}), but this refactoring does not show a significant difference in understandability (E10 in Table~\ref{table:testedEdgesTable}).

\paragraph{For representing character classes}  Use a range (C1) whenever possible to conform to the community standards in Table~\ref{table:nodeCount}.  Using a short OR of single characters (C5) is sometimes more understandable than a CCC of single characters (C2) (E4 in Table~\ref{table:testedEdgesTable}) but a CCC of single characters is more commonly observed in the community, so choose between them depending on your goals.  When a CCC can be represented using defaults (C4) or ranges (C1), the community standards suggest using a range (in Table~\ref{table:nodeCount}), although no understandability difference was detected in this study (E12 Table~\ref{table:testedEdgesTable}).

\paragraph{For a single QST repetition}  When using a single zero-or-one repetition using QST (D2), for understandability, transform this to an OR of the fully expanded String (D3).  This study shows that short expanded ORs are more understandable than the same regex functionality using QST (in Table~\ref{table:testedEdgesTable}).  When concerned about community standards, use QST, which is much more common (Table~\ref{table:nodeCount}).

\paragraph{For single-value repetition}  Do not use DBB with the bounds equal like \cverb!a{2,2}! (S3) - this does not conform to the community standards in Table~\ref{table:nodeCount}.  Which alternative to use has not been determined.
