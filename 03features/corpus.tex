\section{Building the corpus of patterns}
For each escaped pattern, the PCRE-parser produces a tree of feature tokens, which is converted to a vector by counting the number of each token  in the tree.  For a simple example, consider the patterns in Figure~\ref{fig:featureParsing}.  The pattern \verb!`^m+(f(z)*)+'! contains four different types of tokens. It has the kleene star (KLE), which is specified using the asterisk \verb!`*'! character, additional repetition (ADD), which is specified using the plus \verb!`+'! character, capture groups (CG), which are specified using pairs of parenthesis \verb!`(...)'! characters, and the start anchor (STR), which is specified using the caret \verb!`^'! character at the beginning of a pattern. A list of all features and abbreviations is provided in Table~\ref{table:featureStats}.

\begin{figure}[tb]
\centering
\includegraphics[height=0.6in]{nontex/illustrations/featureParsing.eps}
\caption{Two patterns parsed into feature vectors}
\label{fig:featureParsing}
\vspace{-12pt}
\end{figure}

Once all patterns were transformed into vectors, we examined each feature independently for all patterns, tracking the number of patterns and  projects that the each feature appears in at least once.

\input{table/featureStats}

\subsubsection{Feature Usage}
Table~\ref{table:featureStats} displays feature usage from the corpus and relates it to four major regex related projects. Only features appearing in at least 10 projects are listed.
The first column, \emph{rank}, lists the rank of a feature (relative to other features) in terms of the number of projects in which it appears. The next column, \emph{code}, gives a succinct reference string for the feature, and is followed by a \emph{description} column that provides a brief comment on what the feature does.  The \emph{example} column provides a short example of how the feature can be used.

The next four columns, (i.e., \emph{brics}, \emph{hampi}, \emph{Rex}, and \emph{RE2}), map to the four major research projects chosen for our investigation.  We indicate that a project supports a feature with the `\yes' symbol, and indicate that a project does not support the feature with the `\no' symbol.
The final four columns contain two pairs of usage statistics.  The first pair contains the number and percent of \emph{patterns} that a feature appears in, out of the 13,597 patterns that make up the corpus.  The second pair of columns contain the number and percent of \emph{projects} that a feature appears in out of the 1,645 projects scanned that contain at least one utilization.

One notable omission from Table~\ref{table:featureStats} is the literal feature, which is used  to specify matching any specific character.  An example pattern that contains only one literal token is the pattern \verb!`a'!.  This pattern only matches the lowercase letter `a'.  The literal feature was found in \dbfetch{P_LITERAL_PRESENT}\% of patterns.

We consider the literal feature to be necessary for any regex related tool to support, and so exclude it from Table~\ref{table:featureStats} and the rest of the feature analysis.

The eight most commonly used features, ADD, CG, KLE, CCC, ANY, RNG, STR and END,
appear in over half the projects.
CG is more commonly used in patterns than the highest ranked feature (ADD) by a wide margin (over 8\%), even though they appear in similar numbers of projects.

\subsubsection{Feature Support in Regex Tools}
While there are many regex tools available, in this work, we focus on the feature support for  four tools, brics, hampi, Rex and RE2, which offer diversity across developers (i.e., Microsoft, Google, open source, and academia) and applications. Further, as we wanted to perform a feature analysis, these four tools and their features are well-documented, allowing for easy comparison.

To create the tool mappings, we consulted documentation for each tool. For brics, we collected the set of supported features using the formal grammar\footurl{http://www.brics.dk/automaton/doc/index.html?dk/brics/automaton/RegExp.html}.  For hampi, we manually inspected the set of regexes included in the {\tt lib/regex-hampi/sampleRegex} file within the hampi repository\footurl{https://code.google.com/p/hampi/downloads/list} (this may have been an overestimation, as this included more features than specified by the formal grammar\footurl{http://people.csail.mit.edu/akiezun/hampi/Grammar.html}).  For RE2, we used the  supported feature documentation\footurl{https://re2.googlecode.com/hg/doc/syntax.html}.  For Rex, we collected the feature set empirically because we tried to parse all scraped patterns with Rex for the behavioral analysis
, and Rex provides comprehensive error feedback for unsupported features.

Of the four projects selected for this analysis, RE2 supports the most studied features (28 features) followed by hampi (25 features),  Rex (21 features), and brics (12 features).  All projects support the 8 most commonly used features except brics, which does not support STR or END.

No projects support the four look-around features LKA, NLKA, LKB and NLKB.  RE2 and hampi support the LZY, NCG, PNG and OPT features, whereas brics and Rex do not.

% The eight most common features are found in over 50\% of the projects.
% Shown in Table~\ref{table:featureStats}, the STR and END features are present in over half of the scanned projects containing utilizations.  In our survey, over half (56\%) of the respondents answered that they use endpoint anchors frequently or very frequently, and none of them claimed to never use them.

% The LZY feature  is present in over 36\% of scanned projects with utilizations, and yet was not supported by two of the four major regex projects we explored, brics and RE2.
% In our developer survey, 11\% (2) of participants use this feature frequently and 6 (33\%) use it occasionally, showing a modest impact on potential users.

% When survey participants were asked if they prefer to always use numbered (BKR) or named (BKRN) back references, 66\% (12) of survey participants said that they always use BKR, and the remaining 33\% (6) said ``it depends."  No participants preferred named capture groups.  BKR is present in 5\% of scanned projects, while BKRN is present in only 1.7\%, which corroborates our findings that numbered  are generally preferred over named capture groups.



