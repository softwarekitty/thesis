\section{About this thesis}
This thesis will begin by introducing fundamental concepts about regular expressions and nomenclature used.  This is followed by a detailed description of the Python Regular Expression features that will be examined in this thesis and a summary of historical milestones.  To convey the scope of impact that research into regular expressions has, a small survey of applications and research will be provided.  Armed with this overview, the main questions that this thesis will explore will be introduced, followed by a section on related exploratory work.  The next \todoLast{N} sections will detail the studies conducted to explore the research questions of this thesis.  Each of these sections will consist of a description of how the study was designed, followed by a presentation of results and a discussion of implications and opportunities for future work.
Each section describing an experiment may depend on the results of previous sections.  A final discussion will highlight the most important implications and opportunities for future work already presented, as well as any additional implications or future work not mentioned elsewhere.  After a conclusion summarizing everything that has been presented, an appendix of artifacts and a bibliography will complete the thesis.

This thesis will explore many details involving characters, strings and regexes.  To reduce confusion due to typesetting issues, and to avoid repeatedly qualifying quoted text with phrases like `the string' or `the regex', characters will be surrounded in single quotes like \verb!`c'!, strings will be surrounded by double quotes like \verb!"example string"!, and regexes will be presented within a grey box without any quotes like \cverb!a+b*(c|d)e\1f!.  All strings in this thesis should be considered `raw' strings - where in Python and Java source code a literal backslash in a string variable must be escaped so that two backslashes are necessary, only one will be shown in the text of this thesis.  This means that a string presented in this thesis like \verb!"a\dc"! would actually be \verb!"a\\dc"! in source code.  Invisible characters such as newline will be represented within strings in gray,
like \verb!"first line.!\gverb!\n!\verb!second line."!.
 \todoMid{Organize Sections and references}


