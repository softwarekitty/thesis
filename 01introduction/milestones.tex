\section{Milestones in regular expression history}

\subsection{Kleene's theory of regular events}
In 1943 a model for how nets of nerves might `reason' to react to patterns of stimulus was proposed in a paper by McCulloch-Pitts.  In 1951, Kleene further developed this model with the idea of 'regular events'.  In his terminology, `events' are all inputs on a set of neurons in discrete time, a `definite event' E is some explicit sequence of events, and a `regular event' is defined using three operators: 1. logical or, 2. concatenation and 3. the Kleene star which represents zero or more of some definite event.  Kleene showed that `all and only regular events can be represented by nerve nets or finite automata', and went on to show that operations on regular events are closed, and to define an algebra for simplifying regular events.  The formulas used to describe regular events were named `regular expressions' in Kleene's 1956 refined paper.

\subsection{First regex compiler}
Many additional formalisms were built on Kleene's set of three operators, and then around 1967\footurl{https://www.bell-labs.com/usr/dmr/www/qed.html}, Ken Thompson implemented the first regular expression compiler in IBM 7090 assembly for a version of `qed' (quick editor) at Bell labs.  Existing editors were only able to search and replace using whole words.  Thompson's editor was able to search and replace using the features STR, END, ANY, CCC, NCCC and KLE\footurl{https://www.bell-labs.com/usr/dmr/www/qedman.html} in a new language later known as Simple Regular Expressions (SRE).  Although these features provided a useful shorthand, they did not expand the expressiveness of SRE beyond the expressiveness of Kleene Regular Expressions.

\subsection{Early regular expressions in Unix}
Thompson went on to create Unix in 1969 with Dennis M. Ritchie, the core of which was an assembler, a shell and `ed' - an editor with regular expression search/replace capabilities based on qed.  Unix tools grep (1973), sed (1974) and awk (1977) also leveraged regular expression concepts.  The feature set of regular expressions evolved over time, and although it is outside the scope of this thesis to capture all details of this evolutionary process, a major milestone was the creation of egrep by Alfred Aho in 1975 which effectively defined Extended Regular Expressions (ERE).  This new language added the features CG, BKR, SNG, DBB, LWB, QST, ADD and OR as well as 12 default character classes similar to DEC, WRD, WSP, NDEC, NWRD and NWSP but using syntax like \verb![:digit:]! instead of the modern \verb!\d! for DEC.  The BKR feature is noteworthy in that it is the first feature to extend the set of languages expressible by regular expressions beyond the regular languages described by Kleene Regular Expressions. Aho also wrote fgrep, which is optimized for efficiency instead of expressiveness using the Aho–Corasick algorithm\todoMid{REF?}.
%https://en.wikipedia.org/wiki/Grep#Variations, https://en.wikipedia.org/wiki/Aho%E2%80%93Corasick_algorithm

\subsection{Maturity of standards}
In 1979, Hopcroft and Ullman published the `Cindarella' textbook covering automata and theory supporting the ERE language (excluding back-references).  Perl 2 was released in 1988 with some regular expression support, and included shorthand for default character classes like \verb!\d! for DEC.  The Perl community significantly boosted the popularity and user base of regular expressions.  In 1992, Henry Spencer released {\tt regcomp}, a major regular expression library for C.  In the same year, the POSIX.2 standard was also released, officially documenting both Basic Regular Expressions (BRE) and ERE.  In 1997, the O'Reilly book `Mastering Regular Expressions' was first published, and the PCRE standard was first released.  By the time Perl 5.10 was released in 2007, many advanced features had been introduced like recursion, conditionals and subroutines.

% Language theorists are interested in regular expressions that describe languages, and so they use the intersection and union operator, and take special care to formalize how the empty string is handled and relates to their algebreic operations.  We have great respect for the power of these theories, but will not rely much on them because they ignore backreferences, etc....
