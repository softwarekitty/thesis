\section{Basic concepts about regex}


\todoNow{CLEAN THIS AND FULLY DESCRIBE FEATURES AND THE BASIC TERMS WE WILL USE TO DESCRIBE ORDINARY CONCEPTS}
OK so you've never seen a regex before?  Keyword search?  Lame, try abstracting one or two characters.  And what if you want to capture some numbers, or maybe it's okay to repeat something a few times.  Specific number of times?  Okay that's great.  There are 1001 introductions to regex out there, it kind of makes me feel ill just how often this wheel is re-cast. But here we go again.


Great, now that that's done, here is how we are going to refer to the features of modern regex:
\begin{table*}[h!tb]
%gave up looking for year data
\centering
\begin{small}
\caption{Reference codes, descriptions and examples of regex features}
\label{table:featureDescriptions}
\begin{tabular}{l@{ }l@{ }l@{\hskip 0.37in} | l@{ }l@{ }l@{\hskip 0.37in}}
code & description & example & code & description & example \bigstrut \\
\noalign{\hrule height 0.08em}
ADD & one-or-more repetition & \begin{minipage}{0.5in}\begin{verbatim}z+\end{verbatim}\end{minipage}  & NDEC & any non-decimal & \begin{minipage}{0.5in}\begin{verbatim}\D\end{verbatim}\end{minipage}  \bigstrut \\
\noalign{\hrule height 0.04em}
ANY & any non-newline char & \begin{minipage}{0.5in}\begin{verbatim}.\end{verbatim}\end{minipage}  & NLKA & sequence doesn't follow  & \begin{minipage}{0.5in}\begin{verbatim}a(?!yz)\end{verbatim}\end{minipage} \bigstrut \\
\noalign{\hrule height 0.04em}
BKR & match the $i^{th}$ CG & \begin{minipage}{0.5in}\begin{verbatim}\1\end{verbatim}\end{minipage}  & NLKB & sequence doesn't precede & \begin{minipage}{0.5in}\begin{verbatim}(?<!x)yz\end{verbatim}\end{minipage}  \bigstrut \\
\noalign{\hrule height 0.04em}
BKRN & references PNG & \begin{minipage}{0.5in}\begin{verbatim}\g<name>\end{verbatim}\end{minipage}   & NWNW & negated WNW & \begin{minipage}{0.5in}\begin{verbatim}\B\end{verbatim}\end{minipage} \bigstrut \\
\noalign{\hrule height 0.04em}
CCC & custom character class & \begin{minipage}{0.5in}\begin{verbatim}[aeiou]\end{verbatim}\end{minipage}  & NWRD & non-word chars & \begin{minipage}{0.5in}\begin{verbatim}\W\end{verbatim}\end{minipage}  \bigstrut \\
\noalign{\hrule height 0.04em}
CG & a capture group & \begin{minipage}{0.5in}\begin{verbatim}(caught)\end{verbatim}\end{minipage}  & NWSP & any non-whitespace & \begin{minipage}{0.5in}\begin{verbatim}\S\end{verbatim}\end{minipage}  \bigstrut \\
\noalign{\hrule height 0.04em}
DBB & $n\le x \le m$ repetition & \begin{minipage}{0.5in}\begin{verbatim}z{3,8}\end{verbatim}\end{minipage}   & OPT & options wrapper & \begin{minipage}{0.5in}\begin{verbatim}(?i)CasE\end{verbatim}\end{minipage} \bigstrut \\
\noalign{\hrule height 0.04em}
DEC & any of: 0123456789 & \begin{minipage}{0.5in}\begin{verbatim}\d\end{verbatim}\end{minipage} & OR & logical or & \begin{minipage}{0.5in}\begin{verbatim}a|b\end{verbatim}\end{minipage} \bigstrut \\
\noalign{\hrule height 0.04em}
END & end-of-line & \begin{minipage}{0.5in}\begin{verbatim}$\end{verbatim}\end{minipage}  & PNG & named capture group & \begin{minipage}{0.5in}\begin{verbatim}(?P<name>x)\end{verbatim}\end{minipage}  \bigstrut \\
\noalign{\hrule height 0.04em}
ENDZ & absolute end of string & \begin{minipage}{0.5in}\begin{verbatim}\Z\end{verbatim}\end{minipage}   & QST & zero-or-one repetition & \begin{minipage}{0.5in}\begin{verbatim}z?\end{verbatim}\end{minipage}  \bigstrut \\
\noalign{\hrule height 0.04em}
KLE & zero-or-more repetition & \begin{minipage}{0.5in}\begin{verbatim}.*\end{verbatim}\end{minipage}   & RNG & chars within a range & \begin{minipage}{0.5in}\begin{verbatim}[a-z]\end{verbatim}\end{minipage} \bigstrut \\
\noalign{\hrule height 0.04em}
LKA & matching sequence follows & \begin{minipage}{0.5in}\begin{verbatim}a(?=bc)\end{verbatim}\end{minipage}  & SNG & exactly n repetition & \begin{minipage}{0.5in}\begin{verbatim}z{8}\end{verbatim}\end{minipage} \bigstrut \\
\noalign{\hrule height 0.04em}
LKB & matching sequence precedes & \begin{minipage}{0.5in}\begin{verbatim}(?<=a)bc\end{verbatim}\end{minipage}  & STR & start-of-line & \begin{minipage}{0.5in}\begin{verbatim}^\end{verbatim}\end{minipage} \bigstrut \\
\noalign{\hrule height 0.04em}
LWB & at least n repetition & \begin{minipage}{0.5in}\begin{verbatim}z{15,}\end{verbatim}\end{minipage}   & VWSP & matches U+000B & \begin{minipage}{0.5in}\begin{verbatim}\v\end{verbatim}\end{minipage}  \bigstrut \\
\noalign{\hrule height 0.04em}
LZY & as few reps as possible & \begin{minipage}{0.5in}\begin{verbatim}z+?\end{verbatim}\end{minipage}  & WNW & word/non-word boundary & \begin{minipage}{0.5in}\begin{verbatim}\b\end{verbatim}\end{minipage} \bigstrut \\
\noalign{\hrule height 0.04em}
NCCC & negated CCC & \begin{minipage}{0.5in}\begin{verbatim}[^qwxf]\end{verbatim}\end{minipage}  & WRD & [a-zA-Z0-9\_] & \begin{minipage}{0.5in}\begin{verbatim}\w\end{verbatim}\end{minipage} \bigstrut \\
\noalign{\hrule height 0.04em}
NCG & group without capturing & \begin{minipage}{0.5in}\begin{verbatim}a(?:b)c\end{verbatim}\end{minipage}  & WSP & \textbackslash t \textbackslash n \textbackslash r \textbackslash v \textbackslash f or space & \begin{minipage}{0.5in}\begin{verbatim}\s\end{verbatim}\end{minipage}  \bigstrut \\
\noalign{\hrule height 0.06em}
\end{tabular}
\end{small}
\vspace{-12pt}
\end{table*}
% \begin{table*}[h!tb]
% %gave up looking for year data
% \centering
% \begin{small}
% \caption{Reference codes, descriptions and examples of regex features}
% \label{table:featureDescriptions}
% \begin{tabular}{l@{ }l@{ }l@{   }l@{ }l@{ }l}
% code & description & example & code & description & example\\
% \toprule[0.16em]
% ADD & one-or-more repetition & \begin{minipage}{0.5in}\begin{verbatim}z+\end{verbatim}\end{minipage} \\
% \midrule
% ANY & any non-newline char & \begin{minipage}{0.5in}\begin{verbatim}.\end{verbatim}\end{minipage} \\
% \midrule
% BKR & match the $i^{th}$ CG & \begin{minipage}{0.5in}\begin{verbatim}\1\end{verbatim}\end{minipage} \\
% \midrule
% BKRN & references PNG & \begin{minipage}{0.5in}\begin{verbatim}\g<name>\end{verbatim}\end{minipage}  \\
% \midrule
% CCC & custom character class & \begin{minipage}{0.5in}\begin{verbatim}[aeiou]\end{verbatim}\end{minipage} \\
% \midrule
% CG & a capture group & \begin{minipage}{0.5in}\begin{verbatim}(caught)\end{verbatim}\end{minipage} \\
% \midrule
% DBB & $n\le x \le m$ repetition & \begin{minipage}{0.5in}\begin{verbatim}z{3,8}\end{verbatim}\end{minipage}  \\
% \midrule
% DEC & any of: 0123456789 & \begin{minipage}{0.5in}\begin{verbatim}\d\end{verbatim}\end{minipage}\\
% \midrule
% END & end-of-line & \begin{minipage}{0.5in}\begin{verbatim}$\end{verbatim}\end{minipage} \\
% \midrule
% ENDZ & absolute end of string & \begin{minipage}{0.5in}\begin{verbatim}\Z\end{verbatim}\end{minipage}  \\
% \midrule
% KLE & zero-or-more repetition & \begin{minipage}{0.5in}\begin{verbatim}.*\end{verbatim}\end{minipage}  \\
% \midrule
% LKA & matching sequence follows & \begin{minipage}{0.5in}\begin{verbatim}a(?=bc)\end{verbatim}\end{minipage} \\
% \midrule
% LKB & matching sequence precedes & \begin{minipage}{0.5in}\begin{verbatim}(?<=a)bc\end{verbatim}\end{minipage} \\
% \midrule
% LWB & at least n repetition & \begin{minipage}{0.5in}\begin{verbatim}z{15,}\end{verbatim}\end{minipage}  \\
% \midrule
% LZY & as few reps as possible & \begin{minipage}{0.5in}\begin{verbatim}z+?\end{verbatim}\end{minipage} \\
% \midrule
% NCCC & negated CCC & \begin{minipage}{0.5in}\begin{verbatim}[^qwxf]\end{verbatim}\end{minipage} \\
% \midrule
% NCG & group without capturing & \begin{minipage}{0.5in}\begin{verbatim}a(?:b)c\end{verbatim}\end{minipage}  \\
% \midrule
% NDEC & any non-decimal & \begin{minipage}{0.5in}\begin{verbatim}\D\end{verbatim}\end{minipage}  \\
% \midrule
% NLKA & sequence doesn't follow  & \begin{minipage}{0.5in}\begin{verbatim}a(?!yz)\end{verbatim}\end{minipage} \\
% \midrule
% NLKB & sequence doesn't precede & \begin{minipage}{0.5in}\begin{verbatim}(?<!x)yz\end{verbatim}\end{minipage}  \\
% \midrule
% NWNW & negated WNW & \begin{minipage}{0.5in}\begin{verbatim}\B\end{verbatim}\end{minipage} \\
% \midrule
% NWRD & non-word chars & \begin{minipage}{0.5in}\begin{verbatim}\W\end{verbatim}\end{minipage}  \\
% \midrule
% NWSP & any non-whitespace & \begin{minipage}{0.5in}\begin{verbatim}\S\end{verbatim}\end{minipage}  \\
% \midrule
% OPT & options wrapper & \begin{minipage}{0.5in}\begin{verbatim}(?i)CasE\end{verbatim}\end{minipage} \\
% \midrule
% OR & logical or & \begin{minipage}{0.5in}\begin{verbatim}a|b\end{verbatim}\end{minipage} \\
% \midrule
% PNG & named capture group & \begin{minipage}{0.5in}\begin{verbatim}(?P<name>x)\end{verbatim}\end{minipage}  \\
% \midrule
% QST & zero-or-one repetition & \begin{minipage}{0.5in}\begin{verbatim}z?\end{verbatim}\end{minipage}  \\
% \midrule
% RNG & chars within a range & \begin{minipage}{0.5in}\begin{verbatim}[a-z]\end{verbatim}\end{minipage} \\
% \midrule
% SNG & exactly n repetition & \begin{minipage}{0.5in}\begin{verbatim}z{8}\end{verbatim}\end{minipage}  \\
% \midrule
% STR & start-of-line & \begin{minipage}{0.5in}\begin{verbatim}^\end{verbatim}\end{minipage} \\
% \midrule
% VWSP & matches U+000B & \begin{minipage}{0.5in}\begin{verbatim}\v\end{verbatim}\end{minipage}  \\
% \midrule
% WNW & word/non-word boundary & \begin{minipage}{0.5in}\begin{verbatim}\b\end{verbatim}\end{minipage}\\
% \midrule
% WRD & [a-zA-Z0-9\_] & \begin{minipage}{0.5in}\begin{verbatim}\w\end{verbatim}\end{minipage} \\
% \midrule
% WSP & \textbackslash t \textbackslash n \textbackslash r \textbackslash v \textbackslash f or space & \begin{minipage}{0.5in}\begin{verbatim}\s\end{verbatim}\end{minipage} \\
% \bottomrule[0.13em]
% \end{tabular}
% \end{small}
% \vspace{-12pt}
% \end{table*}


Now that you have that table, let's go through it feature-by-feature and really describe the hell out of these.  Got to do that first, really, since we have to talk about when these are introduced right away, and what features break the equivalence with DFAs, and why etc.  And then boy we are going to refer to this a million times in the experimentation sections.

Language theorists are interested in regular expressions that describe languages, and so they use the intersection and union operator, and take special care to formalize how the empty string is handled and relates to their algebreic operations.  We have great respect for the power of these theories, but will not rely much on them because they ignore backreferences, etc....
