%\documentclass{sig-alternate-05-2015}

%\begin{document}
%\bstctlcite{IEEEexample:BSTcontrol}
%
% paper title
% can use linebreaks \\ within to get better formatting as desired
% \title{Refactoring Regular Expressions}


% \numberofauthors{2}
% \author{
% % 1st. author
% \alignauthor
% Carl Chapman\\
%        \affaddr{Department of Computer Science}\\
%        \affaddr{Iowa State University}\\
%        \email{carl1976@iastate.edu}
% \alignauthor
% Kathryn T. Stolee\\
%        \affaddr{Department of Computer Science}\\
%        \affaddr{North Carolina State University}\\
%        \email{ktstolee@ncsu.edu}
% \alignauthor
% }


% \maketitle


\begin{abstract}
Regular expressions (regex) are powerful tools employed across many tasks and platforms.
Regex can be complex, so optimizing understandability of regex is desirable for maintainers.
Because of a rich feature set, there is more than one way to compose a regex to get the same desired behavior.
We define five equivalence classes where the same behavior can be achieved with multiple representations.
With the goal of finding refactorings that improve understandability, we analyze regexes in GitHub to find community standards, and obtain understandability metrics from an empirical study with 180 participants to find out which representations are more  understandable to programmers.
We found, for example, that patterns requiring one or more of some character expressed using kleene star such as \verb!`::*'! are more understandable when expressed using the plus: \verb!`:+'!.  We identify strongly preferred transformations in three of the five equivalence classes and identify opportunities for future work in improving regex refactoring.
%
%as well as other, less conclusive information that can inform future investigation into regex refactoring.

\end{abstract}

\chapter{OVERVIEW}

\section{Introduction}

\todoNow{write the intro like a sales pitch}


\section{Gap in fundamental research into regex use in practice}

Although regex have provided an essential search functionality for software development for the last 47 years, are essential to parsing, compiling, security, database queries and user input validation, and are incorporated into all but the most low-level programming languages, no fundamental research has been published investigating user behaviors, preferences, use cases, pain points, or challenges in composition and comprehension.  Faced with an open field, we formulated \todoLast{N} questions to begin the work of filling this fundamental knowledge gap.  The following section articulates the motivations behind the questions explored in this thesis.

\section{Questions explored in this thesis and their motivations}

\subsection{RQ1: How are regex used in practice, especially what features are most commonly used?}

Regex researchers and tool designers must pick what features to include or exclude, which  can be a difficult  design decision. Supporting advanced features may be more expensive, taking more time and potentially making the project too complex and cumbersome to execute well.  A selection of only the simplest of regex features limits the applicability or relevance of that work. Despite extensive research effort in the area of regex support,  no research has been done about how regexes are used in practice and what features are essential for the most common use cases.

\subsection{RQ2: What behavioral categories can be observed in regex?}
\todoLast{Clean these thoughts}
If we know some categories of regex behavior, then that gives good insight into what users are really doing with regex and in turn, what behaviors are most important for future regex technologies.  Given a sample of the population of regexes in the wild, we expect to see some behavioral groups.  But how to define behavior, and how to automate the investigation enough to handle a large number of regex?  This analysis was very cpu-intensive and ran up against many implementation challenges, but is a successful first attempt to investigate regex composer's behaviors and needs.

\subsection{RQ3: What preferences, behaviors and opinions do professional developers have about using regex?}

\todoLast{Clean these thoughts} Why not just ask software developers about their use habits and preferences in a survey?  That's what we did here.  But it's important to mention that these questions had the benefit of the feature and behavioral analysis



\subsection{RQ4: Within five equivalence classes, what representations are most frequently observed?}

\todoLast{Clean these thoughts}There are many ways to represent the same functional regex, that is, the user has choices to make about how to compose a regex for any given task.  Assuming that regex composers will tend to choose the best representation most of the time, we want to know what representation choices are most frequent.

\subsection{RQ5: What representations are more comprehensible?}

\todoLast{Clean these thoughts}
After defining the equivalence classes and potential  regex refactorings we wanted to know which representations in the equivalence classes  are considered desirable and which might be smelly. Desirability for regexes can be defined many ways, including maintainable,  understandable, and performance.
We focus on refactoring for understandability.

\subsection{RQ6: For each equivalence class, which representation is preferred according to frequency and comprehensibility?}

\todoLast{Clean these thoughts}This section formalizes a technique of ordering the data from the previous two sections.

% \subsection{RQ7: What resources do students of regex employ when faced with composition challenges?}

% \todoLast{Compose thoughts about the video study}




\section{Contributions}

The contributions of this work are:
\begin{itemize} \setlength \itemsep{.1pt}
    \item A survey of 18 professional software developers about their experience with regular expressions,
    \item An empirical analysis of regex feature usage in nearly 14,000 regular expressions in \dbfetch{nProjScanned} open-source Python projects, mapping of those features to those supported by common regex tools and survey results showing the impact of not supporting various features,
    \item An approach for measuring behavioral similarity of regular expressions and qualitative analysis of the most common behaviorally similar clusters, and
    \item An evidence-based discussion of opportunities for future work in supporting programmers who use regular expressions, including refactoring regexes, developing regex similarity analyses, and providing migration support between languages.
\end{itemize}


\section{Outline}

\subsection{Sections of this thesis}
This thesis begins with a background section on the formatting standards and terminology used in this thesis to aide in understanding (Chapter~\ref{ch2}). Next is related work (Chapter~\ref{ch3}), touching on historical milestones and applications of regular expressions, as well as work on mining repositories and refactoring that has similarities to this work. Next, the five studies conducted to explore the five research questions are each presented with their own separate discussion section that focuses on the results that particular study in Chapter~\ref{ch4}.  Then a final discussion highlights the most important implications from each study in Chapter~\ref{ch5}, also presenting any implications gathered from the combination of results from multiple studies. Opportunities for future work are presented next in Chapter~\ref{sec:futureWork}, followed by a conclusion in Chapter~\ref{ch7}, and an appendix of artifacts in Appendices~\ref{ch8},~\ref{ch9},~\ref{ch10},~\ref{ch11} and ~\ref{ch12} and a bibliography.



\input{refactoring/study}

\section{Results}
\label{sec:results}

Next, we present the results of each research question.

\subsection{RQ1: How do developers use regexes?}
\label{rq1:survey}
The survey was completed by 18 participants (82\% response rate) that identified as software developer/maintainers.
Respondents have an average of nine years of programming experience ($\sigma = 4.28$).
On average, survey participants report to compose 172 regexes per year ($\sigma$ = 250) and compose regexes on average once per month, with 28\% composing multiple regexes in a week and an additional 22\% composing regexes once per week. That is, 50\% of respondents uses regexes at least weekly.
Table~\ref{tab:regexenviron} shows how frequently participants compose regexes using each of several languages and technical environments.
Six (33\%) of the survey participants report to compose regexes using general purpose programming languages (e.g., Java, C, C\#) 1-5 times per year and five (28\%) do this 6-10 times per year.  For command line usage in tools such as grep, 6 (33\%) participants use regexes 51+ times per year. Yet, regexes were rarely used in query languages like SQL. Upon further investigation, it turns out the surveyed developers were not on teams that dealt heavily with a database.

\begin{table}[ht]
\caption{Survey results for number of regexes composed per year by technical environment\label{tab:regexenviron}}
\begin{center}
\begin{small}
\begin{tabular}{l | r @{  \horiz} r @{ \horiz } r @{ \horiz } r @{ \horiz } r @{ \horiz } r }
\toprule
\textbf{Language/Environment} & \textbf{0} & \textbf{1-5} & \textbf{6-10} & \textbf{11-20} & \textbf{21-50} & \textbf{51+} \\  \midrule \bigstrut
General  (e.g., Java)  & 1 & 6 & 5 & 3& 1& 2 \\ \midrule \bigstrut
Scripting  (e.g., Perl) &5 &4 &3 &3 &2  &1 \\ \midrule \bigstrut
Query  (e.g., SQL) & 15&2 &0 &0 &1  & 0\\ \midrule \bigstrut
Command line (e.g., grep)   &2 &5 &3 &2 &0  &6 \\ \midrule \bigstrut
Text editor (e.g., IntelliJ)   & 2& 5& 0& 5& 1& 5\\
\bottomrule
\end{tabular}
\end{small}
\end{center}
\vspace{-12pt}
\end{table}


\input{table/regexactivities}

Table~\ref{tab:regexactivities} shows how frequently, on average, the participants use
regexes for various activities.
Participants answered questions using a 6-point likert scale including very frequently~(6), frequently~(5), occasionally~(4), rarely~(3), very rarely~(2), and never~(1).
Averaging across participants, among the most common usages are capturing parts of a string and locating content within a file, with both occurring somewhere between occasionally and frequently.

Using a similar 7-point likert scale that includes `always' as a seventh point, developers indicated that they test their regexes with the same frequency as they test their code (average response was 5.2, which is between frequently and very frequently).  Half of the  developers indicate that they use external tools to test their regexes, and the other half indicated that they only use tests that they write themselves. Of the nine developers using tools, six mentioned online composition aides such as \url{regex101.com} where a regex and input string are entered, and the input string is highlighted according to what is matched.

When asked an open ended question about pain points encountered with regular expressions, we observed three main categories. The most common, ``hard to compose," was represented in 61\% (11) responses. Next,
 39\% (7) developers responded that regexes are ``hard to read" and 17\% (3) indicated difficulties with ``inconsistency across implementations," which manifest when using regexes in multiple languages. These responses do not sum to 18 as three developers provided multiple parts in their answers.

\vspace{6pt}
\textbf{Summary - RQ1:}
Common uses of regexes include locating content within a file, capturing parts of strings, and parsing user input.
The fact that all the surveyed developers compose regexes, and half of the developers use tools to test their regexes indicates the importance of tool development for regex.  Developers complain about regexes being hard to read and hard to write.

\begin{table}[ht]
\begin{center}
\begin{small}
\caption{How saturated are projects with utilizations?}
\label{table:saturation}

\begin{tabular}{l|ccccc}
\toprule
\textbf{Source} & \textbf{Q1} & \textbf{Avg} & \textbf{Med} & \textbf{Q3} & \textbf{Max} \\
 \midrule \bigstrut
utilizations per project & 2 & 32 & 5 & 19 & 1,427 \\
 \midrule \bigstrut
files per project & 2 & 53 & 6 & 21 & 5,963 \\
 \midrule \bigstrut
utilizing files per project & 1 & 11 & 2 & 6 & 541 \\
 \midrule \bigstrut
utilizations per file & 1 & 2 & 1 & 3 & 207 \\
\bottomrule
\end{tabular}
\end{small}
\end{center}
\vspace{-12pt}
\end{table}


\subsection{RQ2: How  is the {\tt re} module used?}
We explore regex utilizations and flags used in the scraped Python projects.
Out of the \dbfetch{nProjScanned}\ projects scanned, \dbfetch{percentProjectsUsingRegex}\% (\dbfetch{nProjectsUsingRegex}) contained at least one regex utilization.  To illustrate how saturated projects are with regexes, we measure utilizations per project, files scanned per project, files contained utilizations, and  utilizations  per file, as shown in Table~\ref{table:saturation}.

Of projects containing at least one utilization, the average utilizations per project was 32 and the maximum  was 1,427.  The project with the most utilizations is a C\# project\footurl{https://github.com/Ouroboros/Arianrhod} that maintains a collection of source code for 20 Python libraries, including larger libraries like {\tt pip}, {\tt celery} and {\tt ipython}.  These larger Python libraries contain many utilizations.
From Table~\ref{table:saturation}, we also see that each project had an average of 11 files containing any utilization, and each of these files had an average of 2 utilizations.

\begin{figure}[tb]
\centering
\includegraphics[width=\columnwidth]{nontex/illustrations/partFunctions.eps}
\vspace{-12pt}
\caption{How often are  {\tt re} functions used? (RQ2)}
\vspace{-6pt}
\label{fig:partFunctions}
\end{figure}

\begin{figure}[tb]
\centering
\includegraphics[width=0.9\columnwidth]{nontex/illustrations/partFlags.eps}
\vspace{-6pt}
\caption{Which behavioral flags are used? (RQ2)}
\vspace{-6pt}
\label{fig:partFlags}
\end{figure}

\input{table/featureStats}

The number of projects that use each of the {\tt re} functions is shown in Figure~\ref{fig:partFunctions}.  The y-axis denotes the total utilizations, with a maximum of \dbfetch{nUsages}. The {\tt re.compile} function encompasses \dbfetch{percentCompile}\% of all utilizations.
Note that compiled objects can also be used to call functions of the re module, ie {\tt compiledObject.findall(...)}, but we ignore these utilizations so that our analysis is easier to automate, and because we are primarily interested in extracting the patterns which these 8 functions contain.

Of all utilizations, \dbfetch{percentFlags0}\% had no flag, or explicitly specified the default flag.  The debug flag, which causes the {\tt re} regex engine to display extra information about its parsing, was never observed. This may be because developers use it for debugging and choose not to commit it to their repositories.
\vspace{6pt}
\textbf{Summary - RQ2:}
Only about half of the Python projects sampled contained any utilizations.  Most utilizations used the {\tt re.compile} function to compile a regex object before actually using the regex to find a match.  Most utilizations did not use a flag to modify matching behavior.

\subsection{RQ3: Regex language feature usage}
\label{results:rq3}

We count the usages of each feature per project and as compared to all distinct regex patterns in the corpus.

\subsubsection{Feature Usage}
\label{sec:featureUsage}
Table~\ref{table:featureStats} displays feature usage from the corpus and relates it to four major regex related projects. Only features appearing in at least 10 projects are listed.
The first column, \emph{rank}, lists the rank of a feature (relative to other features) in terms of the number of projects in which it appears. The next column, \emph{code}, gives a succinct reference string for the feature, and is followed by a \emph{description} column that provides a brief comment on what the feature does.  The \emph{example} column provides a short example of how the feature can be used.

The next four columns, (i.e., \emph{brics}, \emph{hampi}, \emph{Rex}, and \emph{RE2}), map to the four major research projects chosen for our investigation (see Section~\ref{regextoolsresults}).  We indicate that a project supports a feature with the `\yes' symbol, and indicate that a project does not support the feature with the `\no' symbol.
The final four columns contain two pairs of usage statistics.  The first pair contains the number and percent of \emph{patterns} that a feature appears in, out of the 13,597 patterns that make up the corpus.  The second pair of columns contain the number and percent of \emph{projects} that a feature appears in out of the 1,645 projects scanned that contain at least one utilization.

One notable omission from Table~\ref{table:featureStats} is the literal feature, which is used  to specify matching any specific character.  An example pattern that contains only one literal token is the pattern \verb!`a'!.  This pattern only matches the lowercase letter `a'.  The literal feature was found in \dbfetch{P_LITERAL_PRESENT}\% of patterns.

We consider the literal feature to be necessary for any regex related tool to support, and so exclude it from Table~\ref{table:featureStats} and the rest of the feature analysis.

The eight most commonly used features, ADD, CG, KLE, CCC, ANY, RNG, STR and END,
appear in over half the projects.
CG is more commonly used in patterns than the highest ranked feature (ADD) by a wide margin (over 8\%), even though they appear in similar numbers of projects.

\subsubsection{Feature Support in Regex Tools}
\label{regextoolsresults}
While there are many regex tools available, in this work, we focus on the feature support for  four tools, brics, hampi, Rex and RE2, which offer diversity across developers (i.e., Microsoft, Google, open source, and academia) and applications. Further, as we wanted to perform a feature analysis, these four tools and their features are well-documented, allowing for easy comparison.

To create the tool mappings, we consulted documentation for each tool. For brics, we collected the set of supported features using the formal grammar\footurl{http://www.brics.dk/automaton/doc/index.html?dk/brics/automaton/RegExp.html}.  For hampi, we manually inspected the set of regexes included in the {\tt lib/regex-hampi/sampleRegex} file within the hampi repository\footurl{https://code.google.com/p/hampi/downloads/list} (this may have been an overestimation, as this included more features than specified by the formal grammar\footurl{http://people.csail.mit.edu/akiezun/hampi/Grammar.html}).  For RE2, we used the  supported feature documentation\footurl{https://re2.googlecode.com/hg/doc/syntax.html}.  For Rex, we collected the feature set empirically because we tried to parse all scraped patterns with Rex for the behavioral analysis (Section~\ref{rq4:results}), and Rex provides comprehensive error feedback for unsupported features.

Of the four projects selected for this analysis, RE2 supports the most studied features (28 features) followed by hampi (25 features),  Rex (21 features), and brics (12 features).  All projects support the 8 most commonly used features except brics, which does not support STR or END.

No projects support the four look-around features LKA, NLKA, LKB and NLKB.  RE2 and hampi support the LZY, NCG, PNG and OPT features, whereas brics and Rex do not.

\subsubsection{Survey Results for Feature Usage}
The pattern language for Python, which is used to compose regexes, supports default character classes like the ANY or dot character class: \verb!.! meaning, `any character except newline'.
It also supports three other default character classes: \verb!\d!, \verb!\w!, \verb!\s! (and their negations). All of these default character classes can be simulated using the custom character class (CCC) feature, which can create semantically equivalent regexes.
For example  the decimal character class: \verb!\d! is equivalent to a CCC containing all 10 digits:  \verb!\d! $\equiv$ \verb![0123456789]! $\equiv$ \verb![0-9]!.

Other default character classes such as the word character class: \verb!\w! may not be as intuitive to encode in a CCC: \verb![a-zA-Z0-9_]!.

Survey participants were asked if they use only CCC, use CCC more than default, use both equally, use default more than CCC or use only default.  Results for this question are shown in Table~\ref{tab:cccvsdefault}, with 67\% (12) indicating that they use default the most.

 Participants who favored CCC indicated that ``it is more explicit," whereas the participants who favored default character classes said,  ``it is less verbose" and ``I like using built-in code."

\begin{table}
\caption{Survey results for preferences between custom character and default character classes \label{tab:cccvsdefault}}
\begin{center}
\begin{small}
\begin{tabular}{l|c}
\toprule
\textbf{Preference} & \textbf{Frequency} \\  \midrule \bigstrut
use only CCC & 1\\ \midrule \bigstrut
use CCC more than default & 5 \\ \midrule \bigstrut
use both equally & 2\\ \midrule \bigstrut
use default more than CCC & 10\\ \midrule \bigstrut
use only default & 2\\
\bottomrule
\end{tabular}
\end{small}
\end{center}
\vspace{-12pt}
\end{table}


To further explore how participants use various regex features, participants were asked five questions about how frequently they use specific related groups of features:
\begin{itemize} \itemsep -2pt
   \item endpoint anchors (STR, END): \verb!^! and \verb!$!
   \item capture groups(CG): (capture me)
   \item word boundaries (WNW): \verb!word\b!
   \item (negative) look-ahead/behinds (LKA, NLKA, LKB, NLKB): \verb!a(?=bc)!, \verb!(?<!x)yz!, \verb!(?<=a)!, \verb!a(?!yz)!
   \item lazy repetition (LZY): \verb!ab+?!, \verb!xy{2,3}?!
\end{itemize}
These features were
chosen based on the tool feature support explored in Section~\ref{regextoolsresults}.
Results are shown in Table~\ref{tab:regexfeaturegroups}, indicating that lazy repetition and look-ahead features are rarely used and capture groups and endpoint anchors are occasionally to frequently used.

\begin{table}
\caption{Survey results for regex usage frequencies, averaged using a 6-point likert scale: Very Frequently=6, Frequently=5, Occasionally=4, Rarely=3, Very Rarely=2, and Never=1 \label{tab:regexfeaturegroups}}
\begin{center}
\begin{small}
\begin{tabular}{llc}
\toprule
\textbf{Group} & \textbf{Code} &  \textbf{Frequency} \\  \midrule \bigstrut
endpoint anchors & (STR, END) & 4.4\\ \midrule \bigstrut
capture groups & (CG) & 4.2 \\ \midrule \bigstrut
word boundaries & (WNW) & 3.5 \\ \midrule \bigstrut
lazy repetition & (LZY) &  2.9\\ \midrule \bigstrut
\multirow{2}{*}{(neg) look-ahead/behind} &  (LKA, NLKA,  & \multirow{2}{*}{2.5}\\
& LKB, NLKB) & \\
\bottomrule
\end{tabular}
\end{small}
\end{center}
\vspace{-12pt}
\end{table}


\vspace{6pt}
\textbf{Summary - RQ3:}
The eight most common features are found in over 50\% of the projects.
Shown in Table~\ref{table:featureStats}, the STR and END features are present in over half of the scanned projects containing utilizations.  In our survey, over half (56\%) of the respondents answered that they use endpoint anchors frequently or very frequently, and none of them claimed to never use them.

The LZY feature  is present in over 36\% of scanned projects with utilizations, and yet was not supported by two of the four major regex projects we explored, brics and RE2.
In our developer survey, 11\% (2) of participants use this feature frequently and 6 (33\%) use it occasionally, showing a modest impact on potential users.

When survey participants were asked if they prefer to always use numbered (BKR) or named (BKRN) back references, 66\% (12) of survey participants said that they always use BKR, and the remaining 33\% (6) said ``it depends."  No participants preferred named capture groups.  BKR is present in 5\% of scanned projects, while BKRN is present in only 1.7\%, which corroborates our findings that numbered  are generally preferred over named capture groups.

\subsection{RQ4: Regex behavioral similarity}
\label{rq4:results}


In clustering the regular expressions, we are most interested in observing behavior of regexes found in multiple projects.  Starting with the 13,597 patterns of the corpus, we discarded 10,015 (74\%) patterns that were not found in multiple projects.
Then we excluded an additional 711 (5\%) patterns that contain features not supported by Rex.  We studied the remaining 2,871 (21\%) patterns using our similarity analysis technique. The impact is that 923 projects were excluded from the data set for the similarity analysis. Omitted features are indicated in Table~\ref{table:featureStats} for Rex.

From 2,871 distinct patterns, MCL clustering identified 186 clusters with 2 or more patterns, and 2,042 clusters of size 1.
 The average size of clusters larger than size one was 4.5.  Each pattern belongs to exactly one cluster.

Three example strings generated by Rex for the first pattern are: `-()', `*'8(5)', `Oe()'.  For the third pattern, Rex generated these three strings: ` ()', `(q)F', `(n)M'.  The pattern: \verb!\(.*\)$! is very similar, but will not match the string `(n)M', and so was placed in a different cluster.

\begin{table}
\begin{center}
\caption{An example cluster containing 12 regexes, with at least one regex present in 31 different projects.  In this cluster, every regex requires `:'.}
\label{table:exampleCluster}
\begin{small}
\begin{tabular}
{lcc | lcc}
\toprule \bigstrut
\textbf{Index} & \textbf{Pattern} & \textbf{NProjects} & \textbf{Index} & \textbf{Pattern} & \textbf{NProjects} \\
 \midrule \bigstrut
1 & \begin{minipage}{1.6in}\cverb!\s*([^: ]*)\s*:(.*)!\end{minipage} & 9 & 7 & \begin{minipage}{1.6in}\cverb![:]!\end{minipage} & 6 \\
 \midrule \bigstrut
2 & \begin{minipage}{1.6in}\cverb!:+!\end{minipage} & 8 & 8 & \begin{minipage}{1.6in}\cverb!([^:]+):(.*)!\end{minipage} & 6 \\
 \midrule \bigstrut
3 & \begin{minipage}{1.6in}\cverb!(:)!\end{minipage} & 8 & 9 & \begin{minipage}{1.6in}\cverb!\s*:\s*!\end{minipage} & 4 \\
 \midrule \bigstrut
4 & \begin{minipage}{1.6in}\cverb!(:+)!\end{minipage} & 8 & 10 & \begin{minipage}{1.6in}\cverb!\:!\end{minipage} & 2 \\
 \midrule \bigstrut
5 & \begin{minipage}{1.6in}\cverb!(:)(:*)!\end{minipage} & 8 & 11 & \begin{minipage}{1.6in}\cverb!^([^:]*):[^:]*$!\end{minipage} & 2 \\
 \midrule \bigstrut
6 & \begin{minipage}{1.6in}\cverb!^([^:]*): *(.*)!\end{minipage} & 8 & 12 & \begin{minipage}{1.6in}\cverb!^[^:]*:([^:]*)$!\end{minipage} & 2 \\
\bottomrule
\end{tabular}
\vspace{-6pt}
\end{small}
\end{center}
\vspace{-12pt}
\end{table}


Table~\ref{table:exampleCluster} provides an example of a behavioral cluster containing 12 patterns (four longer patterns omitted for brevity). Patterns from this cluster are present in 31 different projects.  All patterns in this cluster share the literal `:' character. The smallest pattern, \verb!`:+'!,  matches one or more colons.


% \begin{figure}[tb]
% \centering
% \includegraphics[width=\columnwidth]{nontex/illustrations/clusterEdgesExample.eps}
% \vspace{-12pt}
% \caption{Example Of Similarity Edges Of One Cluster}
% \vspace{-6pt}
% \label{fig:clusterEdgesExample}
% \end{figure}


Another pattern from this cluster, \verb!([^:]+):(.*)!, requires at least one non-colon character to occur before a colon character.  Our similarity value between these two regexes was below the minimum of 0.75 because Rex generated many strings for `:+' that start with one or more colons.
We observe that the smallest pattern in a cluster provides insight about key characteristic that all the patterns in the cluster have in common.  A shorter pattern will tend to have less extraneous behavior because it is specifying less behavior,
yet, in order for the smallest pattern to be clustered, it had to match most of the strings created by Rex from many other patterns within the cluster, and so we observe that {the smallest pattern is useful as a representative of the cluster}.

For the rest of this paper, a cluster will be represented by one of the shortest patterns it contains, followed by the number of projects any member of the cluster appears in, so the cluster in Table~\ref{table:exampleCluster} will be represented as \verb!`:+'(31)!.  This representation is not an attempt to express all notable behavior of patterns within a cluster, but is a useful and meaningful abbreviation.
Other regexes in the cluster may exhibit more diverse behavior, for example the pattern \verb!`([^: ]+):(.*)'! requires a non-colon character to appear before a colon character.

We manually mapped the top 100 largest clusters based on the number of projects into 6 behavioral categories (determined by inspection).  The largest cluster was left out, as it was composed of patterns that trivially matched almost any string, like \verb!`b*'! and \verb!`^'!.  The remaining 99 clusters were all categorized. These clusters are briefly summarized in Table~\ref{tab:clustercats}, showing the name of the category and the number of clusters it represents, patterns in those clusters, and projects. The most common category is \emph{Multi Matches}, which contains clusters that have alternate behaviors (e.g., matching a comma or a semicolon, as in \verb!`,|;'(18)!). Each cluster was mapped to exactly one category. Next, we describe the categories, ordered by the number of projects the regex patterns map to.

\begin{table}
\begin{center}
\begin{small}
\caption{Cluster categories and sizes, ordered by number of projects containing at least one pattern in the category. \label{tab:clustercats}}
\begin{tabular}{lcccc}
\toprule
\textbf{Category} & \textbf{Clusters} & \textbf{Patterns} & \textbf{Projects} & \textbf{\% Projects} \\  \midrule \bigstrut
Multi Matches & 21 & 237 & 295 & 40\% \\
\midrule \bigstrut
Specific Char & 17 & 103 & 184 & 25\% \\
\midrule \bigstrut
Anchored Patterns & 20 & 85 & 141 & 19\% \\
\midrule \bigstrut
Two or More Chars & 16 & 40 & 120 & 16\% \\
\midrule \bigstrut
Content of Parens & 10 & 46 & 111 & 15\% \\
\midrule \bigstrut
Code Search & 15 & 27 & 92 & 13\% \\
\bottomrule
\end{tabular}
\vspace{-12pt}
\end{small}
\end{center}
\end{table}


\subsubsection{Multiple Matching Alternatives}
The patterns in these clusters match under a variety of conditions by using a character class or a disjunctive \verb!|!.
For example:
\verb!`(\W)'(89)! matches any alphanumeric character, \verb!`(\s)'(89)! matches any whitespace character, \verb!`\d'(58)! matches any numeric character, and \verb!`,|;'(18)! matches a comma or semicolon.  Most of these clusters are represented by patterns that use default character classes, as opposed to custom character classes.  This provides further support for our survey results to the question, \emph{Do you prefer to use custom character classes or default character classes more often?}, in which a majority of participants indicated they use the default classes more than custom.
This category contains 21 clusters, each appearing in an average of 33 projects.

\subsubsection{Specific Character Must Match}
\label{cluster:single}
Each cluster in this category requires one specific character to match, for example:
\verb!`\n\s*'(42)! matches only if a newline is found, \verb!`:+'(31)! matches only if a colon is found, \verb!`%'(22)!, matches only if a percent sign is found and \verb!`}'(14)! matches only if a right curly brace is found.
Table~\ref{table:exampleCluster} presents a cluster that falls under this category. While the cluster is centered on the presence of the \verb!`:'! character, the other regexes in the cluster also exhibit more diverse behavior.
The commonality of this cluster category contrasts with the survey in Section~\ref{rq1:survey} in which participants reported to very rarely or never use regexes to check for a single character (Table~\ref{tab:regexactivities}).
This category contains 17 clusters, each appearing in an average of 17.1 projects.
 These clusters have a combined total of 103 patterns, with at least one pattern present in 184 projects.

\subsubsection{Anchored Patterns}
Each of the clusters uses at least one endpoint anchor to require matches to be absolutely positioned, for example:
\verb!`(\w+)$'(35)! captures the word characters at the end of the input, \verb!`^\s'(16)! matches a whitespace at the beginning of the input, and \verb!`^-?\d+$'(17)! requires that the entire input is an (optionally negative) integer.
These anchors are the only way in regexes to guarantee that a character does (or does not) appear at a particular location by specifying what is allowed. As an example, \verb!^[-_A-Za-z0-9]+$! says that from beginning to end, only \verb![-_A-Za-z0-9]! characters are allowed, so it will fail to match if undesirable characters, such as \verb!?!, appear anywhere in the string.
This category contains 20 clusters, each appearing in an average of 15.4 projects.
These clusters have a combined total of 85 patterns, with at least one pattern present in 141 projects.

\todoMid{The thing I want to mention about anchored patterns (but have struggled to say in the past) is that they are the only way to guarantee that a character does not appear in a particular location by specifying what is allowed.  Consider the regex }
\verb!^[-_A-Za-z0-9]+$!
\todoMid{ which will fail to match if an undesirable character like `?' appears anywhere in the input.  In logic, there is a similar phenomenon.  That is, `Always' is true iff `Not Exists' of the negation is true, and by requiring an entire input to always maintain some abstraction, you can indirectly specify the negation of another (inverse) abstraction.  Even with only one anchor point, a regex like }
\verb!.*[0-9]$!
\todoMid{ is creating an ultimatum about the end being a digit.  Without the endpoint anchors, I don't see how one could specify absolutes about an input. }

\subsubsection{Content of Brackets and Parenthesis}
\label{cluster:contentparens}
The clusters in this category center around finding a pair of characters that surround content, often also capturing that content. For example,
\verb!`\(.*\)'(29)! matches when content is surrounded by parentheses and \verb!`".*"'(25)! matches  when content is surrounded by double quotes.  The cluster \verb!`<(.+)>'(23)! matches and captures content surrounded by angled brackets.
This category contains 10 clusters, each appearing in an average of 18.4 projects.
 These clusters have a combined total of 46 patterns, with at least one pattern present in 111 projects.
\todoMid{include this?, and }
\verb!`\[.*\]'(22)!
\todoMid{matches when content is surrounded by square brackets}

\subsubsection{Two or More Characters in Sequence}
\label{cluster:multiple}
These clusters require several characters in a row to match some pattern, for example:
\verb!`\d+\.\d+'(30)! requires one or more digits followed by a period character, followed by one or more digits.  The cluster \verb!`  '(17)! requires two spaces in a row,
\verb!`([A-Z][a-z]+[A-Z][^ ]+)'(11)!,
and \verb!`@[a-z]+'(9)! requires the at symbol followed by two or more lowercase characters, as in a twitter handle.
This category contains 16 clusters, each appearing in an average of 13 projects.
These clusters have a combined total of 40 patterns, with at least one pattern present in 120 projects.

\todoMid{Again, it might be interesting to look at what particular sequences are looking like.  I think I mention this again in the discussion, but should we put it here instead?}

\subsubsection{Code Search and Variable Capturing}
\label{cluster:search}
These clusters show a recognizable effort to parse source code or URLs. For example,
\verb!`^https?://'(23)! matches a web address, and \verb!`(.+)=(.+)'(9)! matches an assignment statement, capturing both the variable name and value.
The cluster  \verb!`\$\{([\w\-]+)\}'(11)! matches an evaluated string interpolation and captures the code to evaluate.
This category contains 15 clusters, each appearing in an average of 11.7 projects.
These clusters have a combined total of 27 patterns, with at least one pattern present in 92 projects.

\vspace{6pt}
\textbf{Summary - RQ4:}
When tool designers are considering what features to include, data about usage in practice is valuable.  Behavioral similarity clustering  helps to discern these behaviors by looking beyond the structural details of specific patterns and seeing trends in  matching behavior. We are also able to find out what features are being used in these behavioral trends so that we can make assertions about why certain features are important.
We used the behavior of individual patterns to form clusters, and identified six main categories for the clusters.
 Overall, we see that many clusters are defined by the presence of particular tokens, such as the colon for the cluster in Table~\ref{table:exampleCluster}.
We identified six main categories that define regex behavior at a high level: matching with alternatives, matching literal characters, matching with sequences, matching with endpoint anchors, parsing contents of brackets or braces, or searching and capturing code, and can be considered in conjunction with the self-described regex activities from the survey in Table~\ref{tab:regexactivities} to be representative of common uses for regexes.
One of the six common cluster categories, \emph{Code Search and Variable Capturing}, has a very specific purpose of parsing source code files. This shows a very specific and common use of regular expressions in practice.

\input{refactoring/table/groupANOVATable}
\chapter{DISCUSSION}
\section{Implications of the thesis as a whole}

\section{Opportunities for future work studying regular expressions}
\subsection{Semantic search}
\subsection{Ephemeral regex}
\subsection{Comparing regex usage across communities}
\subsection{Evolution of patterns}


\chapter{RELATED WORK}

\section{Milestones In Regular Expression History}

\subsection{Kleene's theory of regular events}
In \citeyear{McCulloch1943Logical}, a model for how nets of nerves might `reason' to react to patterns of stimulus was proposed by \citeauthor{McCulloch1943Logical}.  In 1951, \citeauthor{Kleene1951RAND} further developed this model with the idea of 'regular events'.  In his terminology, `events' are all inputs on a set of neurons in discrete time, a \emph{definite event} is some explicit sequence of events, and a \emph{regular event} is defined using three operators: 1. logical OR (\verb!|!), 2. concatenation and 3. KLE (\verb!*!) repetition which represents zero or more of some definite event.  Kleene showed that `all and only regular events can be represented by nerve nets or finite automata', and went on to show that operations on regular events are closed, and to define an algebra for simplifying regular events.  The formulas used to describe regular events were named `regular expressions' in Kleene's \citeyear{kleene56} refined paper.

\subsection{First regex compiler}
Many additional formalisms were built on Kleene's set of three operators, and then in 1967 Ken Thompson filed a patent~\cite{ThompsonBell1971} and published a paper (\citeyear{Thompson:1968:PTR:363347.363387}) for his implementation of the first regular expression compiler.  This compiler was written in IBM 7090 assembly for a version of `qed'\footurl{https://www.bell-labs.com/usr/dmr/www/qed.html} (quick editor) at Bell labs.  Existing editors were only able to search and replace using whole words.  Thompson's compiler  enabled qed to search and replace using regular expressions.  As described in his paper, Thompson's compiler accepted Kleene Regular Expressions and ordinary characters as input.  Later versions of this compiler eventually supported the new features STR (\verb!^!), END (\verb!$!), ANY (\verb!.!), CCC (\verb![...]!), RNG (\verb![a-z]!) and NCCC (\verb![^...]!).  Although these features provided a useful shorthand, and could be considered a new language, whatever could be expressed using these features could also be expressed using only Kleene Regular Expressions features~\citep{Hopcroft:2006:IAT:1196416}.

\subsection{Early regular expressions in Unix}
Thompson went on to create Unix in \citeyear{Ritchie:1974:UTS:361011.361061} with Dennis M. Ritchie.  Early Unix relied on `ed' - an editor with regular expression search/replace capabilities based on qed.  Unix tools grep (1973), sed (1974) and awk (1977) also leveraged regular expression concepts \citep{UnixReader1987}.  The feature set of regular expressions evolved over time, and although it is outside the scope of this thesis to capture all details of this evolutionary process, a major milestone was the creation of egrep by Alfred Aho in 1975 which effectively defined Extended Regular Expressions~\citep{Hume:1988:TTG:55329.55333}.  This new language added the features CG ( \verb!(...)! ), BKR (\verb!(a)\1!), SNG (\verb!a{1}!), DBB (\verb!a{1,3}!), LWB (\verb!a{1,}!), QST (\verb!a?!), ADD (\verb!a+!) as well as 12 default character classes similar to DEC (\verb!\d!), WRD (\verb!\w!) and WSP (\verb!\s!), but using syntax like \verb![:digit:]!.  The BKR feature is noteworthy in that it is the first feature to extend the set of languages expressible by regular expressions beyond the regular languages described by Kleene Regular Expressions~\citep{Hopcroft:2006:IAT:1196416}. Aho also wrote fgrep, which is optimized for efficiency instead of expressiveness using the Aho–Corasick algorithm~\citep{Aho:1975:ESM:360825.360855}.

\subsection{Maturity of standards}
In \citeyear{HopcroftUllman1979}, Hopcroft and Ullman published the `Cindarella' textbook covering automata and theory supporting the syntax of grep (excluding back-references).  Perl 2 was released in 1988 with some regular expression support~\citep{perlhist}, and included shorthand for default character classes like \verb!\d! for DEC.  The Perl community significantly boosted the popularity and user base of regular expressions~\citep{perlTimeline}.

In \citeyear{IEEE1994POSIX2}, IEEE released the POSIX.2 standard, detailing specifications for shells and utilities, formally specifying the Basic Regular Expressions (BRE) and Extended Regular Expressions (ERE) languages.  In 1997, the O'Reilly book `Mastering Regular Expressions'~\citep{Friedl:2006:MRE:1209014} was first published, providing tutorials on regular expression usage in plain language.  In 1999, Henry Spencer released POSIX.2-compliant {\tt regcomp}~\citep{spencerTimeline}, which is a regular expression library for C, as part of 4.4BSD Unix.  By the time Perl 5.10 was released in 2007~\citep{perl5.10Release}, many advanced features had been introduced like recursion, conditionals and subroutines.


\section{Applications Of Regex}
A variety of applications for regular expressions is explored in this section.

\subsection{End user applications}
\paragraph{Find and replace utility}
The task for which regular expressions were first implemented is finding and replacing strings in blocks of text.  This remains a central application for regular expressions, which programmers can use in a surprising number of ways to save effort.  Utilities provided by text editors such as Emacs, Notepad++, Sublime Text and Eclipse include: incremental find and replace within a single file, batch find and replace within a single file or group of files, highlighting matched text and counting the number of matching strings.  Emacs also provides utilities that delete all lines or retain all lines containing a match, and aligning columns by a regex-defined delimiter.

\paragraph{System administration}
System administrators and power users rely on command-line utilities to accomplish complex computing tasks, often dealing with file names and the contents of configuration files.  The output of one utility can be used as the input for another utility using \emph{pipes}, and the mini-programs written using piped commands often rely on regular expressions to filter or transform strings in one step or another.  For example, {\tt grep} allows users to find strings that match a regex, and {\tt sed} provides a replace functionality.  The {\tt find} utility searches filenames based on a regex, and tools like {\tt git} and {\tt cron} use regexes written in configuration files (`.gitignore' and `crontab', respectively) to specify sets of files or sets of dates and times.

\paragraph{Searching fields in relational databases}
The popular SQL query language\todoLast{Donald D. Chamberlin and Raymond F. Boyce} uses the `LIKE' operator and an exotic regular expression syntax (\verb!`%'! for zero-or-more repetition, \verb!`_'! to match any character, and typical character classes using brackets) to search fields for strings.  Modern relational database systems have expanded this syntax considerably with functions such as {\tt REGEXP\_LIKE} in Oracle, {\tt REGEXP} in MySQL, {\tt\$regex} in MongoDB and {\tt regexp\_replace} in PostrgreSQL.


\subsection{Research and industry applications}
\paragraph{Meta-programming}  Regular expressions are central to YACC and LEX, which are critical compiler tools for generating parsers used in the compilation process and lexing source files, respectively.  So here regex are used as a meta-programming language specifying the behavior of a parser.  Regular expressions have also been used for test case generation~\cite{Ghosh:2013:JAT:2486788.2486925, Galler:2014:STD:2683035.2683100, Anand:2013:OSM:2503903.2503991, Tillmann:2014:TAT:2642937.2642941},  and as specifications for string constraint solvers~\cite{Trinh:2014:SSS:2660267.2660372, hampi}.  Some data mining frameworks use regular expressions as queries (e.g., ~\cite{Begel:2010:CDE:1806799.1806821}).

\paragraph{Mission-critical domains}  Regexes are employed in MySQL injection prevention~\cite{Yeole:2011:ADT:1980022.1980229} and network intrusion detection~\cite{network}, and in more diverse applications like DNA sequencing alignment~\cite{1594922} or querying RDF data~\cite{Lee:2010:PSQ:1871871.1871877, Alkhateeb:2009:ESR:1540656.1540975}.  Efforts have also been made to expedite the processing of regular expressions on large bodies of text~\cite{Baeza-Yates:1996:FTS:235809.235810}.

\subsection{Composition and analysis tools}

\paragraph{Composition tools} Regular expression understandability has not been studied directly, though prior work has suggested that regexes are hard to read and understand since there are tens of thousands of bug reports related to regular expressions~\cite{Spishak:2012:TSR:2318202.2318207}.  Due in part to their common use across programming languages and how susceptible regexes are to error, many researchers and practitioners have developed tools to support more robust regex creation~\cite{Spishak:2012:TSR:2318202.2318207} or to allow visual debugging~\cite{Beck:2014:RVD:2591062.2591111}.

Tools have also been developed to make regexes easier to understand, and many are available online.  Some tools will, for example, highlight parts of regex patterns that match parts of strings as a tool to aid in comprehension.\footurl{https://regex101.com/}  Other tools allow users to compose regular expressions using natural language\footurl{https://github.com/VerbalExpressions/PHPVerbalExpressions}.  Building on the perspective that regexes are difficult to create, other research has focused on removing the human from the creation process by learning regular expressions from  text~\cite{Babbar:2010:CBA:1871840.1871848, Li:2008:REL:1613715.1613719}.

\paragraph{Analysis tools} Research tools like Hampi~\cite{hampi}, and Rex~\cite{rex}, and commercial tools like brics~\cite{brics} all support the use of regular expressions in various ways. Hampi was developed  in academia and uses regular expressions as a specification language for a constraint solver. Rex was developed by Microsoft Research and generates strings for regular expressions that can be used in  applications such as test case generation~\cite{Anand:2013:OSM:2503903.2503991, Tillmann:2014:TAT:2642937.2642941}. Brics is an open-source package that creates automata from regular expressions for manipulation and evaluation. Automata.Z3\footurl{https://github.com/AutomataDotNet/Automata} is one of a suite of tools developed by Microsoft to analyze regular expressions.


\section{Similar Research}

\subsection{Mining and surveys for language feature analysis}
Mining properties of open source repositories is a well-studied topic, focusing, for example, on API usage patterns~\cities{Linares-Vasquez:2014:MEA:2597073.2597085} and bug characterizations~\cities{Chen:2014:ESD:2597073.2597108}.
Exploring language feature usage by mining source code has been studied extensively for
Smalltalk~\cities{Callau:2011:DUD:1985441.1985448, Callau:2013:DUD:2589712.2589718},
JavaScript~\cities{Richards:2010:ADB:1809028.1806598},
and Java~\cities{Dyer:2014:MBA:2568225.2568295, Grechanik:2010:EIL:1852786.1852801, Parnin:2013:AUJ:2589712.2589717, Livshits:2005:RAJ:2099708.2099724},
and more specifically,
Java generics~\cities{Parnin:2013:AUJ:2589712.2589717} and
Java reflection~\cities{Livshits:2005:RAJ:2099708.2099724}.
To the author's knowledge, this is the first work to mine and evaluate regular expression usages from existing software repositories.
Surveys have been used to measure adoption of various programming languages~\cities{Meyerovich:2013:EAP:2509136.2509515, Dattero:2004:PLG:962081.962087}, and been combined with  repository analysis~\cities{Meyerovich:2013:EAP:2509136.2509515}, but have not focused on regexes.

\subsection{Refactoring and smells}
Regular expression refactoring has also not been studied directly, though refactoring literature abounds~\cities{Mens:2004:SSR:972215.972286, Opdyke:1992:ROF:169783, Griswold:1993:AAP:152388.152389}.
The closest to regex refactoring comes from research toward  expediting the processing of regular expressions on large bodies of text~\cities{Baeza-Yates:1996:FTS:235809.235810}, which could be thought of as refactoring for performance.

In software, code smells have been found to hinder understandability of source code \\*\cities{abbes2011empirical, du2006does, Hermans2016}.
Once removed through refactoring, the code becomes more understandable, easing the burden on the programmer.
In regular expressions, such code smells have not yet been defined, perhaps in part because it is not clear what makes a regex smelly.

Code smells in object-oriented languages were introduced by Fowler~\cities{Fowl1999}. Researchers have studied the impact of code smells on program comprehension~\cities{abbes2011empirical, du2006does}, finding that the more smells in the code, the harder the comprehension. This is similar to the work in this thesis, except we aim to identify which  regex representations can be considered smelly.
Code smells have been extended to other language paradigms including end-user programming languages~\cities{Hermans2012intra, Hermans2012intraExt, stoleeicse, stoleeTSE}. The code smells identified in this work are representations that are not common or not well understood by developers. This concept of using community standards to define smells has been used in other refactoring literature  for end-user programmers~\cities{stoleeicse, stoleeTSE}.




\chapter{CONCLUSION}
\section{Summary of contributions}








\balance

\section*{Acknowledgements}
This work is supported in part by  NSF SHF-EAGER-1446932.


% \bibliographystyle{IEEEtran}
% \bibliography{biblio,stolee}

% \end{document}

