\specialchapt{ABSTRACT}

%Regular expressions are ubiquitous in programming languages and tools, providing searching and text modifying functionality that could not be done easily any other way.
%Several major research projects have produced tools aimed at supporting reasoning, verification and automated testing for code using regular expressions.  Other projects have studied the logic of regular expressions for certain niche uses like malicious packet detection, etc etc,
Though regular expressions (regex) are baked into every major language, have inspired several tools and research projects, and have been around since the first days of Unix (1960?), no one has ever formally studied how they are used in practice, or what can be done to make them easier to understand.  This thesis presents the original work of studying a sample of regex taken from Python projects pulled from Github, determining what features are used most often, defining some categories that illuminate common use cases, and identifying areas of significance for tool builders.  Furthermore, this thesis defines an equivalence class model used to explore comprehension of regex, identifying the most common and most understandable representations of semantically identical regex, suggesting several refactorings and preferred representations.  Opportunities for future work include the novel and rich field of regex refactoring, semantic search of regexes, and further fundamental research into regex usage and understandability.
%This work also presents the results of a developer survey that investigates many questions related to regex use, including what the major pain points are.  The two major pain points discovered are difficulties in composing regex and in understanding regex composed by others.  Though many tools exist to help with composing regex, very little work has been done to make regex more understandable.  This work presents three techniques used to identify comprehension problems associated with specific regexes.  Furthermore, a set of equivalence classes was identified so that within each equivalence class, several
