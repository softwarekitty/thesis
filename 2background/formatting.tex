\section{Formatting And Feature Acronyms}
To reduce confusion due to typesetting issues, and to avoid repeatedly qualifying quoted text with phrases like `the string' or `the regex', characters will be surrounded in single quotes like \verb!`c'!, strings will be surrounded by double quotes like \verb!"example string"!, and regexes will be presented within a grey box without any quotes like \cverb!a+b*(c|d)e\1f!.  Pattern fragments used to represent feature tokens (usually to provide an example of what an acronym means), appear in parenthesis with a light grey background, like \bverb![^...]!.

All strings in this thesis are `raw' strings.  This means that a string presented in this thesis like \verb!"a\dc"! would actually be \verb!"a\\dc"! in source code, and a single slash in a regex appears as \cverb!\\! where the pattern in source code is \verb!"\\\\"!.  Invisible characters such as newline will be represented within strings in gray, like \verb!"first line.!\gverb!\n!\verb!second line."!.

This thesis discusses the features used by regular expressions in depth.  To facilitate this discussion, every feature is assigned an acronym composed of two to four capitol letters.  For example, the Kleene star feature, \bverb!*!, representing zero or more of some element, is referred to using the acronym `KLE'.  A concise presentation of the 34 features this thesis focuses on is presented in Table~\ref{table:featureDescriptions}.  A detailed description of all these features is provided in Appendix~\ref{app:featureDescriptions}.  All other features mentioned in this thesis, but not studied in detail are briefly described in Appendix~\ref{app:unrankedDescriptions}.
