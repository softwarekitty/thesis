% Chapter 1 of the Thesis Template File
\chapter{OVERVIEW}

This is the opening paragraph to my thesis which
explains in general terms the concepts and hypothesis
which will be used in my thesis.

With more general information given here than really
necessary.

\section{Introduction From Features}

Regular expressions (regexes) are an abstraction of keyword search that enables the identification of text using a pattern instead of an exact keyword.
Regexes are commonly used for parsing text using a general purpose language like Python, validating content entered into web forms using Javascript, and searching text files for a particular pattern using tools like grep, vim or Eclipse.

Although regexes are powerful and versatile, they can be hard to understand,  maintain, and debug, resulting in tens of thousands of bug reports~\cite{Spishak:2012:TSR:2318202.2318207}.

Due in part to their common use across programming languages and how susceptible regexes are to error, many researchers and practitioners have developed tools to support more robust regex creation~\cite{Spishak:2012:TSR:2318202.2318207} or to allow visual debugging~\cite{Beck:2014:RVD:2591062.2591111}. Other research has focused on learning regular expressions from  text~\cite{Babbar:2010:CBA:1871840.1871848, Li:2008:REL:1613715.1613719}, avoiding human composition altogether.
Researchers have also explored applying regexes to test case generation~\cite{Ghosh:2013:JAT:2486788.2486925, Galler:2014:STD:2683035.2683100, Anand:2013:OSM:2503903.2503991, Tillmann:2014:TAT:2642937.2642941},
as specifications for string constraint solvers~\cite{Trinh:2014:SSS:2660267.2660372, hampi} and using regexes as queries in a data mining framework~\cite{Begel:2010:CDE:1806799.1806821}.
Regexes are also employed in critical missions like MySQL injection prevention~\cite{Yeole:2011:ADT:1980022.1980229} and network intrusion detection~\cite{network}, or in more diverse applications like DNA sequencing alignment~\cite{1594922}.

Regex researchers and tool designers must pick what features to include or exclude, which  can be a difficult  design decision. Supporting advanced features may be more expensive, taking more time and potentially making the project too complex and cumbersome to execute well.  A selection of only the simplest of regex features limits the applicability or relevance of that work. Despite extensive research effort in the area of regex support,  no research has been done about how regexes are used in practice and what features are essential for the most common use cases.


\emph{The goal of this work is to explore 1) the context in which developers use regular expressions, and 2) the features and similarities of  regular expressions found in Python\footnote{Python is the fourth most common language on GitHub (after Java, Javascript and Ruby) and  Python's regex pattern language is close enough to other regex libraries that our conclusions are likely to generalize.} projects}.

First, we survey professional developers about how they use regexes and their pain points.  Second, we gather a sample of regexes from Python projects and analyze the frequency of feature usage (e.g., kleene star: \verb!*! and the end anchor: \verb!$! are features).    Third, we investigate what features are supported by four large projects that aim to support regex usage (brics~\cite{brics}, hampi~\cite{hampi}, Rex~\cite{rex}, and RE2~\cite{re2}), and which features are not supported, but are frequently used by developers.  Finally, we cluster regular expressions that appear in multiple projects by behavior, investigating high-level behavioral themes in regex usage.

Our results indicate that regexes are most frequently used in command line tools and IDEs.    Capturing the contents of brackets and searching for delimiter characters were some of the most apparent  behavioral themes observed in our regex clusters, and developers frequently use regexes to parse source code.
The contributions of this work are:
\begin{itemize} \setlength \itemsep{.1pt}
    \item A survey of 18 professional software developers about their experience with regular expressions,
    \item An empirical analysis of regex feature usage in nearly 14,000 regular expressions in \dbfetch{nProjScanned} open-source Python projects, mapping of those features to those supported by common regex tools and survey results showing the impact of not supporting various features,
    \item An approach for measuring behavioral similarity of regular expressions and qualitative analysis of the most common behaviorally similar clusters, and
    \item An evidence-based discussion of opportunities for future work in supporting programmers who use regular expressions, including refactoring regexes, developing regex similarity analyses, and providing migration support between languages.
\end{itemize}

\section{Introduction From Refactoring}

Regular expressions are used frequently by developers for many purposes, such as parsing files, validating user input, or querying a database.
Regexes are also employed in MySQL injection prevention~\cite{Yeole:2011:ADT:1980022.1980229} and network intrusion detection~\cite{network}.
However, recent research has suggested that regular expressions (regexes) are hard to understand, hard to compose, and error prone~\cite{Spishak:2012:TSR:2318202.2318207}.
Given the difficulties with working with regular expressions and how often they appear in software projects and processes, it seems fitting that efforts should be made to ease the burden on developers.

Tools have been developed to make regexes easier to understand, and many are available online.
Some tools will, for example, highlight parts of regex patterns that match parts of strings as a tool to aid in comprehension.\footurl{https://regex101.com/}
Others will automatically generate strings that are matched by the regular expessions~\cite{hampi}.
Other tools will automatically generate regexes when given a list of strings to match~\cite{Babbar:2010:CBA:1871840.1871848, Li:2008:REL:1613715.1613719}.
The commonality of such tools provides evidence that people need help with regex composition and understandability.

In software, code smells have been found to hinder understandability of source code~\cite{abbes2011empirical, du2006does}.
Once removed through refactoring, the code becomes more understandable, easing the burden on the programmer.
In regular expressions, such code smells have not yet been defined, perhaps in part because it is not clear what makes a regex smelly.

\subsection{Hypothesis}

\todoMid{Do we want to have these sections below like hypothesis, etc?}

Here one particular hypothesis is explained in depth
and is examined in the light of current literature.

\subsubsection{Parts of the hypothesis}

Here one particular part of the hypothesis that is
currently being explained is examined and particular
elements of that part are given careful scrutiny.

\subsection{Second Hypothesis}

Here one particular hypothesis is explained in depth
and is examined in the light of current literature.

\subsubsection{Parts of the second hypothesis}

Here one particular part of the hypothesis that is
currently being explained is examined and particular
elements of that part are given careful scrutiny.

\section{Criteria Review}

Here certain criteria are explained thus eventually
leading to a foregone conclusion.
