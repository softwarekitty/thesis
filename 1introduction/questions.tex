\subsection{Research questions}

Although regex have provided an essential search functionality for software development for half a century, are essential to parsing, compiling, security, database queries and user input validation, and are incorporated into all but the most low-level programming languages, no fundamental research has been published investigating use cases, measuring feature usage or determining optimal representations for understandability.  Faced with an open field, these four questions were formulated to begin the work of filling this fundamental knowledge gap.  The following section articulates the motivations behind the questions explored in this thesis.

\subsubsection{RQ1: How are regex used in practice, especially what features are most commonly used?}

The features that allow regex users to compactly represent sets of strings are what power regular expressions.  Gathering fundamental statistics about what features are used can inform many other issues in regular expression research, such as language and tool design.

\subsubsection{RQ2: What behavioral categories can be observed in regex?}

If a well-fitting categorization scheme for regex behavior can be devised, these categories can provide insight into what users are really doing with regexes and in turn, what behaviors are most important for future regex technologies.

\subsubsection{RQ3: Within five equivalence classes, what representations are most frequently observed?}

There are many ways to represent the same functional regex, that is, the user has choices to make about how to compose a regex for any given task.  Assuming that regex composers will tend to choose the best representation most of the time, what representations are chosen?

\subsubsection{RQ4: Within five equivalence classes, what representations are more comprehensible?}

Regexes in source code must be understood in order to be properly maintained, but regexes can be hard to understand.  If the most understandable representation for a particular class of regular expressions can be determined, then the understandability of regexes can be increased through refactoring, easing the burden on maintainers.
