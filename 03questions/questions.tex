\chapter{RESEARCH QUESTIONS}

\section{Gap in fundamental research into regex use in practice}

\section{Questions explored in this thesis and their motivations}

\subsection{RQ1: How are regex used in practice, especially what features are most commonly used?}

Regex researchers and tool designers must pick what features to include or exclude, which  can be a difficult  design decision. Supporting advanced features may be more expensive, taking more time and potentially making the project too complex and cumbersome to execute well.  A selection of only the simplest of regex features limits the applicability or relevance of that work. Despite extensive research effort in the area of regex support,  no research has been done about how regexes are used in practice and what features are essential for the most common use cases.

\subsection{RQ2: What preferences, behaviors and opinions do professional developers have about using regex?}

\subsection{RQ3: What behavioral categories can be observed in regex?}

\subsection{RQ4: Within five equivalence classes, what representations are most frequently observed?}

\subsection{RQ5: What representations are more comprehensible?}

\subsection{RQ6: For each equivalence class, which representation is preferred according to frequency and comprehensibility?}

After defining the equivalence classes and potential  regex refactorings as described in Section~\ref{sec:refactoring}, we wanted to know which representations in the equivalence classes  are considered desirable and which might be smelly. Desirability for regexes can be defined many ways, including maintainable,  understandable, and performance.
We focus on refactoring for understandability.
