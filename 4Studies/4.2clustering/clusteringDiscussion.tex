\subsection{Discussion of cluster categories}
\subsubsection{Implications}
When tool designers are considering what features to include, data about usage in practice is valuable.  Behavioral similarity clustering  helps to discern these behaviors by looking beyond the structural details of specific patterns and seeing trends in  matching behavior. We are also able to find out what features are being used in these behavioral trends so that we can make assertions about why certain features are important.
We used the behavior of individual patterns to form clusters, and identified six main categories for the clusters.
 Overall, we see that many clusters are defined by the presence of particular tokens, such as the colon for the cluster in Table~\ref{table:exampleCluster}.
We identified six main categories that define regex behavior at a high level: matching with alternatives, matching literal characters, matching with sequences, matching with endpoint anchors, parsing contents of brackets or braces, or searching and capturing code, and can be considered in conjunction with the self-described regex activities from the survey in Table~\ref{table:regextasks} to be representative of common uses for regexes.
One of the six common cluster categories, \emph{Code Search and Variable Capturing}, has a very specific purpose of parsing source code files. This shows a very specific and common use of regular expressions in practice.

\paragraph{Finding Specific Content}
Two categorical clusters, \emph{Specific Characters Must Match} and \emph{Two or More Characters in Sequence}, deal with identifying the presence of specific character(s).
While multiple character matching subsumes single character matching, the overarching theme is that these regexes are looking to validate strings based on the presence of very specific content, as would be done for many common activities listed in Table~\ref{table:regextasks}, such as, ``Locating content within a file or files."
More study is needed into what content is most frequently searched for, but from our cluster analysis we found that version numbers, twitter or user handles, hex values, decimal numbers, capitalized words, and particular combinations of whitespace, slashes and other delimiters were discernible targets.

Capturing the contents of brackets and searching for delimiter characters were some of the most apparent  behavioral themes observed in our regex clusters, and developers frequently use regexes to parse source code.

%taking break at line 1200

% \todoMid{Again, it might be interesting to look at what particular sequences are looking like.  I think I mention this again in the discussion, but should we put it here instead?}
% \todoMid{The thing I want to mention about anchored patterns (but have struggled to say in the past) is that they are the only way to guarantee that a character does not appear in a particular location by specifying what is allowed.  Consider the regex }
% \verb!^[-_A-Za-z0-9]+$!
% \todoMid{ which will fail to match if an undesirable character like `?' appears anywhere in the input.  In logic, there is a similar phenomenon.  That is, `Always' is true iff `Not Exists' of the negation is true, and by requiring an entire input to always maintain some abstraction, you can indirectly specify the negation of another (inverse) abstraction.  Even with only one anchor point, a regex like }
% \verb!.*[0-9]$!
% \todoMid{ is creating an ultimatum about the end being a digit.  Without the endpoint anchors, I don't see how one could specify absolutes about an input. }


\subsubsection{Opportunities for future work}
\subsubsection{Threats to validity}
why not k-means?
