\subsection{Markov clustering}
\subsubsection{Background}
\subsubsection{Tuning parameters}
\subsubsection{Results}

From 2,871 distinct patterns, MCL clustering identified 186 clusters with 2 or more patterns, and 2,042 clusters of size 1.
 The average size of clusters larger than size one was 4.5.  Each pattern belongs to exactly one cluster.

Three example strings generated by Rex for the first pattern are: `-()', `*'8(5)', `Oe()'.  For the third pattern, Rex generated these three strings: ` ()', `(q)F', `(n)M'.  The pattern: \verb!\(.*\)$! is very similar, but will not match the string `(n)M', and so was placed in a different cluster.

\begin{table}
\begin{center}
\caption{An example cluster containing 12 regexes, with at least one regex present in 31 different projects.  In this cluster, every regex requires `:'.}
\label{table:exampleCluster}
\begin{small}
\begin{tabular}
{lcc | lcc}
\toprule \bigstrut
\textbf{Index} & \textbf{Pattern} & \textbf{NProjects} & \textbf{Index} & \textbf{Pattern} & \textbf{NProjects} \\
 \midrule \bigstrut
1 & \begin{minipage}{1.6in}\cverb!\s*([^: ]*)\s*:(.*)!\end{minipage} & 9 & 7 & \begin{minipage}{1.6in}\cverb![:]!\end{minipage} & 6 \\
 \midrule \bigstrut
2 & \begin{minipage}{1.6in}\cverb!:+!\end{minipage} & 8 & 8 & \begin{minipage}{1.6in}\cverb!([^:]+):(.*)!\end{minipage} & 6 \\
 \midrule \bigstrut
3 & \begin{minipage}{1.6in}\cverb!(:)!\end{minipage} & 8 & 9 & \begin{minipage}{1.6in}\cverb!\s*:\s*!\end{minipage} & 4 \\
 \midrule \bigstrut
4 & \begin{minipage}{1.6in}\cverb!(:+)!\end{minipage} & 8 & 10 & \begin{minipage}{1.6in}\cverb!\:!\end{minipage} & 2 \\
 \midrule \bigstrut
5 & \begin{minipage}{1.6in}\cverb!(:)(:*)!\end{minipage} & 8 & 11 & \begin{minipage}{1.6in}\cverb!^([^:]*):[^:]*$!\end{minipage} & 2 \\
 \midrule \bigstrut
6 & \begin{minipage}{1.6in}\cverb!^([^:]*): *(.*)!\end{minipage} & 8 & 12 & \begin{minipage}{1.6in}\cverb!^[^:]*:([^:]*)$!\end{minipage} & 2 \\
\bottomrule
\end{tabular}
\vspace{-6pt}
\end{small}
\end{center}
\vspace{-12pt}
\end{table}


Table~\ref{table:exampleCluster} provides an example of a behavioral cluster containing 12 patterns (four longer patterns omitted for brevity). Patterns from this cluster are present in 31 different projects.  All patterns in this cluster share the literal `:' character. The smallest pattern, \verb!`:+'!,  matches one or more colons.

% \begin{figure}[tb]
% \centering
% \includegraphics[width=\columnwidth]{nontex/illustrations/clusterEdgesExample.eps}
% \vspace{-12pt}
% \caption{Example Of Similarity Edges Of One Cluster}
% \vspace{-6pt}
% \label{fig:clusterEdgesExample}
% \end{figure}

Another pattern from this cluster, \verb!([^:]+):(.*)!, requires at least one non-colon character to occur before a colon character.  Our similarity value between these two regexes was below the minimum of 0.75 because Rex generated many strings for `:+' that start with one or more colons.
We observe that the smallest pattern in a cluster provides insight about key characteristic that all the patterns in the cluster have in common.  A shorter pattern will tend to have less extraneous behavior because it is specifying less behavior,
yet, in order for the smallest pattern to be clustered, it had to match most of the strings created by Rex from many other patterns within the cluster, and so we observe that {the smallest pattern is useful as a representative of the cluster}.

For the rest of this paper, a cluster will be represented by one of the shortest patterns it contains, followed by the number of projects any member of the cluster appears in, so the cluster in Table~\ref{table:exampleCluster} will be represented as \verb!`:+'(31)!.  This representation is not an attempt to express all notable behavior of patterns within a cluster, but is a useful and meaningful abbreviation.
Other regexes in the cluster may exhibit more diverse behavior, for example the pattern \verb!`([^: ]+):(.*)'! requires a non-colon character to appear before a colon character.
