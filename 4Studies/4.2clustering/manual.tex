\subsubsection{Implementation details for manual categorization}
The following categorization scheme was created after three visual scans of the top 3000 regexes (by number of projects), and eight iterative manual categorization passes over the top 130 regexes:

\subsection{20 refined categories}
\label{sec:20refinedCategories}

\paragraph{Commonly observed categories.} In this work, regexes will belong to the first category in the ordered list described here (categories are greedy).  Very long regexes (over 100 chars long) are treated as their own category without further investigation (unless recognized very easily), because their behavior is significantly complex so as not to properly belong to any other category.  From behavioral clustering, it was determined that web-based, bracket-parsing and code-based categories cluster strongly.  For other regexes, the degree of freedom limits the clustering technique: regexes composed of long strings with very little abstraction are not free to cluster by conceptual category because their string matching behavior rarely overlaps.  Some regexes contain recognizable words but also contain substantial abstract sections, usually to parse a file or a line representing a row of data.  Some regular patterns like dates, phone numbers and emails don't contain words but have a recognizable form.  A large portion of regexes contain just one or two characters, or character classes that may be used for de-serialization or delimiter-based work, but are hard to guess the exact application for.  In these free-clustering regexes, anchors often make a big difference, and will be considered at the same time.  Finally, a few catch-all categories for regexes that are difficult to guess the application for are used.  The first category is questions, which have a regularity the purpose of which is hard to understand such as \cverb![a-f0-9]{40}!.  , vanilla, such as \cverb![^\w\s-]!, \cverb!\d+!, \cverb!! or \cverb!! which


\begin{itemize}
anchors are ignored for less free regexes

zero freedom
\item[ r ] repeating        simple language of repeating sequence like \cverb!a+! (no defaults)
\item[ u ] Unicode          using \cverb!\x[a-f0-9][a-f0-9]!, octal or unicode
\item[ f ] file             recognize a file filename with an extension

web based
\item[ h ] html      scanning for known HTML tags like \cverb!<div>.*!
\item[ x ] xml       non-HTML tags usually composed of whole words or with attributes
\item[ w ] web       parsing urls, (possibly local) paths, html escape chars, IP addresses and web protocols

code based
\item[ = ] operator     recognize a line of code (has `=', `+' or operators) without capturing
\item[ m ] message      a phrase containing variable information
\item[ g ] args         recognize a shell command or arguments (usually with arguments)
\item[ c ] code         recognize source code contents like 'include' or `std::bitset' or functions like out.println() without `='
\item[ b ] brackets  capturing or matching all content within brackets like \cverb!<.*?>! or \cverb!<[^>]*?>!
\item[ l ] label        a phrase that could be used to find labeled data

mostly-free-clustering regular strings

\item[ d ] dates         parsing a simple string code like a date format
\item[ n ] numbers       parsing a phone number, ssn, numeric date, regular number
\item[ i ] identifier    identifiers (alphanumeric) with a restriction (must be capitol first, have a `@'), or dealing with word-default variants - functionally is also a tuple.  There is a fuzzy line between a message and an identifier - a message is just an identifier with phrases in place of rules

free-clustering strings
anchored prefix for fre-clustering strings a, so \cverb!\s*! is s for `space', but \cverb!^\s*$! is `as' because of \cverb!^! and \cverb!$! (only needs one anchor to require an anchor prefix).
\item[ \\ ] backslash    dealing mostly with backslashes like \cverb!\\(.)! - easy to recognize because the pattern for this will contain four backslashes: \verb!"\\\\"!
\item[ s ] space         whitespace only

\item[ p ] punctuation   contains punctuation (not just slashes), maybe with whitespace like, \cverb!\s*,\s*!
\item[ o ] ordinary         simple one or a handful of chars, like [aeiou] or (a|b)


catch-all (no anchor prefix)
\item[ L ] LONG        too long to deal with, standard is 140 chars (a tweet)
\item[ q ] question    effective regular pattern but does not fit into any category
\item[ v ] vanilla     a regex is vanilla if it would be hard not to match it, like \cverb!\w+!
\end{itemize}

The logical order is arranged so that special knowledge domains, like code keywords take precedence - this is so that the portion of the population that is using regexes in a particular way can be observed (not lumped into a more general category).  This guiding principal leads to the following order of identified categories (i.e., if more than one category could be used, which to use):

disqualified for categorization: 2
\begin{itemize}
\item[ L ] LONG        too long to deal with, standard is 140 chars (a tweet)
\item[ u ] Unicode          using \cverb!\x[a-f0-9][a-f0-9]!, octal or unicode
\end{itemize}

containers: 5
\begin{itemize}
\item[ c ] known programming language keywords and/or syntax in optional code fragment, maybe containing any of the following
\item[ g ] shell commands or arguments, indicated by the -arg syntax and tool keywords, or emphasizing `-' alone for this purpose
\item[ h ] requires valid HTML tags or attributes like <div>,  href=...
\item[ x ] requires populated XML-like tags or attributes, including non-html assignments like shoe=running
\item[ = ] expresses some operator or assignment syntax, language not known (or it would be a c, g or h)
\end{itemize}

specialized regular forms: 4
\begin{itemize}
\item[ / ] forward-slashy paths, including local and web paths, and protocol or file system prefixes like https: or , and patterns emphasizing the `/' character alone for this purpose.  Theoretically this could contain html or xml tags, but html containing slashy paths should be categorized as h, and that seems more likely.
\item[ f ] deals with known file extensions, or apparent ones prefixed by a period, or patterns emphasizing the `.'  alone for this purpose.  This includes .c, .xml and .html if not used in containing code.
\item[ t ] requires a syntax representing time, like a date format
\item[ n ] numbers - in a syntax like telephone numbers, scientific numbers, or just \\d emphasized with the intent to capture a free number
\end{itemize}

generalized regular forms: 4
\begin{itemize}
\item[ i ] identifiers - a non-free, rule bound section and optional label/delim .  These are specialized and used specifically to identify a user, part number, library book, etc.
\item[ m ] messages requires a label or delimiter section, and a free section, which often is captured (including in brackets also parens, etc.).  This includes things like `your file is (.*)\\.py' which have a filename, but are not exclusively filenames
\item[ b ] contains bracket contents parsing, including curly, parenthesis, angle and square brackets
\item[ > ] requires some SGML <angle bracket language syntax>, like bracket comments <!-- --!>, or opening or closing brackets, including the single bracket characters `<' and `>'
\end{itemize}


specialized exact: 2
\begin{itemize}
\item[ l ] label  a literal sequence of more than one character, like `oh hai', and nothing else
\item[ s ] space  requires one or more whitespace, including newlines, returns, tabs and invisibles, and nothing else
\end{itemize}


specialized inexact: 3
\begin{itemize}
\item[ \\ ] backslash  requires a finite number of literal backslashes, which when escaped looks like \verb!"\\\\"!
\item[ d ] delimiter requires one or a finite number of punctuation chars.  The intuition is that these can be used to locate and potentially separate tuples, or expected data.
\item[ o ] ordinary   requires a finite number of ordinary chars, like [aeiou] or (a|b){3,4}
\end{itemize}


generalized inexact: 2
\begin{itemize}
\item[ v ] vanilla - a regex is vanilla if it would be hard not to match it, like \cverb!\w+!, \cverb!.*!, \cverb!\S!, \cverb![0-9A-z-*&^#@&,./]!.  Optionally repeats.  CCCs with 10 or fewer chars are no longer vanilla.
\item[ r ] repeating - has repetition like \cverb!;+! or \cverb!a[efg]*!.  Because this is so low, it should not contain digits, space or word char classes, or large char classes.
\end{itemize}

undetermined: 1
\begin{itemize}
\item[ q ] question    substantial regular pattern but does not fit into any category
\end{itemize}















