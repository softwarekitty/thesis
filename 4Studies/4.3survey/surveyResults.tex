\subsection{Summary of survey results}
\label{sec:surveyResults}

The survey was sent to 22 employees at Dwolla, and 20 attempted a response.  Two employees were eliminated by responding `no' when asked if they were a professional software developer that had used regex in a work environment.  The remaining 18 participants (82\% response rate) have an average of nine years of programming experience ($\sigma = 4.28$).

\subsubsection{Contextual usage frequency}
\label{sec:compositionFrequency}
To increase accuracy of recall, participants were first prompted to recall per-environment, and per-task before asking for the overall number of regexes composed per year.  On average, survey participants report to compose 172 regexes per year ($\sigma$ = 250) and compose regexes on average once per month, with 28\% composing multiple regexes in a week and an additional 22\% composing regexes once per week. That is, 50\% of respondents uses regexes at least weekly.

\begin{table}[ht]
\caption{Survey results for number of regexes composed per year by technical environment\label{tab:regexenviron}}
\begin{center}
\begin{small}
\begin{tabular}{l | r @{  \horiz} r @{ \horiz } r @{ \horiz } r @{ \horiz } r @{ \horiz } r }
\toprule
\textbf{Language/Environment} & 0 & 1-5 & 6-10 & 11-20 & 21-50 & 51+ \\  \midrule \bigstrut
General  (e.g., Java)  & 1 & 6 & 5 & 3& 1& 2 \\ \midrule \bigstrut
Scripting  (e.g., Perl) &5 &4 &3 &3 &2  &1 \\ \midrule \bigstrut
Query  (e.g., SQL) & 15&2 &0 &0 &1  & 0\\ \midrule \bigstrut
Command line (e.g., grep)   &2 &5 &3 &2 &0  &6 \\ \midrule \bigstrut
Text editor (e.g., IntelliJ)   & 2& 5& 0& 5& 1& 5\\
\bottomrule
\end{tabular}
\end{small}
\end{center}
\vspace{-12pt}
\end{table}


\paragraph{Usage frequency by technical environment} Table~\ref{tab:regexenviron} summarizes how frequently participants compose regexes using each of several languages and technical environments (complete response data is available in Appendix~\ref{table:surveyQ04}.)  Six (33\%) of the survey participants report to compose regexes using general purpose programming languages (e.g., Java, C, C\#) 1-5 times per year and five (28\%) do this 6-10 times per year.  For command line usage in tools such as grep, 6 (33\%) participants use regexes 51+ times per year. Yet, regexes were rarely used in query languages like SQL. Upon further investigation, it turns out the surveyed developers were not on teams that dealt heavily with a database.

\begin{table}
\caption{Survey results for regex usage frequencies for tasks, averaged using a 6-point likert scale: Very Frequently=6, Frequently=5, Occasionally=4, Rarely=3, Very Rarely=2, and Never=1 \label{tab:regextasks}}
\begin{center}
\begin{small}
\begin{tabular}{l|c}
\toprule
\textbf{Task} & \textbf{Frequency} \\  \midrule \bigstrut
Locating content within a file or files & 4.4\\ \midrule \bigstrut
Capturing parts of strings & 4.3 \\ \midrule \bigstrut
Parsing user input & 4.0\\ \midrule \bigstrut
Counting lines that match a pattern & 3.2\\ \midrule \bigstrut
Counting  substrings that match a pattern & 3.2\\  \midrule \bigstrut
Parsing generated text & 3.0\\  \midrule \bigstrut
Filtering collections (lists, tables, etc.) & 3.0 \\ \midrule \bigstrut
Checking for a single character & 1.7\\
\bottomrule
\end{tabular}
\end{small}
\end{center}
\vspace{-12pt}
\end{table}


\paragraph{Usage frequency by task category} Table~\ref{table:regextasks} summarizes how frequently, on average, the participants use
regexes for various tasks.
Participants answered questions using a 6-point likert scale including very frequently~(6), frequently~(5), occasionally~(4), rarely~(3), very rarely~(2), and never~(1).  Complete response data is available in Appendix~\ref{table:surveyQ05}.  Averaging across participants, among the most common usages are capturing parts of a string and locating content within a file, with both occurring somewhere between occasionally and frequently.

\begin{table}
\caption{Survey results for regex usage frequencies, averaged using a 6-point likert scale: Very Frequently=6, Frequently=5, Occasionally=4, Rarely=3, Very Rarely=2, and Never=1 (RQ3) \label{tab:regexfeaturegroups}}
\begin{center}
\begin{small}
\begin{tabular}{llc}
\toprule
\textbf{Group} & \textbf{Code} &  \textbf{Frequency} \\  \midrule \bigstrut
endpoint anchors & (STR, END) & 4.4\\ \midrule \bigstrut
capture groups & (CG) & 4.2 \\ \midrule \bigstrut
word boundaries & (WNW) & 3.5 \\ \midrule \bigstrut
lazy repetition & (LZY) &  2.9\\ \midrule \bigstrut
\multirow{2}{*}{(neg) look-ahead/behind} &  (LKA, NLKA,  & \multirow{2}{*}{2.5}\\
& LKB, NLKB) & \\
\bottomrule
\end{tabular}
\end{small}
\end{center}
\vspace{-12pt}
\end{table}


\subsubsection{Feature usage questions}

\paragraph{Rarely used OPT and DBB features}  When asked if they have ever used the OPT feature (\verb!(?i)!), 78\% (14) said that they had never used it, with the rest saying they had.  However the reverse is true for the DBB feature, with 78\% (14) saying they \emph{had} used it before, and the rest saying they had not.

\paragraph{Use of the WRD default}  A special investigation was made into the WRD default after a trend in the corpus was observed where, with some frequency, custom character classes are made that slightly modify the WRD default.  For example, the regex \cverb![\w-]! might be used to represent characters allowed in a username.
Participants were asked what they use the WRD default character class for, but the intention of the question may not have been clear enough, because 50\% (9) participants said they don't know or don't use it, and 44\% (8) participants said they use it to match WRD characters.  However, one participant said `often we pair it with other characters we know are found in traditional writing such as punctuation marks.'.

\paragraph{Features poorly supported by analysis tools} Participants were asked about how frequently they used five feature groups poorly supported by analysis tools.  Their responses are shown in Table~\ref{tab:regexfeaturegroups}, indicating that lazy repetition and look-ahead features are rarely used and capture groups and endpoint anchors are occasionally to frequently used. Complete response data is available in Appendix~\ref{table:surveyQ09T13}.

\begin{table}
\caption{Responses to survey Q18: If you know it won't affect your use case would you prefer to use '.*' or '.+' ? \label{table:KLEvsADD}}
\begin{center}
\begin{small}
\begin{tabular}{lc|c|c|c}
\toprule
&\textbf{Always KLE} & \textbf{Mostly KLE} & \textbf{Mostly ADD} & \textbf{Always ADD}\\  \midrule
& 7 & 5 & 4 & 2 \\
\bottomrule
\end{tabular}
\end{small}
\end{center}
\vspace{-12pt}
\end{table}


\subsubsection{Refactoring preference questions}
\paragraph{Pairing ANY with KLE or ADD}  Participants were asked if, given a guarantee that it would not affect their use case, they would prefer to use ANY with KLE (\verb!*!) or with ADD (\verb!+!).  Participants generally preferred to use KLE, with 39\% (7) saying `always use \cverb!.*!', 28\% (5) saying `use \cverb!.*! more than \cverb!.+!', 22\% (4) saying `use \cverb!.+! more than \cverb!.*!', and 11\% (2) saying  `always use \cverb!.+!'.  These results are displayed in Table~\ref{table:KLEvsADD}.
Semantically, there is a difference between these two, which triggered a rich, skeptical response in the section asking them to explain their preference.  Aside from this understandable skepticism, those who preferred ADD were referring to the expected content, and those who preferred KLE said that they are `used to using' KLE or `need \verb!*! more often'.

\begin{table}
\caption{Survey results for preferences between custom character and default character classes (RQ3) \label{tab:cccvsdefault}}
\begin{center}
\begin{small}
\begin{tabular}{l|c}
\toprule
\textbf{Preference} & \textbf{Frequency} \\  \midrule \bigstrut
use only CCC & 1\\ \midrule \bigstrut
use CCC more than default & 5 \\ \midrule \bigstrut
use both equally & 2\\ \midrule \bigstrut
use default more than CCC & 10\\ \midrule \bigstrut
use only default & 2\\
\bottomrule
\end{tabular}
\end{small}
\end{center}
\vspace{-12pt}
\end{table}


\paragraph{CCC vs defaults} Semantically equivalent regexes can be created using the CCC (\verb![...]!) feature, or using default character classes like DEC (\verb!\d!) (e.g, \cverb![0-9]! $\equiv$ \cverb!\d!).  Survey participants were asked if they use only CCC, use CCC more than default, use both equally, use default more than CCC or use only default.  Results for this question are shown in Table~\ref{tab:cccvsdefault}, with 67\% (12) indicating that they use defaults the most.
Participants who favored CCC indicated that ``it is more explicit," whereas the participants who favored default character classes said,  ``it is less verbose" and ``I like using built-in code."

\paragraph{Named vs Numbered Backreferences}  Participants were asked whether, assuming a backreference was needed, they would `always use numbered backreferences', `it depends', or  `always use named backreferences'.  No participants said that they would always use named backrefernces, with 33\% (6) participants saying `it depends', and 67\% (12) saying they would always use numbered backreferences.  For the three participants who responded to why `it depends', two said that the more captures they need, the more likely they are to use named backreferences.  The third response seemed to indicate that they have never needed to use backreferences.

\begin{table}[!htbp]
\centering
\begin{normalsize}
\label{table:codeVsRegexTest}
\caption{\small{Q5: Please describe how often you compose regex for a particular problem type. }}
\begin{tabular}{l|c|c|c|c|c|c|c}
\hline
 & \textbf{Always} & \textbf{V. Freq} & \textbf{Freq.} & \textbf{Occ.} & \textbf{Rarely} & \textbf{V. Rarely} & \textbf{Never} \\
\noalign{\hrule height 0.08em}
test code & 4 & 7 & 5 & 1 & 0 & 0 & 1\\
\hline
test regex & 3 & 4 & 5 & 5 & 1 & 1 & 0\\
\noalign{\hrule height 0.08em}
\end{tabular}
\end{normalsize}
\end{table}


\subsubsection{Best practices and pain points}
\paragraph{Code testing vs regex testing}\label{sec:surveyTesting}  Participants answered these questions using a 7-point likert scale using always (7) as a seventh point in addition to the the six points outlined in Section~\ref{sec:compositionFrequency}. Developers were asked how frequently they test their code, with 89\% (16) indicating that they test their code at least frequently.  Developers were also asked how frequently they test their regexes, with 89\% (16) indicating that they test their code at least occasionally.  These results are summarized in Table~\ref{table:codeVsRegexTest}. Comparing the testing frequency between regex and other code for each participant, 39\% (7) test code more frequently than regex, 44\% (8) test both with the same frequency, and 17\% (3) test regex more frequently than code.  Prior work suggests that regexes are hard understand, causing tens of thousands of reported bugs per year~\cities{Spishak:2012:TSR:2318202.2318207}.  This finding is reinforced by the trend in participant responses of testing code more than regex.

\paragraph{Regex testing tools} When asking if they use testing tools, 50\% (9) indicated that they use some tool, with 33\% (6) using an online tool like \url{regex101.com} where a regex and input string are entered, and the input string is highlighted according to what is matched.  The remaining 17\% (3) developers who used testing tools used Scalacheck, Regexlib and IDE plugins.
Of the 9 developers who did not use testing tools, 17\% (3) indicated that they write their own tests for their regexes, and 33\% (6) answered `none'.

\paragraph{Parsing HTML}\label{sec:surveyParsingHTML}   In some cases, parsing balanced delimiters using regexes is known to cause problems for programmers\footurl{http://stackoverflow.com/questions/1732348}, because (most) regular expressions cannot fully describe balanced delimiter rules. Participants were asked if they had ever tried to parse HTML or XML using regexes.  Participants were also asked if they would rather use a custom parser or a regex, and why.  Only 28\% (5) of the participants admitted to using regexes on HTML.  When asked if they would `use only custom parser' or `use only regex', or `it depends', 22\% (4) said they would favor the parser, and only 11\% (2) said they would favor regex.  The remaining 67\% (12) said `it depends'.  Of those who said it depends, there was much agreement that parsers are more powerful, better at handling complexity, and provide more readable code.  Regexes were mentioned positively for handling less structured inputs, form validation and `one-off' programs.

\paragraph{Pain points and free responses}\label{sec:painPoints}
Participants were asked an open-ended question about problems with regular expressions: `What pain points have you encountered with regular expressions?'.  Three main categories of response were observed. The most common, ``hard to compose," was represented in 61\% (11) responses. Next, 39\% (7) developers responded that regexes are ``hard to read" and 17\% (3) indicated difficulties with ``inconsistency across implementations," which manifest when using regexes in multiple languages. These responses do not sum to 18 as three developers provided multiple parts in their answers.  Complete response data is available in Appendix~\ref{table:surveyQ27}.  Participants were given a chance to add anything else that had not been asked about.  Many comments conveyed a sense of mixed feelings - that regex are useful but require caution.

