\section{Summary of survey results}

The survey was completed by 18 participants (82\% response rate) that identified as software developer/maintainers.
Respondents have an average of nine years of programming experience ($\sigma = 4.28$).
On average, survey participants report to compose 172 regexes per year ($\sigma$ = 250) and compose regexes on average once per month, with 28\% composing multiple regexes in a week and an additional 22\% composing regexes once per week. That is, 50\% of respondents uses regexes at least weekly.
Table~\ref{tab:regexenviron} shows how frequently participants compose regexes using each of several languages and technical environments.
Six (33\%) of the survey participants report to compose regexes using general purpose programming languages (e.g., Java, C, C\#) 1-5 times per year and five (28\%) do this 6-10 times per year.  For command line usage in tools such as grep, 6 (33\%) participants use regexes 51+ times per year. Yet, regexes were rarely used in query languages like SQL. Upon further investigation, it turns out the surveyed developers were not on teams that dealt heavily with a database.

\begin{table}[ht]
\caption{Survey results for number of regexes composed per year by technical environment\label{tab:regexenviron}}
\begin{center}
\begin{small}
\begin{tabular}{l | r @{  \horiz} r @{ \horiz } r @{ \horiz } r @{ \horiz } r @{ \horiz } r }
\toprule
\textbf{Language/Environment} & \textbf{0} & \textbf{1-5} & \textbf{6-10} & \textbf{11-20} & \textbf{21-50} & \textbf{51+} \\  \midrule \bigstrut
General  (e.g., Java)  & 1 & 6 & 5 & 3& 1& 2 \\ \midrule \bigstrut
Scripting  (e.g., Perl) &5 &4 &3 &3 &2  &1 \\ \midrule \bigstrut
Query  (e.g., SQL) & 15&2 &0 &0 &1  & 0\\ \midrule \bigstrut
Command line (e.g., grep)   &2 &5 &3 &2 &0  &6 \\ \midrule \bigstrut
Text editor (e.g., IntelliJ)   & 2& 5& 0& 5& 1& 5\\
\bottomrule
\end{tabular}
\end{small}
\end{center}
\vspace{-12pt}
\end{table}


\begin{table}
\caption{Survey results for regex usage frequencies for tasks, averaged using a 6-point likert scale: Very Frequently=6, Frequently=5, Occasionally=4, Rarely=3, Very Rarely=2, and Never=1 \label{table:regextasks}}
\begin{center}
\begin{small}
\begin{tabular}{l|c}
\toprule
\textbf{Task} & \textbf{Frequency} \\  \midrule \bigstrut
Locating content within a file or files & 4.4\\ \midrule \bigstrut
Capturing parts of strings & 4.3 \\ \midrule \bigstrut
Parsing user input & 4.0\\ \midrule \bigstrut
Counting lines that match a pattern & 3.2\\ \midrule \bigstrut
Counting  substrings that match a pattern & 3.2\\  \midrule \bigstrut
Parsing generated text & 3.0\\  \midrule \bigstrut
Filtering collections (lists, tables, etc.) & 3.0 \\ \midrule \bigstrut
Checking for a single character & 1.7\\
\bottomrule
\end{tabular}
\end{small}
\end{center}
\vspace{-12pt}
\end{table}


Table~\ref{tab:regextasks} shows how frequently, on average, the participants use
regexes for various tasks.
Participants answered questions using a 6-point likert scale including very frequently~(6), frequently~(5), occasionally~(4), rarely~(3), very rarely~(2), and never~(1).
Averaging across participants, among the most common usages are capturing parts of a string and locating content within a file, with both occurring somewhere between occasionally and frequently.

\begin{table}
\caption{Results of subtracting the average task frequency of ephemeral users from the average task frequency of persistent users, ordered by difference \label{table:regexPersistingTasks}}
\begin{center}
\begin{small}
\begin{tabular}{lccc}
\toprule
\textbf{Task} & \textbf{Persistence Freq.} & \textbf{Ephemeral Freq.} & \textbf{Difference} \\  \midrule \bigstrut
Counting  substrings that match a pattern & 3 & 1.7 & 1.2\\  \midrule \bigstrut
Parsing user input & 3.6 & 2.7 & 0.9\\ \midrule \bigstrut
Capturing parts of strings & 3.8 & 3.1 & 0.7\\ \midrule \bigstrut
Parsing generated text & 2.4 & 1.9 & 0.5\\  \midrule \bigstrut
Locating content within a file or files & 3.6 & 3.2 & 0.4\\ \midrule \bigstrut
Filtering collections (lists, tables, etc.) & 2.2 & 1.9 & 0.3\\ \midrule \bigstrut
Counting lines that match a pattern & 1.8 & 2.1 & -0.3\\
\bottomrule
\end{tabular}
\end{small}
\end{center}
\vspace{-12pt}
\end{table}

\subsection{Ephemeral vs persistent users}
Only 27\% (5)
% (L,N,P,Q,T) T counts be
of participants wrote regular expressions that persist (general purpose, scripting, etc.) more frequently than in a text editor or command line tool (where they will not persist).  This result indicates that non-persistent, or \emph{ephemeral} regexes are most frequently used type of regexes.  It also suggests that a distinction can be made between types of regular expression users: those who primarily use regexes that are used once and then forgotten (ephemeral users), and those who primarily use regexes that are maintained as an artifact (persistent users).

The most frequently performed task according to Table~\ref{table:regextasks} is `locating content within a file or files'.  This result agrees with the idea that regexes are used more often in text editors and command line tools than in general purpose languages, since locating content is often done within a text editor.  The five participants who write persistent regexes more often also answered the task frequency questions differently.  Table~\ref{table:regexPersistingTasks} describes the tasks more frequently performed by persistent users than by ephemeral users.

\begin{table}
\caption{Survey results for regex usage frequencies, averaged using a 6-point likert scale: Very Frequently=6, Frequently=5, Occasionally=4, Rarely=3, Very Rarely=2, and Never=1 \label{tab:regexfeaturegroups}}
\begin{center}
\begin{small}
\begin{tabular}{llc}
\toprule
\textbf{Group} & \textbf{Code} &  \textbf{Frequency} \\  \midrule \bigstrut
endpoint anchors & (STR, END) & 4.4\\ \midrule \bigstrut
capture groups & (CG) & 4.2 \\ \midrule \bigstrut
word boundaries & (WNW) & 3.5 \\ \midrule \bigstrut
lazy repetition & (LZY) &  2.9\\ \midrule \bigstrut
\multirow{2}{*}{(neg) look-ahead/behind} &  (LKA, NLKA,  & \multirow{2}{*}{2.5}\\
& LKB, NLKB) & \\
\bottomrule
\end{tabular}
\end{small}
\end{center}
\vspace{-12pt}
\end{table}


\begin{table}
\caption{Survey results for preferences between custom character and default character classes \label{tab:cccvsdefault}}
\begin{center}
\begin{small}
\begin{tabular}{l|c}
\toprule
\textbf{Preference} & \textbf{Frequency} \\  \midrule \bigstrut
use only CCC & 1\\ \midrule \bigstrut
use CCC more than default & 5 \\ \midrule \bigstrut
use both equally & 2\\ \midrule \bigstrut
use default more than CCC & 10\\ \midrule \bigstrut
use only default & 2\\
\bottomrule
\end{tabular}
\end{small}
\end{center}
\vspace{-12pt}
\end{table}


\subsection{Feature-specific questions}
The pattern language for Python, which is used to compose regexes, supports default character classes like the ANY or dot character class: \cverb!.! meaning, `any character except newline'.
It also supports three other default character classes: \cverb!\d!, \cverb!\w!, \cverb!\s! (and their negations). All of these default character classes can be simulated using the custom character class (CCC) feature, which can create semantically equivalent regexes.
For example  the decimal character class: \cverb!\d! is equivalent to a CCC containing all 10 digits:  \cverb!\d! $\equiv$ \cverb![0123456789]! $\equiv$ \cverb![0-9]!.

Other default character classes such as the word character class: \cverb!\w! may not be as intuitive to encode in a CCC: \cverb![a-zA-Z0-9_]!.

Survey participants were asked if they use only CCC, use CCC more than default, use both equally, use default more than CCC or use only default.  Results for this question are shown in Table~\ref{tab:cccvsdefault}, with 67\% (12) indicating that they use default the most.

Participants who favored CCC indicated that ``it is more explicit," whereas the participants who favored default character classes said,  ``it is less verbose" and ``I like using built-in code."

To further explore how participants use various regex features, participants were asked five questions about how frequently they use specific related groups of features:
\begin{itemize} \itemsep -2pt
   \item endpoint anchors (STR, END): \cverb!^! and \cverb!$!
   \item capture groups(CG): \cverb!(capture me)!
   \item word boundaries (WNW): \cverb!word\b!
   \item (negative) look-ahead/behinds (LKA, NLKA, LKB, NLKB): \cverb!a(?=bc)!, \cverb!(?<!x)yz!, \cverb!(?<=a)!, \cverb|a(?!yz)|
   \item lazy repetition (LZY): \cverb!ab+?!, \cverb!xy{2,3}?!
\end{itemize}

Results are shown in Table~\ref{tab:regexfeaturegroups}, indicating that lazy repetition and look-ahead features are rarely used and capture groups and endpoint anchors are occasionally to frequently used.

\subsection{Testing regex and composition tools}

Participants were asked if they test their code, test their regex and if they do test their regex, what tools are used (if any).  Using a 7-point likert scale similar to those already described that includes `always' as a seventh point, developers indicated that they test their regexes with the same frequency as they test their code (average response was 5.2, which is between frequently and very frequently).  Half of the  developers indicate that they use external tools to test their regexes, and the other half indicated that they only use tests that they write themselves. Of the nine developers using tools, six mentioned online composition aides such as \url{regex101.com} where a regex and input string are entered, and the input string is highlighted according to what is matched.

\subsection{Pain points}
\label{sec:painPoints}
In question 27 of the survey, participants were asked an open-ended question about problems with regular expressions: `What pain points have you encountered with regular expressions?'.  Three main categories of response were observed. The most common, ``hard to compose," was represented in 61\% (11) responses. Next, 39\% (7) developers responded that regexes are ``hard to read" and 17\% (3) indicated difficulties with ``inconsistency across implementations," which manifest when using regexes in multiple languages. These responses do not sum to 18 as three developers provided multiple parts in their answers.

