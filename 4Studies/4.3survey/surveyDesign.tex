When trying to assess how developers use regular expressions, input directly from developers is useful and relevant.  The goal of this section is to answer the question, `What preferences, behaviors and opinions do professional developers have about using regex?'.  Professional software developers at Dwolla, a startup company focused on moving money, were asked about their usage and preferences when using regular expressions.  The results of this survey indicate that developers use regular expressions most frequently in text editors and on the command line, have a strong preference for numbered over named backreferences, and about half of developers use online testing tools when composing regular expressions.


\subsection{Survey design}

The survey designed to understand the context of when and how programmers use regular expressions was designed and implemented using Google Forms containing 30 questions.  All survey questions are in Appendix~\ref{app:surveyQuestions}. The questions asked about regex usage frequency, technical environment, tasks regexes were used for, pain points, and the use of various language
features. Participation was voluntary and participants were entered in a lottery for a \$50 gift card.

\paragraph{Self-qualifying questions} Participants were first asked if they are a developer or maintainer of software, and if they had used regexes in a work environment.  If a negative response to either of those questions was received, the Google Form skipped to the end with a `thank you' message.  These questions helped to guarantee that only people who self-identify as developers who use regex can participate.

\subsubsection{Contextual usage frequency}
The number of regexes used by a developer per year overall can be used as an indication of interest in regular expressions.  However, in trial surveys, recalling the number of regexes used was challenging.  This was addressed by first prompting recall per-environment, and per-task before asking for the overall number.  The question of how many regexes are composed by developers in particular contexts and for particular tasks is also of interest, because it provides a way to quantify the relative importance of supporting those contexts and tasks.

\subsubsection{Feature usage and refactoring questions}
To provide another perspective on feature usage frequency, developers were asked about the frequency of use for some features not supported the analysis tools examined in Table~\ref{table:featuresInTools}.  They were also asked about which features they prefer to use when two equivalent options will both work, providing information to inform refactoring recommendations.  One question asked about the what participants use the WRD default character class for, because many close variants of this default were observed in the corpus.

\subsubsection{Best practices questions}
\paragraph{Testing and composition tools} Prior work suggests that regexes are hard understand, causing tens of thousands of reported bugs per year~\cities{Spishak:2012:TSR:2318202.2318207}.   Participants were asked about their testing of regexes and testing of code in general, so that a comparison could be made.  Participants were also asked about what tools they use to test regexes.

\paragraph{Parsing HTML} Because parsing a markup language like HTML using regular expressions is a common mistake\footurl{http://stackoverflow.com/questions/1732348}, (typically an HTML parser is a better choice) participants were asked if they had ever tried to parse HTML or XML.  A separate questions asked if, when parsing text, participants prefer to use regular expressions or write a custom parser.

\paragraph{Pain points and free responses} Participants were asked open-ended questions about what pain points they have experienced and what else they have to say about using regexes.  The intention of these questions was to provide an opportunity for information not covered by other questions to arise.

\subsubsection{Participants}
The goal of the survey was to understand the practices of professional developers. Thus, the survey was deployed to 22 professional developers at Dwolla, a small software company that provides tools for online and mobile payment management. While this sample comes from a single company, we note anecdotally that Dwolla is a start-up and most of the developers worked previously for other software companies, and thus bring their past experiences with them. Surveyed developers have nine years of experience, on average, indicating the results may generalize beyond a single, small software company, but further study is needed.
