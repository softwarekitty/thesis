% \subsection{Overview of feature analysis study}
\label{sec:featureOverview}
The primary goal of this experiment was to answer the question, `How are regex used in practice, especially what features are most commonly used?'.  Python was chosen because its regular expression feature set seemed to contain a good balance between having some advanced features, and not having too many rare features (this assumption was confirmed, as discussed in Section~\ref{sec:rankedFeatureSupport}.).  A \emph{pattern} is a string that is parsed according to the feature syntax of a variant into units of meaning called \emph{feature tokens}.  For example the pattern \verb!"a*"! is parsed into the \bverb!a! ordinary character token, and the \bverb!*! KLE token.
In order to obtain data about feature usage frequency, a large number of patterns used to create regexes were required.  One obvious place to obtain these patterns was by looking at source code that calls the {\tt re} module.  One call to this module found in source code (not running live) will be referred to as a \emph{utilization}.  Utilizations are explained in further detail in Section~\ref{sec:utilizations}

With these needs in mind, a tool was implemented that does the following:
\begin{itemize} \itemsep -1pt
\item finds projects containing Python on GitHub
\item clones the repositories containing these projects
\item builds the AST of source code using files from these projects
\item populates a database with information about utilizations found
\end{itemize}

Implementation details of this tool, and some of the challenges faced are discussed in Appendix~\ref{app:miningImplementation}.  Once the data about utilizations had been collected, some questions about the utilizations themselves were explored.  This exploration can be read about in Section~\ref{sec:utilizations}.

The patterns obtained from the utilizations were parsed using a PCRE parser to create Table~\ref{table:featureStatsOnly}.  This table summarizes the findings of this experiment, that is, for each feature described in Appendix~\ref{app:featureDescriptions}, this table shows the number of projects using that feature in some pattern (as well as other data).  These findings are presented in Section~\ref{sec:featureResults}.

The knowledge of how frequently each feature is used can provide context when comparing the sets of features supported by various regular expression analysis tools and language variants besides Python Regular Expressions.  An exploration of 68 features (34 ranked by this study, and 34 other unranked features) is in Section~\ref{sec:featureSupport}.  Finally a discussion of the impact of this study and threats to validity is in Section~\ref{sec:featureDiscussion}.
