\subsection{Utilizations of the re module}
\label{sec:utilizations}

\noindent \textbf{Utilization}: A \emph{utilization} occurs whenever a regex is used in source code.  We detect utilizations by statically analyzing source code and recording calls to the {\tt re} module in Python.

\subsubsection{Utilization defined}
\label{sec:utilizationDefined}
Within a Python source code file, a {utilization} of the {\tt re} module is composed of a function, a pattern, and 0 or more flags.  Figure~\ref{fig:exampleUsage} presents an example of one {utilization}, with key components labeled. The function call is {\tt re.compile}, \verb!"(0|-?[1-9][0-9]*)$"! is the pattern, and {\tt re.MULTILINE} is an (optional) flag. When executed, this {utilization}  will compile a regex into the variable {\tt r1} from the pattern \verb!"(0|-?[1-9][0-9]*)$"!.  The resulting regex \cverb!(0|-?[1-9][0-9]*)$! is composed of two regex fragments: \cverb!0! and \cverb!-?[1-9][0-9]*! operated on by the OR \verb!|!, and contained in a CG \verb!(! \verb!)! so that the following the END feature (\cverb!$!) applies regardless of which fragment is matched. Because of the {\tt re.MULTILINE} flag used, the END specifies a position at the end of every line (instead of only the end of the last line).

\begin{figure}[tb]
\centering
\includegraphics[width=\columnwidth]{nontex/illustrations/exampleUsage.eps}
\vspace{-12pt}
\caption{Example of one regex utilization}
\vspace{-6pt}
\label{fig:exampleUsage}
\end{figure}

The regex fragment \cverb!0! matches \verb!"0"!, and the fragment on the right of the OR, \cverb!-?[1-9][0-9]*!, matches all positive or negative integers (not starting with 0) like \verb!"123"!, \verb!"9"!, \verb!"-10000"! or \verb!"-8"!. When combined the full regex \cverb!(0|-?[1-9][0-9]*)$! matches all positive and negative integers at the end of lines.  For example the multi-line string: \verb!"line 1: xyz 85!\gverb!\n!\verb!line2: -2!\gverb!\n!\verb!last line!\gverb!\n!\verb!"! will match at the end of the first two lines.  Expressing zero or one dash characters using the regex fragment \cverb!-?! is useful so that the sign of the integer will be part of the capture, (e.g., from \verb!"A: -9"!\gverb!\n!, \verb!"-9"! is captured, not just \verb!"9"!).

\noindent \textbf{Pattern}: A \emph{pattern} is extracted from a utilization, as shown in Figure~\ref{fig:exampleUsage}. As described in Section~\ref{sec:nomenclature}, a pattern is an ordered series of regular expression language feature tokens which can be compiled by an engine into a regex.  A regex compiled from the pattern in Figure~\ref{fig:exampleUsage} .

Note that because the vast majority of regular expression features are shared across most general programming languages (e.g., Java, C, C\#, or Ruby), a Python pattern will (almost always) behave the same when used in other languages as mentioned in Section~\ref{sec:usuallyOk}, whereas a utilization is not universal in the same way (i.e., it is very unlikely to compile in other languages because of variations in programming language syntax and the names of functions).

\subsubsection{Omission of calls to compiled objects}
Every utilization recorded using the technique described in Section~\ref{sec:miningImplementation} is an invocation directly using the {\tt re} library, like {\tt re.compile(...)} or {\tt re.search(...)}.  However, this technique is not able to record calls on compiled objects.  For example the regex described in Section~\ref{sec:utilizationDefined} is stored in the variable {\tt r1}.  The regex in this variable can be used to call {\tt re} module functions but will not use the re library directly.  For example the code {\tt r1.search("-45")} would not be recorded by the technique used in this work.  However, since our primary focus is on patterns and the features of their compiled regexes, and these patterns must be compiled before becoming regexes, this omission only impacts the interpretation of Figure~\ref{fig:partFunctions}, which describes which function calls were observed.  Notice that \emph{every compilation of a pattern to a regex} is captured by the technique used in this study.
