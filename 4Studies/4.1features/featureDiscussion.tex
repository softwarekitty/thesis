\subsection{Discussion of feature analysis results}
\label{sec:featureDiscussion}

\subsubsection{General Implications}
Based on these results we draw implications for programmers requiring portability, language and tool designers, and specifically, developers of the {\tt re} module.

\paragraph{Defining a set of regex features that generalizes.} As shown in Table~\ref{table:rankedFeatureSupport}, the feature set used in this study (except for PNG, NBKR and ENDZ) generalizes well across all feature-rich modern variants.  A portable set of features is useful when developing regex coding standards that enable portability, and for software development where minimizing vendor lock-in, or planning for possible migration across languages is important.

\paragraph{Providing a reference for language and tool designers.} The largest implication for language and tool designers is that now, if they need to know what features are supported, or how frequently a feature is used, such information is available in a concise format.

\paragraph{For developers of the re module.} Specific details about utilizations of the {\tt re} module, such as function and flag usage frequency, and saturation within GitHub projects, has implications for the developers of the {\tt re} module.  In many cases, the observed practice diverges from the intended design.  Namely, the near-irrelevance of the `locale' flag, the general trend toward compiling objects (presumably avoiding magic strings), and the apparent misconception that only one flag can be used at a time (the documentation clearly states that a bitwise-or of flags is effective\footurl{https://docs.python.org/2/library/re.html}, but this was never observed in over 50,000 utilizations).

\subsubsection{Threats to validity}
Compared to the overall number of Python projects in existence, the number of projects used in this study is small.  It may not be representative of Python projects on a whole.  This is mitigated by the pseudo-random nature of selecting projects based on an arbitrary unit of division, several hundred thousand repository creation events apart (Section~\ref{sec:selectingProjects}).

As discussed thoroughly in Section~\ref{sec:patternsAreNotRegexes}, patterns are not regexes, and there is always a risk that the patterns used to build the corpus would not port to other languages, making this analysis limited in application to only Python programmers.  This risk is mitigated by Table~\ref{table:rankedFeatureSupport}, which proves a high level of portability for all regexes studied, and therefore some guarantee of applicability across languages.

Human error may have invalidated some feature presence/absence entries in Table~\ref{table:rankedFeatureSupport}.  This is mitigated by the choice of languages, selected to make initial testing relatively fast, and re-testing by outside parties also relatively fast.

