\subsection{Population characteristics}

\begin{table}[!ht]
\begin{small}
\fbox{\parbox{\textwidth-10pt}{
\begin{enumerate}
\item
\begin{tabular} {lrr}
\textbf{What is your gender?} & \textbf{n} & \textbf{\%}\\ \hline
Male & 149 & 83\%\\
Female & 27& 15\%\\
Prefer not to say & 4& 2\%
\end{tabular}
\item \textbf{What is your age?} \\
$\mu = 31$, $\sigma = 9.3$

\item

\begin{tabular} {l |rr}
\textbf{Education Level?} & \textbf{n} & \textbf{\%}\\ \hline
High School & 5 & 3\%\\
Some college, no degree & 46 & 26\%\\
Associates degree & 14 & 8\%\\
Bachelors degree & 78 & 43\%\\
Graduate degree & 37 & 21\%\\
\end{tabular}
\item
\begin{tabular} {lrr}
\textbf{Familiarity with regexes?} & \textbf{n} & \textbf{\%}\\ \hline
Not familiar at all & 5 & 3\%\\
Somewhat not familiar & 16 & 9\%\\
Not sure & 2 & 1\%\\
Somewhat familiar & 121 & 67\%\\
Very familiar & 36 & 20\%\\
\end{tabular}
\item \textbf{How many regexes do you compose each year?} \\
$\mu = 67$, $\sigma = 173$
\item \textbf{How many regexes (not written by you) do you read each year?} \\
$\mu = 116$, $\sigma = 275$
\end{enumerate}
}}
\caption{Participant Profiles, $n=180$ \label{participantprofile}}
\end{small}
\end{table}

In total, there were 180 participants in the study.
A majority were male (83\%) with an average age of 31. Most had
at least an Associates degree (72\%) and most were at least somewhat familiar with regexes prior to the study (87\%). On average,
participants compose 67 regexes per year with a range of 0 to 1000.
Participants read more regexes than they write with an average of 116 and a range from 0 to 2000.
Figure~\ref{participantprofile} summarizes the self-reported participant characteristics from the qualification survey.

\subsection{Matching and composition comprehension results}
\label{sec:comprehensionResults}

For each of the 180 HITs, a matching and composition score was computed for each of the 10 regexes, using the metrics described in Section~\ref{sec:understandabilityMetrics}.  Since 30 separate participants responded to five string matching problems and one composition problem for each of the 60 regexes, there were 30 independent understandability evaluations for each representation.  An average of 0.53 out of 30 of these responses were NAs per regex, with the maximum number of NAs being four.  These 26-30 independent matching scores for each regex were used to determine if an understandability preference exists for each of the 35 pairwise comparisons.

The regexes \cverb!((q4f)?ab)! and \cverb!(q4fab|ab)!, represent a transformation between D2 and D3. The former have an average matching score of 79\% and the latter have an average matching score of 85\%. The average composition score for the former is 83\% and 97\% for the latter. Thus, the community found \cverb!(q4fab|ab)! from D3 more understandable.
The other pairwise comparison performed between D2 and D3 uses the pair \cverb!(deedo(do)?)! and \cverb!(deedo|deedodo)!.
Considering both of these regex pairs, the \emph{overall matching score} for the regexes belonging to D2 is 78\% and the \emph{overall matching score} for D3 is 87\%.
The \emph{overall composition score} for D2 is 88\%, with 97\% for D3. Thus, the community found D3 to be more understandable than D2, from the perspective of both understandability metrics, suggesting a refactoring from D2 to D3.  These measurements comparing overall representation understandability are presented in summary in Table~\ref{table:testedEdgesTable}, with this specific example appearing in the E2 row. The pairs of regexes used to determine the values for each edge are listed in Appendix~\ref{app:edgeToPairMap}.  The \emph{Index} column enumerates all the pairwise comparisons evaluated in this experiment, \emph{Nodes} lists the two representations, \emph{Pairs} shows how many comparisons were performed, \emph{Match1} gives the overall matching score for the first representation listed and \emph{Match2} gives the overall matching score for the second representation listed. $H_0^{match}$ shows the results of using the Mann-Whitney test of means to compare the matching scores, testing the null hypothesis $H_0$: that $\mu_{match1} = \mu_{match2}$.  The p-values from these tests are presented in this column. The last three columns display the average composition scores for the representations and the relevant p-value, also using the Mann-Whitney test of means.


\begin{table*}\begin{small}\begin{center}\caption{Averaged Info About Edges (sorted by lowest of either p-value)}\label{table:testedEdgesTable}\begin{tabular}
{llccccccc}
\textbf{Index} & \textbf{Nodes} & \textbf{Pairs} & \textbf{Match1} & \textbf{Match2} & \textbf{$H_0^{match} $} & \textbf{Compose1} & \textbf{Compose2} &  \textbf{$H_0^{comp}$} \bigstrut \\
\toprule[0.16em]
E1 & T1 -- T4 & 2 & 80\% & 60\% & 0.001 & 87\% & 37\% & $<$\textbf{0.001}\\
E2 & D2 -- D3 & 2 & 78\% & 87\% & \textbf{0.011} & 88\% & 97\% & 0.085\\
E3 & C2 -- C5 & 4 & 85\% & 86\% & 0.602 & 88\% & 95\% & \textbf{0.063}\\
E4 & C2 -- C4 & 1 & 83\% & 92\% & \textbf{0.075} & 60\% & 67\% & 0.601\\
\midrule[0.05em]
E5 & L2 -- L3 & 2 & 86\% & 91\% & 0.118 & 97\% & 100\% & 0.159\\
E6 & D1 -- D2 & 2 & 84\% & 78\% & 0.120 & 93\% & 88\% & 0.347\\
E7 & C1 -- C2 & 2 & 94\% & 90\% & 0.121 & 93\% & 90\% & 0.514\\
E8 & T2 -- T4 & 2 & 84\% & 81\% & 0.498 & 65\% & 52\% & 0.141\\
E9 & C1 -- C5 & 2 & 94\% & 90\% & 0.287 & 93\% & 93\% & 1.000\\
E10 & T1 -- T3 & 3 & 88\% & 86\% & 0.320 & 72\% & 76\% & 0.613\\
E11 & D1 -- D3 & 2 & 84\% & 87\% & 0.349 & 93\% & 97\% & 0.408\\
E12 & C1 -- C4 & 6 & 87\% & 84\% & 0.352 & 86\% & 83\% & 0.465\\
E13 & C3 -- C4 & 2 & 61\% & 67\% & 0.593 & 75\% & 82\% & 0.379\\
E14 & S1 -- S2 & 3 & 85\% & 86\% & 0.776 & 88\% & 90\% & 0.638\\
\bottomrule[0.13em]\end{tabular}\end{center}\end{small}\end{table*}


In Table~\ref{table:testedEdgesTable}, a thin horizontal line separates the top four edges from the bottom 10.  In these four edges, there is a statistically significant difference between the representations for at least one of the metrics considering $\alpha = 0.10$. These represent the strongest evidence for suggesting the directions of refactoring based on the understandability metrics defined in this study. Specifically, $\overrightarrow{T4 T1}$, $\overrightarrow{D2 D3}$, $\overrightarrow{C2 C5}$ and $\overrightarrow{C2 C4}$ are likely to improve understandability.  The specific nodes, regexes, matching scores and compositions scores that led to these refactoring suggestions are shown in Table~\ref{table:alpha10}

\begin{table}[!ht]
\begin{center}
\caption{Equivalent regexes with a significant difference in understandability, $\alpha=0.1$}
\label{table:alpha10}
\begin{small}
\begin{tabular}
{lccc c lccc}
\textbf{Code} & \textbf{Regex} & \textbf{Match} & \textbf{Compose} & \textbf{Refactoring} & \textbf{Node} & \textbf{Regex} & \textbf{Match} & \textbf{Compose} \bigstrut \\
\noalign{\hrule height 0.08em}
T4 & \begin{minipage}{0.9in}\cverb!([\072\073])!\end{minipage} & 66\% & 50\% & \multirow{ 2}{*}{$\overrightarrow{T4 T1}$} & T1 & \begin{minipage}{1.2in}\cverb!([:;])!\end{minipage} & 81\% & 87\% \bigstrut   \\
T4 & \begin{minipage}{0.9in}\cverb!([\0175\0173])!\end{minipage} & 54\% & 23\% & & T1 & \begin{minipage}{1.2in}\cverb!([}{])!\end{minipage} & 79\% & 87\%   \bigstrut  \\
\noalign{\hrule height 0.04em}
D2 & \begin{minipage}{0.9in}\cverb!((q4f)?ab)!\end{minipage} & 79\% & 83\% & \multirow{ 2}{*}{$\overrightarrow{D2 D3}$} & D3 & \begin{minipage}{1.2in}\cverb!(q4fab|ab)!\end{minipage} & 85\% & 97\%  \bigstrut   \\
D2 & \begin{minipage}{0.9in}\cverb!(deedo(do)?)!\end{minipage} & 77\% & 93\% & & D3 & \begin{minipage}{1.2in}\cverb!(deedo|deedodo)!\end{minipage} & 90\% & 97\%  \bigstrut   \\
\noalign{\hrule height 0.08em}
C2 & \begin{minipage}{0.9in}\cverb!([:;])!\end{minipage} & 81\% & 87\% & \multirow{ 4}{*}{$\overrightarrow{C2 C5}$} & C5 & \begin{minipage}{1.2in}\cverb!(:|;)!\end{minipage} & 94\% & 100\%  \bigstrut   \\
C2 & \begin{minipage}{0.9in}\cverb!no[wxyz]5!\end{minipage} & 87\% & 90\% &  & C5 & \begin{minipage}{1.2in}\cverb!no(w|x|y|z)5!\end{minipage} & 94\% & 97\%  \bigstrut   \\
C2 & \begin{minipage}{0.9in}\cverb!([}{])!\end{minipage} & 79\% & 87\% & & C5 & \begin{minipage}{1.2in}\cverb!(\{|\})!\end{minipage} & 70\%  & 93\%  \bigstrut  \\
C2 & \begin{minipage}{0.9in}\cverb!tri[abcdef]3!\end{minipage} & 93\% & 90\% & & C5 & \begin{minipage}{1.2in}\cverb!tri(a|b|c|d|e|f)3!\end{minipage} & 86\% & 90\%   \bigstrut  \\
\noalign{\hrule height 0.04em}
C2 & \begin{minipage}{0.9in}\cverb![\t\r\f\n ]!\end{minipage} & 83\% & 60\% & $\overrightarrow{C2 C4}$ & C4 & \begin{minipage}{1.2in}\cverb![\s]!\end{minipage} & 92\%  & 67\%  \bigstrut  \\
\noalign{\hrule height 0.08em}
\end{tabular}
\end{small}
\end{center}
\vspace{-12pt}
\end{table}


In addition to the edges with a significant difference in understandability,  edges E5 through E8 exhibit noteworthy trends in understandability.  The regexes and matching scores for these nodes are shown in Table~\ref{table:interestingEdges}.  For edges E7 and E8, the potential refactorings $\overrightarrow{C2 C1}$ and $\overrightarrow{T4 T2}$ are also suggested by the community standards determined in Section~\ref{sec:nodeCountingResults}.  E5 is also notable because in \emph{both matching and composition}, all scores consistently indicate a small preference for L3 over L2 (5\% in overall matching and 3\% in overall composing). \label{sec:compL2L3} More work is needed to determine if a significant difference in understandability exists between these nodes.

\begin{table}[!ht]
\begin{center}
\caption{Additional equivalent regexes for which some preference in understandability is suggested}
\label{table:interestingEdges}
\begin{small}
\begin{tabular}
{lccc c lccc}
\begin{scriptsize}\textbf{Node}\end{scriptsize} & \begin{scriptsize}\textbf{Regex}\end{scriptsize} & \begin{scriptsize}\textbf{Match}\end{scriptsize} & \begin{scriptsize}\textbf{Comp.}\end{scriptsize} & \begin{scriptsize}\textbf{Ref.}\end{scriptsize} & \begin{scriptsize}\textbf{Node}\end{scriptsize} & \begin{scriptsize}\textbf{Regex}\end{scriptsize} & \begin{scriptsize}\textbf{Match}\end{scriptsize} & \begin{scriptsize}\textbf{Comp.}\end{scriptsize} \bigstrut \\
\noalign{\hrule height 0.08em}
L2 & \begin{minipage}{0.92in}\cverb!zaa*!\end{minipage} & 87\% & 97\% & \multirow{ 2}{*}{$\overrightarrow{L2 L3}$} & L3 & \begin{minipage}{1.0in}\cverb!za+!\end{minipage} & 91\% & 100\%  \bigstrut   \\
L2 & \begin{minipage}{0.92in}\cverb!RR*!\end{minipage} & 86\% & 97\% & & L3 & \begin{minipage}{1.0in}\cverb!R+!\end{minipage} & 92\%  & 100\%  \bigstrut  \\
\noalign{\hrule height 0.04em}
D2 & \begin{minipage}{0.92in}\cverb!((q4f)?ab)!\end{minipage} & 79\% & 83\% & \multirow{ 2}{*}{$\overrightarrow{D2 D1}$} & D1 & \begin{minipage}{1.0in}\cverb!((q4f){0,1}ab)!\end{minipage} & 83\% & 97\%  \bigstrut   \\
D2 & \begin{minipage}{0.92in}\cverb!(deedo(do)?)!\end{minipage} & 77\% & 93\% &  & D1 & \begin{minipage}{1.0in}\cverb!(dee(do){1,2})!\end{minipage} & 85\% & 90\%  \bigstrut   \\
\noalign{\hrule height 0.04em}
C2 & \begin{minipage}{0.92in}\cverb!tri[abcdef]3!\end{minipage} & 93\% & 90\% & \multirow{ 2}{*}{$\overrightarrow{C2 C1}$} & C1 & \begin{minipage}{1.0in}\cverb!tri[a-f]3!\end{minipage} & 94\% & 97\%  \bigstrut   \\
C2 & \begin{minipage}{0.92in}\cverb!no[wxyz]5!\end{minipage} & 87\% & 90\% & & C1 & \begin{minipage}{1.0in}\cverb!no[w-z]5!\end{minipage} & 93\%  & 90\%  \bigstrut  \\
\noalign{\hrule height 0.08em}
T4 & \begin{minipage}{0.92in}\begin{scriptsize}\cverb!xyz[\0133-\0140]!\end{scriptsize}\end{minipage} & 71\% & 33\% & \multirow{ 2}{*}{$\overrightarrow{T4 T2}$} & T2 & \begin{minipage}{1.0in}\cverb!xyz[\x5b-\x5f]!\end{minipage} & 79\% & 60\%  \bigstrut   \\
T4 & \begin{minipage}{0.92in}\begin{scriptsize}\cverb!t[\072-\073]+p!\end{scriptsize}\end{minipage} & 90\% & 70\% &  & T2 & \begin{minipage}{1.0in}\cverb!t[\x3a-\x3b]+p!\end{minipage} & 89\% & 70\%  \bigstrut   \\
\noalign{\hrule height 0.08em}
\end{tabular}
\end{small}
\end{center}
\vspace{-12pt}
\end{table}

