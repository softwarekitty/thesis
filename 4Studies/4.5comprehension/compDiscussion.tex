\subsection{Discussion of comprehension results}

\subsubsection{Implications}
Two statistically significant refactorings $\overrightarrow{T4 T1}$ and $\overrightarrow{D2 D3}$ were identified by the results presented in Table~\ref{table:testedEdgesTable}.  A detailed view of the results for these refactorings is presented in Table~\ref{table:pairwiseRefactorings}.  The first refactoring, $\overrightarrow{T4 T1}$, makes sense because the octal syntax is far more exotic and difficult to understand than plain characters. Composition improves notably from 23\% for \cverb!([\0175\0173])! to 87\% for \cverb!([}{])!.  This results seems likely to generalize, as there is no reason to think that participants were less familiar with octal than programmers in general.

The second refactoring $\overrightarrow{D2 D3}$, reduces confusion caused by the QST feature, by expanding the entire set of strings specified by the regex into an OR.  The OR feature is fundamental to regular expressions, and so the regexes in D3 are very straightforward -  essentially lists of strings, whereas the QST repetition may take a little thought.  This result seems likely to generalize for very simple examples like the one that was tested, using only one QST operator.

This refactoring is not likely to scale, however, because a slightly more complicated regex like \cverb!a?b*(cd)?e?! would expand to the very long regex \cverb!ab*cde|b*cde|ab*e|b*e|ab*cd|b*cd|ab*|b*! which introduces the new challenge of visually parsing and remembering eight strings.

Although not statistically significant within the chosen alpha (0.05), all tested refactorings \emph{out} of C2 (into C5, C4 and C1) provided at least a slight advantage on average in both matching and composing scores.  This may not indicate a refactoring (suggesting what node to choose), but instead indicate that C2 is a smelly node.

The most notable difference in measured understandability is between \cverb![\t\r\f\n ]! from C2 with a matching score of 83\%, and \cverb![\s]! from C4 with a matching score of 92\%.

Moving from C2 to C5 is not as clear cut, with examples supporting both directions.  For the regex \cverb!([:;])!, the matching score increased from 81\% to 94\% when moving to \cverb!(:|;)!  However for \cverb!tri[abcdef]3!, the matching score decreased from 93\% to 86\% when moving to \cverb!tri(a|b|c|d|e|f)3!.

Moving from C2 regex \cverb!no[wxyz]5! with a matching score of 87\% to either C1 or C5 boosted the matching score to 94\% or 93\%, respectively.

\subsubsection{Threats to validity}
Mechanical turk may not be an ideal source for regex comprehension study participants.  This threat was mitigated by requiring workers to pass a pre-qualification test, and by checking the composed regexes for validity before accepting any HIT.

The regexes chosen to represent nodes may have exhibited more variety in understandability than is present in the refactorings.  This risk is mitigated by composing regexes of approximately equal difficulty, as much as possible, from the perspective of the author.

Design flaws have reduced the coverage of equivalence classes, so that not all refactoring possibilities are fully explored.  This risk cannot be fully mitigated.  Several improvements to the experiment design are possible and with the benefit of experiences gained during the course of this study, the author acknowledges that a superior experiment could be executed.  However, with the rigorous treatment of excluding all data that was obtained under misconceptions, and making due with a partial result, this experiment retains valid results.
