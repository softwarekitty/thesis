\subsection{Discussion of comprehension results}

\subsubsection{Implications}
Two statistically significant refactorings $\overrightarrow{T4 T1}$ and $\overrightarrow{D2 D3}$ were identified by the results presented in Table~\ref{table:testedEdgesTable}.  A detailed view of the results for these refactorings is presented in Table~\ref{table:pairwiseRefactorings}.  The first refactoring, $\overrightarrow{T4 T1}$, makes sense because the octal syntax is far more exotic and difficult to understand than plain characters. Composition improves notably from 23\% for \cverb!([\0175\0173])! to 87\% for \cverb!([}{])!.  This results seems likely to generalize, as there is no reason to think that participants were less familiar with octal than programmers in general.

The second refactoring $\overrightarrow{D2 D3}$, reduces confusion caused by the QST feature, by expanding the entire set of strings specified by the regex into an OR.  The OR feature is fundamental to regular expressions, and so the regexes in D3 are very straightforward -  essentially lists of strings, whereas the QST repetition may take a little thought.  This result seems likely to generalize for very simple examples like the one that was tested, using only one QST operator.

This refactoring is not likely to scale, however, because a slightly more complicated regex like \cverb!a?b*(cd)?e?! would expand to the very long regex \cverb!ab*cde|b*cde|ab*e|b*e|ab*cd|b*cd|ab*|b*! which introduces the new challenge of visually parsing and remembering eight strings.

Although not statistically significant within the chosen alpha (0.05), all tested refactorings \emph{out} of C2 (into C5, C4 and C1) provided at least a slight advantage on average in both matching and composing scores.  This may not indicate a refactoring (suggesting what node to choose), but instead indicate that C2 is a smelly node.

The most notable difference in measured understandability is between \cverb![\t\r\f\n ]! from C2 with a matching score of 83\%, and \cverb![\s]! from C4 with a matching score of 92\%.

Moving from C2 to C5 is not as clear cut, with examples supporting both directions.  For the regex \cverb!([:;])!, the matching score increased from 81\% to 94\% when moving to \cverb!(:|;)!  However for \cverb!tri[abcdef]3!, the matching score decreased from 93\% to 86\% when moving to \cverb!tri(a|b|c|d|e|f)3!.

Moving from C2 regex \cverb!no[wxyz]5! with a matching score of 87\% to either C1 or C5 boosted the matching score to 94\% or 93\%, respectively.


\subsubsection{Opportunities for future work}
\todoLast{easy section}

\todoMid{Discuss the backslash explosions and readability issues!}

\subsubsection{Threats to validity}
many - bad behavior by MT participants is possible.  Also three noticeable design flaws: the mistake with \cverb!\..*!, the mistakes with pairings of nodes that could not be used (bc study was implemented before equiv class design was completed), and that some edges are not covered - example?
Also these results are fairly think and the signal is not very loud, with the exception of T4 to T1.  A more perfectly designed study, with a better set of participants could really build on what was done here - could do it better.  However, the results shown seem to be valid, if quite limited.


% There are many more ways to approach understandability, such as deciding what content is captured by a regex, identifying all the matched substrings in a block of text, deciding which regexes in a set are equivalent, finding the minimum modification to some text so that a given regex will match it, and many more.  One of the most straightforward ways to address understandability is to directly ask software professionals which from a list of equivalent regexes they prefer and why.  It may also be meaningful to provide some code that exists around a regex as context.  The example regexes we used were inspired by real regexes, but at least one side of the refactoring was contrived and we did not focus on any specific community (the 1544 projects we obtained regexes from were randomly obtained).  If understandability measurements used regexes sampled from the codebase of a specific community(most frequently observed regexes, most buggy regexes, regexes on the hottest execution paths, etc.), and measured the understanding of programming professionals working in that community, then the measurements and the refactorings they imply would be more likely to have a direct and certain positive impact.


% the most preferred nodes for each group are C1, D2, T1, L2, and S2.  In this section the practical issues of
% With one exception, these are the same for recommendations based on projects. The difference is that L3 appears in more projects than L2, so it is not clear which is more desirable based on community standards.


% \paragraph{Understandability}
% We identified and utilized three new ways to measure understandability of regexes: deciding if certain strings match or not, composing strings that are supposed to match, and measuring the frequency of a regex type in a community.
% In another study, we did a survey where software professionals indicated that understandability of regexes they find in source code is a major pain point.  In this study, our participants indicated that they read about twice as many regexes as they compose.  What is the impact on maintainers, developers and contributors to open-source projects of not being able to understand a regex that they find in the code they are working with?  Presumably this is a frustrating experience - how much does a confusing regex slow down a software professional?  What bugs or other negative factors can be attributed to or associated with regexes that are difficult to understand?  How often does this happen and in what settings?  Future work could tailor an in-depth exploration of the overall costs of confusing regexes and the potential benefits of refactoring or other treatments for confusing regexes.

% Presently, we are not aware of coding standards for regular expressions, but this work suggests that enforcing standard representations for various regex constructs could ease comprehension.
