\subsection{Discussion of comprehension results}

\subsubsection{Implications}
Four statistically significant refactorings for understandability: $\overrightarrow{T4 T1}$, $\overrightarrow{D2 D3}$, $\overrightarrow{C2 C5}$ and $\overrightarrow{C2 C4}$ were identified by the results presented in Table~\ref{table:testedEdgesTable}.  A detailed view of the results for these refactorings is presented in Table~\ref{table:alpha10}.  The refactoring $\overrightarrow{T4 T1}$ is also indicated by the analysis of community support in Section~\ref{sec:nodeCountingResults}.

Four additional trends in understandability results were found that are not statistically significant with $\alpha=0.1$, but show potential for the refactorings, $\overrightarrow{L2 L3}$, $\overrightarrow{D2 D3}$, $\overrightarrow{C2 C1}$ and $\overrightarrow{T4 T2}$.  Two of these, $\overrightarrow{D2 D3}$ and $\overrightarrow{T4 T2}$, are refactorings also indicated by the analysis of community support in Section~\ref{sec:nodeCountingResults}.

\paragraph{T4 to T4 is always recommended} The first refactoring, $\overrightarrow{T4 T1}$, makes sense because the octal syntax is far more exotic and difficult to understand than plain characters. Composition improves notably from 23\% for \cverb!([\0175\0173])! to 87\% for \cverb!([}{])!.  This results seems likely to generalize, as there is no reason to think that participants were less familiar with octal than programmers in general.

\paragraph{D2 to D3 is recommended for small repetitions} The second refactoring $\overrightarrow{D2 D3}$, reduces confusion caused by the QST feature, by expanding the entire set of strings specified by the regex into an OR.  The OR feature is fundamental to regular expressions, and so the regexes in D3 are very straightforward -  essentially lists of strings, whereas the QST repetition may take a little thought.  This result seems likely to generalize for very simple examples like the one that was tested, using only one QST operator.

This refactoring is not likely to scale, however, because a slightly more complicated regex like \cverb!a?b*(cd)?e?! would expand to the very long regex \cverb!ab*cde|b*cde|ab*e|b*e|ab*cd|b*cd|ab*|b*! which introduces the new challenge of visually parsing and remembering eight strings.

\paragraph{Refactoring out of C2 needs more study} Refactorings out of C2 to (C1, C4 and C5) all displayed a strong trend for improving understandability.  More study is needed into determining which node to choose when refactoring out of C2.


\subsubsection{Threats to validity}
Mechanical Turk may not be an ideal source for regex comprehension study participants.  This threat was mitigated by requiring workers to pass a pre-qualification test, and by checking the composed regexes for potential validity before accepting any HIT, as mentioned in Section~\ref{sec:workerStatistics}.

Poor understandability could be due to an overly long and complex regex, instead of the representation being tested.  This risk is mitigated by composing regexes of approximately equal difficulty, as much as possible, from the perspective of the author.

Design flaws have reduced the coverage of equivalence classes, so that not all refactoring possibilities are fully explored.  Several improvements to the experiment design are possible and with the benefit of experiences gained during the course of this study, a superior experiment could be executed.  However, with the rigorous treatment of excluding all data that was obtained under misconceptions, and making due with a partial result, this experiment retains valid results.
