% ISU HEADER SECTION:
% Template file for a standard thesis
\documentclass[12pt]{article}
\usepackage{calc}
\usepackage{enumitem}
\usepackage[pdftex]{graphicx}
\usepackage[margin=0.8in]{geometry}

% for checking margins:
% \usepackage{showframe}

% Standard, old-style
% \usepackage{isutraditional}
% \chaptertitle
% Old-style, thesis numbering down to subsubsection
% \alternate
% \usepackage{rotating}
% Bibliography without numbers or labels
% \usepackage{natbib}
% \bibliographystyle{apa}


\usepackage{algorithm}
\usepackage{algpseudocode}
\usepackage{balance}
% \usepackage{carlthesis}
\usepackage{tikz}
\newcommand{\yes}{\tikz\draw[black,fill=black] (0,0) circle (.5ex);}
\newcommand{\sorta}{\tikz\draw[gray,fill=gray] (0,0) circle (.5ex);}
\newcommand{\no}{\tikz\draw[black] (0,0) circle (.5ex);}
\usepackage[update,prepend,outdir=./nontex/]{epstopdf}
\usepackage{import}
\usepackage{multirow}
\usepackage{multicol}
\usepackage{enumitem}
\usepackage{afterpage}
\usepackage{listings}
\usepackage{placeins}

\usepackage[pdftex,hypertexnames=false,linktocpage=true]{hyperref}
\hypersetup{colorlinks=true,linkcolor=blue,anchorcolor=blue,citecolor=blue,filecolor=blue,urlcolor=blue,bookmarksnumbered=true,pdfview=FitB}
\newcommand{\footurl}[1]{\footnote{\url{#1}}}

\RequirePackage{booktabs}
\renewcommand*\cmidrule{\midrule[0.001em]} % Thin middle lines

\RequirePackage{pifont}
\newcommand{\xmark}{\ding{53}}

\RequirePackage{fancyvrb,newverbs}
\definecolor{cverbbg}{gray}{0.93}
\definecolor{bverbbg}{gray}{0.975}
\definecolor{iverbbg}{gray}{0.96}
\newcommand{\verbatimfont}[1]{\def\verbatim@font{#1}}%
\newverbcommand{\cverb}
  {\setbox\verbbox\hbox\bgroup}
  {\egroup\colorbox{cverbbg}{\box\verbbox}}
\newverbcommand{\gverb}
{\color{gray}}
{}
\newverbcommand{\bverb}
  {(\setbox\verbbox\hbox\bgroup}
  {\egroup\colorbox{bverbbg}{\box\verbbox})}

\RequirePackage{bigstrut}
\setlength\bigstrutjot{2pt}
\newcommand{\horiz}{\hspace{2.1pt}}
\newcommand{\var}[1]{{\ttfamily#1}}
\widowpenalty=10000
\clubpenalty=10000
\displaywidowpenalty = 10000
\hyphenation{second-ly ap-pen-dix}
\pagenumbering{gobble}

\begin{document}

\begin{table*}[h!tb]
\centering
\begin{small}
\caption{What regular expression languages support features studied in this thesis?}
\label{table:featureVariationLanguages}
\begin{tabular}{ll@{  \horiz}lc @{   \horiz} c @{   \horiz}c @{   \horiz}c @{   \horiz}c @{   \horiz}c @{   \horiz}c @{   \horiz}c @{   \horiz}c}rank & code & example & Python & Perl & .Net & \begin{footnotesize}JavaScript\end{footnotesize} &  Java & \begin{footnotesize}POSIX ERE\end{footnotesize} & Ruby & RE2 & VIM \\
1 & ADD & \begin{minipage}{0.5in}\begin{verbatim}z+\end{verbatim}\end{minipage} & \yes & \yes & \yes & \yes & \yes & \yes & \yes & \yes\\
\midrule
2 & CG & \begin{minipage}{0.5in}\begin{verbatim}(caught)\end{verbatim}\end{minipage} & \yes & \yes & \yes & \yes & \yes & \yes & \yes & \yes\\
\midrule
3 & KLE & \begin{minipage}{0.5in}\begin{verbatim}.*\end{verbatim}\end{minipage} & \yes & \yes & \yes & \yes & \yes & \yes & \yes & \yes\\
\midrule
4 & CCC & \begin{minipage}{0.5in}\begin{verbatim}[aeiou]\end{verbatim}\end{minipage} & \yes & \yes & \yes & \yes & \yes & \yes & \yes & \yes\\
\midrule
5 & ANY & \begin{minipage}{0.5in}\begin{verbatim}.\end{verbatim}\end{minipage} & \yes & \yes & \yes & \yes & \yes & \yes & \yes & \yes\\
\midrule
6 & RNG & \begin{minipage}{0.5in}\begin{verbatim}[a-z]\end{verbatim}\end{minipage} & \yes & \yes & \yes & \yes & \yes & \yes & \yes & \yes\\
\midrule
7 & STR & \begin{minipage}{0.5in}\begin{verbatim}^\end{verbatim}\end{minipage} & \yes & \yes & \yes & \yes & \yes & \yes & \yes & \yes\\
\midrule
8 & END & \begin{minipage}{0.5in}\begin{verbatim}$\end{verbatim}\end{minipage} & \yes & \yes & \yes & \yes & \yes & \yes & \yes & \yes\\
\midrule
9 & NCCC & \begin{minipage}{0.5in}\begin{verbatim}[^qwxf]\end{verbatim}\end{minipage} & \yes & \yes & \yes & \yes & \yes & \yes & \yes & \yes\\
\midrule
10 & WSP & \begin{minipage}{0.5in}\begin{verbatim}\s\end{verbatim}\end{minipage} & \yes & \yes & \yes & \yes & \yes & \no & \yes & \yes\\
\midrule
11 & OR & \begin{minipage}{0.5in}\begin{verbatim}a|b\end{verbatim}\end{minipage} & \yes & \yes & \yes & \yes & \yes & \yes & \yes & \yes\\
\midrule
12 & DEC & \begin{minipage}{0.5in}\begin{verbatim}\d\end{verbatim}\end{minipage} & \yes & \yes & \yes & \yes & \yes & \no & \yes & \yes\\
\midrule
13 & WRD & \begin{minipage}{0.5in}\begin{verbatim}\w\end{verbatim}\end{minipage} & \yes & \yes & \yes & \yes & \yes & \no & \yes & \yes\\
\midrule
14 & QST & \begin{minipage}{0.5in}\begin{verbatim}z?\end{verbatim}\end{minipage} & \yes & \yes & \yes & \yes & \yes & \yes & \yes & \yes\\
\midrule
15 & LZY & \begin{minipage}{0.5in}\begin{verbatim}z+?\end{verbatim}\end{minipage} & \yes & \yes & \yes & \yes & \yes & \no & \yes & \yes\\
\midrule
16 & NCG & \begin{minipage}{0.5in}\begin{verbatim}a(?:b)c\end{verbatim}\end{minipage} & \yes & \yes & \yes & \yes & \yes & \no & \yes & \yes\\
\midrule
17 & PNG & \begin{minipage}{0.5in}\begin{verbatim}(?P<name>x)\end{verbatim}\end{minipage} & \yes & \yes & \no & \no & \no & \no & \no & \yes\\
\midrule
18 & SNG & \begin{minipage}{0.5in}\begin{verbatim}z{8}\end{verbatim}\end{minipage} & \yes & \yes & \yes & \yes & \yes & \yes & \yes & \yes\\
\midrule
19 & NWSP & \begin{minipage}{0.5in}\begin{verbatim}\S\end{verbatim}\end{minipage} & \yes & \yes & \yes & \yes & \yes & \no & \yes & \yes\\
\midrule
20 & DBB & \begin{minipage}{0.5in}\begin{verbatim}z{3,8}\end{verbatim}\end{minipage} & \yes & \yes & \yes & \yes & \yes & \yes & \yes & \yes\\
\midrule
21 & NLKA & \begin{minipage}{0.5in}\begin{verbatim}a(?!yz)\end{verbatim}\end{minipage} & \yes & \yes & \yes & \yes & \yes & \no & \yes & \no\\
\midrule
22 & WNW & \begin{minipage}{0.5in}\begin{verbatim}\b\end{verbatim}\end{minipage} & \yes & \yes & \yes & \yes & \yes & \no & \yes & \yes\\
\midrule
23 & NWRD & \begin{minipage}{0.5in}\begin{verbatim}\W\end{verbatim}\end{minipage} & \yes & \yes & \yes & \yes & \yes & \no & \yes & \yes\\
\midrule
24 & LWB & \begin{minipage}{0.5in}\begin{verbatim}z{15,}\end{verbatim}\end{minipage} & \yes & \yes & \yes & \yes & \yes & \yes & \yes & \yes\\
\midrule
25 & LKA & \begin{minipage}{0.5in}\begin{verbatim}a(?=bc)\end{verbatim}\end{minipage} & \yes & \yes & \yes & \yes & \yes & \no & \yes & \no\\
\midrule
26 & OPT & \begin{minipage}{0.5in}\begin{verbatim}(?i)CasE\end{verbatim}\end{minipage} & \yes & \yes & \yes & \no & \yes & \no & \yes & \yes\\
\midrule
27 & NLKB & \begin{minipage}{0.5in}\begin{verbatim}(?<!x)yz\end{verbatim}\end{minipage} & \yes & \yes & \yes & \no & \yes & \no & \yes & \no\\
\midrule
28 & LKB & \begin{minipage}{0.5in}\begin{verbatim}(?<=a)bc\end{verbatim}\end{minipage} & \yes & \yes & \yes & \no & \yes & \no & \yes & \no\\
\midrule
29 & ENDZ & \begin{minipage}{0.5in}\begin{verbatim}\Z\end{verbatim}\end{minipage} & \yes & \no & \no & \no & \no & \no & \no & \yes\\
\midrule
30 & BKR & \begin{minipage}{0.5in}\begin{verbatim}\1\end{verbatim}\end{minipage} & \yes & \yes & \yes & \yes & \yes & \yes & \yes & \no\\
\midrule
31 & NDEC & \begin{minipage}{0.5in}\begin{verbatim}\D\end{verbatim}\end{minipage} & \yes & \yes & \yes & \yes & \yes & \no & \yes & \yes\\
\midrule
32 & BKRN & \begin{minipage}{0.5in}\begin{verbatim}(P?=name)\end{verbatim}\end{minipage} & \yes & \yes & \no & \no & \no & \no & \no & \no\\
\midrule
33 & VWSP & \begin{minipage}{0.5in}\begin{verbatim}\v\end{verbatim}\end{minipage} & \yes & \yes & \yes & \yes & \yes & \yes & \no & \yes\\
\midrule
34 & NWNW & \begin{minipage}{0.5in}\begin{verbatim}\B\end{verbatim}\end{minipage} & \yes & \yes & \yes & \yes & \yes & \no & \yes & \yes\\
\midrule
\end{tabular}
\end{small}
\vspace{-12pt}
\end{table*}

Table~\ref{table:rankedFeatureSupport} compares support for the studied 34 features amongst Perl, Python, Ruby, .Net, JavaScript, RE2, Java and POSIX ERE (i.e., grep, sed, etc.).  Each feature (besides python-style named groups or named backreferences and the nuanced ENDZ feature) is supported by five or more of the other seven (non-Python) languages.
\pagebreak

\begin{table*}[h!tb]
\centering
\begin{footnotesize}
\caption{What features, not studied in this thesis, are supported in various languages?}
\label{table:unrankedFeatureSupport}
\begin{tabular}{l@{  \horiz}lc @{   \horiz} c @{   \horiz}c @{   \horiz}c @{   \horiz}c @{   \horiz}c @{   \horiz}c @{   \horiz}c} \\
\textbf{Code} & \textbf{Example} & \textbf{Python} & \textbf{Perl} & \textbf{.Net}  & \textbf{Ruby} &  \textbf{Java} & \textbf{RE2} & \begin{footnotesize}\textbf{JavaScript}\end{footnotesize} & \begin{footnotesize}\textbf{POSIX ERE}\end{footnotesize}\\
\toprule
RCUN & \begin{minipage}{0.8in}\begin{verbatim}(?n)\end{verbatim}\end{minipage} & \no & \yes & \no & \no & \no & \no & \no & \no  \\
\midrule
RCUZ & \begin{minipage}{0.8in}\begin{verbatim}(?R)\end{verbatim}\end{minipage} & \no & \yes & \no & \no & \no & \no & \no & \no  \\
\midrule
GPLS & \begin{minipage}{0.8in}\begin{verbatim}\g{+1}\end{verbatim}\end{minipage} & \no & \yes & \no & \no & \no & \no & \no & \no  \\
\midrule
GBRK & \begin{minipage}{0.8in}\begin{verbatim}\g{name}\end{verbatim}\end{minipage} & \no & \yes & \no & \no & \no & \no & \no & \no  \\
\midrule
GSUB & \begin{minipage}{0.8in}\begin{verbatim}\g<name>\end{verbatim}\end{minipage} & \yes & \yes & \no & \yes & \no & \no & \no & \no  \\
\midrule
KBRK & \begin{minipage}{0.8in}\begin{verbatim}\k<name>\end{verbatim}\end{minipage} & \no & \yes & \yes & \yes & \yes & \no & \no & \no  \\
\midrule
IFC & \begin{minipage}{0.8in}\begin{verbatim}(?(cond)X)\end{verbatim}\end{minipage} & \no & \yes & \yes & \no & \no & \no & \no & \no  \\
\midrule
IFEC & \begin{minipage}{0.8in}\begin{verbatim}(?(cnd)X|else)\end{verbatim}\end{minipage} & \no & \yes & \yes & \no & \no & \no & \no & \no  \\
\midrule
ECOD & \begin{minipage}{0.8in}\begin{verbatim}(?{code})\end{verbatim}\end{minipage} & \no & \yes & \no & \no & \no & \no & \no & \no  \\
\midrule
ECOM & \begin{minipage}{0.8in}\begin{verbatim}(?#comment)\end{verbatim}\end{minipage} & \yes & \yes & \yes & \yes & \no & \no & \no & \no  \\
\midrule
PRV & \begin{minipage}{0.8in}\begin{verbatim}\G\end{verbatim}\end{minipage} & \no & \yes & \yes & \yes & \yes & \no & \no & \no  \\
\midrule
LHX & \begin{minipage}{0.8in}\begin{verbatim}\uFFFF\end{verbatim}\end{minipage} & \no & \yes & \yes & \yes & \yes & \no & \yes & \no  \\
\midrule
POSS & \begin{minipage}{0.8in}\begin{verbatim}a?+\end{verbatim}\end{minipage} & \no & \yes & \no & \yes & \yes & \no & \no & \no  \\
\midrule
NNCG & \begin{minipage}{0.8in}\begin{verbatim}(?<name>X)\end{verbatim}\end{minipage} & \no & \yes & \yes & \yes & \yes & \no & \no & \no  \\
\midrule
MOD & \begin{minipage}{0.8in}\begin{verbatim}(?i)z(?-i)z\end{verbatim}\end{minipage} & \no & \yes & \yes & \yes & \yes & \yes & \no & \no  \\
\midrule
ATOM & \begin{minipage}{0.8in}\begin{verbatim}(?>X)\end{verbatim}\end{minipage} & \no & \yes & \yes & \yes & \yes & \no & \no & \no  \\
\midrule
CCCI & \begin{minipage}{0.8in}\begin{verbatim}[a-z&&[^f]]\end{verbatim}\end{minipage} & \no & \no & \no & \yes & \yes & \no & \no & \no  \\
\midrule
STRA & \begin{minipage}{0.8in}\begin{verbatim}\A\end{verbatim}\end{minipage} & \yes & \yes & \yes & \yes & \yes & \yes & \no & \no  \\
\midrule
LNLZ & \begin{minipage}{0.8in}\begin{verbatim}\Z\end{verbatim}\end{minipage} & \no & \yes & \yes & \yes & \yes & \yes & \no & \no  \\
\midrule
FINL & \begin{minipage}{0.8in}\begin{verbatim}\z\end{verbatim}\end{minipage} & \no & \yes & \yes & \yes & \yes & \yes & \no & \no  \\
\midrule
QUOT & \begin{minipage}{0.8in}\begin{verbatim}\Q...\E\end{verbatim}\end{minipage} & \no & \yes & \no & \no & \yes & \yes & \no & \no  \\
\midrule
JAVM & \begin{minipage}{0.8in}\begin{verbatim}\p{javaMirrored}\end{verbatim}\end{minipage} & \no & \no & \no & \no & \yes & \no & \no & \no  \\
\midrule
UNI & \begin{minipage}{0.8in}\begin{verbatim}\pL\end{verbatim}\end{minipage} & \no & \yes & \no & \no & \yes & \yes & \no & \no  \\
\midrule
NUNI & \begin{minipage}{0.8in}\begin{verbatim}\PS\end{verbatim}\end{minipage} & \no & \yes & \no & \no & \yes & \yes & \no & \no  \\
\midrule
OPTG & \begin{minipage}{0.8in}\begin{verbatim}(?flags:re)\end{verbatim}\end{minipage} & \no & \yes & \yes & \yes & \yes & \yes & \no & \no  \\
\midrule
EREQ & \begin{minipage}{0.8in}\begin{verbatim}[[=o=]]\end{verbatim}\end{minipage} & \no & \no & \no & \no & \no & \no & \no & \yes  \\
\midrule
PXCC & \begin{minipage}{0.8in}\begin{verbatim}[:alpha:]\end{verbatim}\end{minipage} & \no & \yes & \no & \yes & \no & \yes & \yes & \yes  \\
\midrule
TRIV & \begin{minipage}{0.8in}\begin{verbatim}[^]\end{verbatim}\end{minipage} & \no & \no & \no & \no & \no & \no & \yes & \no  \\
\midrule
CCSB & \begin{minipage}{0.8in}\begin{verbatim}[a-f-[c]]\end{verbatim}\end{minipage} & \no & \no & \yes & \no & \no & \no & \no & \no  \\
\midrule
VLKB & \begin{minipage}{0.8in}\begin{verbatim}(?<=ab.+)\end{verbatim}\end{minipage} & \no & \no & \yes & \no & \no & \no & \no & \no  \\
\midrule
BAL & \begin{minipage}{0.8in}\begin{verbatim}(?<close-open>)\end{verbatim}\end{minipage} & \no & \no & \yes & \no & \no & \no & \no & \no  \\
\midrule
NCND & \begin{minipage}{0.8in}\begin{verbatim}(?(<n>)X|else)\end{verbatim}\end{minipage} & \no & \yes & \yes & \yes & \no & \no & \no & \no  \\
\midrule
BRES & \begin{minipage}{0.8in}\begin{verbatim}(?|(A)|(B))\end{verbatim}\end{minipage} & \no & \no & \no & \no & \no & \no & \no & \no  \\
\midrule
QNG & \begin{minipage}{0.8in}\begin{verbatim}(?'name're)\end{verbatim}\end{minipage} & \no & \no & \yes & \yes & \no & \no & \no & \no  \\
\bottomrule
\end{tabular}
\end{footnotesize}
\vspace{-12pt}
\end{table*}

Table~\ref{table:unrankedFeatureSupport} describes feature support for a selection of 34 \emph{unranked} features (not in the studied feature set) chosen from the eight languages being investigated. No feature is supported by five or more other languages without also being supported by Python.

\end{document}
