\section{Gap in fundamental research into regex use in practice}

Although regex have provided an essential search functionality for software development for the last 47 years, are essential to parsing, compiling, security, database queries and user input validation, and are incorporated into all but the most low-level programming languages, no fundamental research has been published investigating user behaviors, preferences, use cases, pain points, or challenges in composition and comprehension.  Faced with an open field, we formulated \todoLast{N} questions to begin the work of filling this fundamental knowledge gap.  The following section articulates the motivations behind the questions explored in this thesis.

\section{Questions explored in this thesis and their motivations}

\subsection{RQ1: How are regex used in practice, especially what features are most commonly used?}

Regex researchers and tool designers must pick what features to include or exclude, which  can be a difficult  design decision. Supporting advanced features may be more expensive, taking more time and potentially making the project too complex and cumbersome to execute well.  A selection of only the simplest of regex features limits the applicability or relevance of that work. Despite extensive research effort in the area of regex support,  no research has been done about how regexes are used in practice and what features are essential for the most common use cases.

\subsection{RQ2: What behavioral categories can be observed in regex?}
\todoLast{Clean these thoughts}
If we know some categories of regex behavior, then that gives good insight into what users are really doing with regex and in turn, what behaviors are most important for future regex technologies.  Given a sample of the population of regexes in the wild, we expect to see some behavioral groups.  But how to define behavior, and how to automate the investigation enough to handle a large number of regex?  This analysis was very cpu-intensive and ran up against many implementation challenges, but is a successful first attempt to investigate regex composer's behaviors and needs.

\subsection{RQ3: What preferences, behaviors and opinions do professional developers have about using regex?}

\todoLast{Clean these thoughts} Why not just ask software developers about their use habits and preferences in a survey?  That's what we did here.  But it's important to mention that these questions had the benefit of the feature and behavioral analysis



\subsection{RQ4: Within five equivalence classes, what representations are most frequently observed?}

\todoLast{Clean these thoughts}There are many ways to represent the same functional regex, that is, the user has choices to make about how to compose a regex for any given task.  Assuming that regex composers will tend to choose the best representation most of the time, we want to know what representation choices are most frequent.

\subsection{RQ5: What representations are more comprehensible?}

\todoLast{Clean these thoughts}
After defining the equivalence classes and potential  regex refactorings we wanted to know which representations in the equivalence classes  are considered desirable and which might be smelly. Desirability for regexes can be defined many ways, including maintainable,  understandable, and performance.
We focus on refactoring for understandability.

\subsection{RQ6: For each equivalence class, which representation is preferred according to frequency and comprehensibility?}

\todoLast{Clean these thoughts}This section formalizes a technique of ordering the data from the previous two sections.

% \subsection{RQ7: What resources do students of regex employ when faced with composition challenges?}

% \todoLast{Compose thoughts about the video study}


