
\todoNow{Create Intro For Related Section?}
This is the opening paragraph to my thesis which
explains in general terms the concepts and hypothesis
which will be used in my thesis.

With more general information given here than really
necessary.

\section{Related Work From Features}
\label{sec:related}
Regular expressions have been a focus point in a variety of research objectives. From the user perspective, tools have been developed to support more robust creation~\cite{Spishak:2012:TSR:2318202.2318207} or to allow visual debugging~\cite{Beck:2014:RVD:2591062.2591111}.
Building on the perspective that regexes are difficult to create, other research has focused on removing the human from the creation process by learning regular expressions from  text~\cite{Babbar:2010:CBA:1871840.1871848, Li:2008:REL:1613715.1613719}.

Regarding applications, regular expressions have been used for test case generation~\cite{Ghosh:2013:JAT:2486788.2486925, Galler:2014:STD:2683035.2683100, Anand:2013:OSM:2503903.2503991, Tillmann:2014:TAT:2642937.2642941},  and
as specifications for string constraint solvers~\cite{Trinh:2014:SSS:2660267.2660372, hampi}.
Regexes are also employed in MySQL injection prevention~\cite{Yeole:2011:ADT:1980022.1980229} and network intrusion detection~\cite{network}, or in more diverse applications like DNA sequencing alignment~\cite{1594922} or querying RDF data~\cite{Lee:2010:PSQ:1871871.1871877, Alkhateeb:2009:ESR:1540656.1540975}.

As a query language, lightweight regular expressions are pervasive in search. For example,
some data mining frameworks use regular expressions as queries (e.g., ~\cite{Begel:2010:CDE:1806799.1806821}). Efforts have also been made to expedite the processing of regular expressions on large bodies of text~\cite{Baeza-Yates:1996:FTS:235809.235810}.

Research tools like Hampi~\cite{hampi}, and Rex~\cite{rex}, and commercial tools like brics\cite{brics} and RE2~\cite{re2}, all support the use of regular expressions in various ways. Hampi was developed  in academia and uses regular expressions as a specification language for a constraint solver. Rex was developed by Microsoft Research and generates strings for regular expressions that can be used in  applications such as test case generation~\cite{Anand:2013:OSM:2503903.2503991, Tillmann:2014:TAT:2642937.2642941}. Brics is an open-source package that creates automata from regular expressions for manipulation and evaluation.
RE2 is an open-source tool created by Google to power code search with an efficient regex engine.


Mining properties of open source repositories is a well-studied topic, focusing, for example, on API usage patterns~\cite{Linares-Vasquez:2014:MEA:2597073.2597085} and bug characterizations~\cite{Chen:2014:ESD:2597073.2597108}.
Exploring language feature usage by mining source code has been studied extensively for
Smalltalk~\cite{Callau:2011:DUD:1985441.1985448, Callau:2013:DUD:2589712.2589718},
JavaScript~\cite{Richards:2010:ADB:1809028.1806598},
and Java~\cite{Dyer:2014:MBA:2568225.2568295, Grechanik:2010:EIL:1852786.1852801, Parnin:2013:AUJ:2589712.2589717, Livshits:2005:RAJ:2099708.2099724},
and more specifically,
Java generics~\cite{Parnin:2013:AUJ:2589712.2589717} and
Java reflection~\cite{Livshits:2005:RAJ:2099708.2099724}.
To our knowledge, this is the first work to mine and evaluate regular expression usages from existing software repositories. Related to mining work, regular expressions have been used to form queries in mining framework~\cite{Begel:2010:CDE:1806799.1806821}, but have not been the focus of the mining activities.
Surveys have been used to measure adoption of various programming languages~\cite{Meyerovich:2013:EAP:2509136.2509515, Dattero:2004:PLG:962081.962087}, and been combined with  repository analysis~\cite{Meyerovich:2013:EAP:2509136.2509515}, but have not focused on regexes.

\section{Related Work From Refactoring}
\label{sec:relatedR}

Regular expression understandability has not been studied directly, though prior work has suggested that regexes are hard to read and understand since there are tens of thousands of bug reports related to regular expressions~\cite{Spishak:2012:TSR:2318202.2318207}.
To aid in regex creation and understanding,  tools have been developed to support more robust creation~\cite{Spishak:2012:TSR:2318202.2318207} or to allow visual debugging~\cite{Beck:2014:RVD:2591062.2591111}. Building on the perspective that regexes are difficult to create, other research has focused on removing the human from the creation process by learning regular expressions from  text~\cite{Babbar:2010:CBA:1871840.1871848, Li:2008:REL:1613715.1613719}.

Regular expression refactoring has also not been studied directly, though refactoring literature abounds~\cite{Mens:2004:SSR:972215.972286, Opdyke:1992:ROF:169783, Griswold:1993:AAP:152388.152389}.
The closest to regex refactoring comes from research toward  expediting the processing of regular expressions on large bodies of text~\cite{Baeza-Yates:1996:FTS:235809.235810}, which could be thought of as refactoring for performance.

Code smells in object-oriented languages were introduced by Fowler~\cite{Fowl1999}. Researchers have studied the impact of code smells on program comprehension~\cite{abbes2011empirical, du2006does}, finding that the more smells in the code, the harder the comprehension. This is similar to our work, except we aim to identify which  regex representations can be considered smelly.
Code smells have been extended to other language paradigms including end-user programming languages~\cite{Hermans2012intra, Hermans2012intraExt, stoleeicse, stoleeTSE}. The code smells identified in this work are representations that are not common or not well understood by developers. This concept of using community standards to define smells has been used in other refactoring literature  for end-user programmers~\cite{stoleeicse, stoleeTSE}.

Exploring language feature usage by mining source code has been studied extensively for
Smalltalk~\cite{Callau:2011:DUD:1985441.1985448},
JavaScript~\cite{Richards:2010:ADB:1809028.1806598},
and Java~\cite{Dyer:2014:MBA:2568225.2568295, Grechanik:2010:EIL:1852786.1852801, Parnin:2013:AUJ:2589712.2589717, Livshits:2005:RAJ:2099708.2099724},
and more specifically,
Java generics~\cite{Parnin:2013:AUJ:2589712.2589717} and
Java reflection~\cite{Livshits:2005:RAJ:2099708.2099724}.
Our prior work (~\cite{chapman2016}, under review) was the first to mine and evaluate regular expression usages from existing software repositories. The intention of the prior work~\cite{chapman2016} was to explore regex language features  usage and surveyed developers about regex usage. In this work, we define potential refactorings and use the mined corpus to find support for the presence of various regex representations in the wild. Beyond that, we measure regex understandability and suggest canonical representations for regexes to enhance conformance to community standards and understandability.

