\section{Categorization of clusters}
\subsection{Implementation details}
\subsection{Results}


We manually mapped the top 100 largest clusters based on the number of projects into 6 behavioral categories (determined by inspection).  The largest cluster was left out, as it was composed of patterns that trivially matched almost any string, like \verb!`b*'! and \verb!`^'!.  The remaining 99 clusters were all categorized. These clusters are briefly summarized in Table~\ref{tab:clustercats}, showing the name of the category and the number of clusters it represents, patterns in those clusters, and projects. The most common category is \emph{Multi Matches}, which contains clusters that have alternate behaviors (e.g., matching a comma or a semicolon, as in \verb!`,|;'(18)!). Each cluster was mapped to exactly one category. Next, we describe the categories, ordered by the number of projects the regex patterns map to.

\begin{table}
\begin{center}
\begin{small}
\caption{Cluster categories and sizes, ordered by number of projects containing at least one pattern in the category. \label{tab:clustercats}}
\begin{tabular}{lcccc}
\toprule
\textbf{Category} & \textbf{Clusters} & \textbf{Patterns} & \textbf{Projects} & \textbf{\% Projects} \\  \midrule \bigstrut
Multi Matches & 21 & 237 & 295 & 40\% \\
\midrule \bigstrut
Specific Char & 17 & 103 & 184 & 25\% \\
\midrule \bigstrut
Anchored Patterns & 20 & 85 & 141 & 19\% \\
\midrule \bigstrut
Two or More Chars & 16 & 40 & 120 & 16\% \\
\midrule \bigstrut
Content of Parens & 10 & 46 & 111 & 15\% \\
\midrule \bigstrut
Code Search & 15 & 27 & 92 & 13\% \\
\bottomrule
\end{tabular}
\vspace{-12pt}
\end{small}
\end{center}
\end{table}


\subsubsection{Multiple Matching Alternatives}
The patterns in these clusters match under a variety of conditions by using a character class or a disjunctive \verb!|!.
For example:
\verb!`(\W)'(89)! matches any alphanumeric character, \verb!`(\s)'(89)! matches any whitespace character, \verb!`\d'(58)! matches any numeric character, and \verb!`,|;'(18)! matches a comma or semicolon.  Most of these clusters are represented by patterns that use default character classes, as opposed to custom character classes.  This provides further support for our survey results to the question, \emph{Do you prefer to use custom character classes or default character classes more often?}, in which a majority of participants indicated they use the default classes more than custom.
This category contains 21 clusters, each appearing in an average of 33 projects.

\subsubsection{Specific Character Must Match}
\label{cluster:single}
Each cluster in this category requires one specific character to match, for example:
\verb!`\n\s*'(42)! matches only if a newline is found, \verb!`:+'(31)! matches only if a colon is found, \verb!`%'(22)!, matches only if a percent sign is found and \verb!`}'(14)! matches only if a right curly brace is found.
% Table~\ref{table:exampleCluster} presents a cluster that falls under this category. While the cluster is centered on the presence of the \verb!`:'! character, the other regexes in the cluster also exhibit more diverse behavior.
The commonality of this cluster category contrasts with the survey in
(Section) in which participants reported to very rarely or never use regexes to check for a single character (Table~\ref{tab:regexactivities}).
This category contains 17 clusters, each appearing in an average of 17.1 projects.
 These clusters have a combined total of 103 patterns, with at least one pattern present in 184 projects.

\subsubsection{Anchored Patterns}
Each of the clusters uses at least one endpoint anchor to require matches to be absolutely positioned, for example:
\verb!`(\w+)$'(35)! captures the word characters at the end of the input, \verb!`^\s'(16)! matches a whitespace at the beginning of the input, and \verb!`^-?\d+$'(17)! requires that the entire input is an (optionally negative) integer.
These anchors are the only way in regexes to guarantee that a character does (or does not) appear at a particular location by specifying what is allowed. As an example, \verb!^[-_A-Za-z0-9]+$! says that from beginning to end, only \verb![-_A-Za-z0-9]! characters are allowed, so it will fail to match if undesirable characters, such as \verb!?!, appear anywhere in the string.
This category contains 20 clusters, each appearing in an average of 15.4 projects.
These clusters have a combined total of 85 patterns, with at least one pattern present in 141 projects.

\todoMid{The thing I want to mention about anchored patterns (but have struggled to say in the past) is that they are the only way to guarantee that a character does not appear in a particular location by specifying what is allowed.  Consider the regex }
\verb!^[-_A-Za-z0-9]+$!
\todoMid{ which will fail to match if an undesirable character like `?' appears anywhere in the input.  In logic, there is a similar phenomenon.  That is, `Always' is true iff `Not Exists' of the negation is true, and by requiring an entire input to always maintain some abstraction, you can indirectly specify the negation of another (inverse) abstraction.  Even with only one anchor point, a regex like }
\verb!.*[0-9]$!
\todoMid{ is creating an ultimatum about the end being a digit.  Without the endpoint anchors, I don't see how one could specify absolutes about an input. }

\subsubsection{Content of Brackets and Parenthesis}
\label{cluster:contentparens}
The clusters in this category center around finding a pair of characters that surround content, often also capturing that content. For example,
\verb!`\(.*\)'(29)! matches when content is surrounded by parentheses and \verb!`".*"'(25)! matches  when content is surrounded by double quotes.  The cluster \verb!`<(.+)>'(23)! matches and captures content surrounded by angled brackets.
This category contains 10 clusters, each appearing in an average of 18.4 projects.
 These clusters have a combined total of 46 patterns, with at least one pattern present in 111 projects.
\todoMid{include this?, and }
\verb!`\[.*\]'(22)!
\todoMid{matches when content is surrounded by square brackets}

\subsubsection{Two or More Characters in Sequence}
\label{cluster:multiple}
These clusters require several characters in a row to match some pattern, for example:
\verb!`\d+\.\d+'(30)! requires one or more digits followed by a period character, followed by one or more digits.  The cluster \verb!`  '(17)! requires two spaces in a row,
\verb!`([A-Z][a-z]+[A-Z][^ ]+)'(11)!,
and \verb!`@[a-z]+'(9)! requires the at symbol followed by two or more lowercase characters, as in a twitter handle.
This category contains 16 clusters, each appearing in an average of 13 projects.
These clusters have a combined total of 40 patterns, with at least one pattern present in 120 projects.

\todoMid{Again, it might be interesting to look at what particular sequences are looking like.  I think I mention this again in the discussion, but should we put it here instead?}

\subsubsection{Code Search and Variable Capturing}
\label{cluster:search}
These clusters show a recognizable effort to parse source code or URLs. For example,
\verb!`^https?://'(23)! matches a web address, and \verb!`(.+)=(.+)'(9)! matches an assignment statement, capturing both the variable name and value.
The cluster  \verb!`\$\{([\w\-]+)\}'(11)! matches an evaluated string interpolation and captures the code to evaluate.
This category contains 15 clusters, each appearing in an average of 11.7 projects.
These clusters have a combined total of 27 patterns, with at least one pattern present in 92 projects.

\vspace{6pt}
