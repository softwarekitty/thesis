\section*{Similarity Matrix Creation Details}
\label{app:similarityMatrixCreation}

A free trial of Windows 7 was run within VMware on a macbook pro.  The Rex~\cities{rex} executable\footurl{http://research.microsoft.com/en-us/downloads/7f1d87be-f6d9-495d-a699-f12599cea030/} requires .Net 4.5, and the similarity matrix creating program was written in C\# using visual studio 2013.  First the patterns for 3,582 Python regexes appearing in multiple projects were used to try and generate strings using Rex, which rejected 711 patterns.  For the remaining 2,871 patterns that Rex could generate strings for, the test strings were stored in a distinct file for each regex, and delimited by a large random string (Rex often needs to generate muti-line test strings).

The filtered corpus of Rex-compatible regexes was written to a file to increase loading speed in the next step.  For each regex in the filtered corpus, the test strings stored for that regex were loaded and all other regexes attempted to match those strings.  Although regular expression engines usually perform a match quickly, an occasional pathological combination of regex and test string would cause the entire program to stall.  The Parallel.For(...) functionality of C\# was used to allow work to continue, but eventually the program had to be stopped using an interrupt.  This caused incomplete rows of data which needed to be pruned by a separate program and re-calculated.  All rows were verified in a final step before exporting the similarity matrix.

\section*{Markov Clustering Arguments}
\label{app:mclTuning}

The mcl tool takes many arguments, with the main value, $i$, controlling inflation.  A larger value of $i$ will produce more, smaller clusters, and visa versa.  A cutoff value $p$ below which edges are treated as zero, is also provided.  A third value $k$ can be used to customize the number of neighbor nodes to track per computation~\cities{mclManual}.  The default values for these three are 2, 0.75 and 4 respectively.  Extensive experimentation comparing the contents of clusters using various values for $i$, $p$ and $k$ led to the choice of $i=1.8$, $p=0.75$ and $k=83$.  Under the advisement of the mcl manual, the directional edges produced by the similarity determining technique were averaged to form a symmetric edge weight matrix before clustering~\cities{mclManual}.
