\section*{Description Of Studied Features}
\label{sec:featureDescriptions}

\subsection*{Elements}

\subsubsection*{Elements: Ordinary characters}

Ordinary characters in regexes specify a literal match of those characters, for example \cverb!z! matches \verb!"z"! and \verb!"abz"!.  Regexes can be concatenated together to create a new regex, so that \cverb!z! and \cverb!q! can become \cverb!zq!, which matches \verb!"XYzq"! but not \verb!"z"!.  Python Regular Expressions\footurl{https://github.com/python/cpython/blob/master/Lib/re.py} use the special characters \verb!.!,\verb!^!,\verb!$!,\verb!*!,\verb!+!,\verb!?!,\verb!{!,\verb!}!,\verb![!,\verb!]!,\verb!(!,\verb!)!,\verb!|! and \verb!\! to implement the features that allow compact specification of sets of strings.  These special characters can be escaped using the backslash to be treated as ordinary characters, for example \cverb!zq\$! matches \verb!"zq$"!.

\subsubsection*{Elements: Escaped characters and VWSP}

Several characters need be written in Python strings using the backslash.  These characters are the backslash:\verb!\\!, bell:\verb!\a!, backspace\verb!\b!, form feed:\verb!\f!, newline:\verb!\n!, carriage return:\verb!\r!, horizontal tab:\verb!\t! and vertical whitespace:\verb!\v!.  The vertical tab is rarely used and was examined on its own as an individual feature with the code VWSP.  Every character can also be expressed in hex or octal form.  For example, a regex expressing the newline character \cverb!\n! is equivalent to the same character expressed in hex: \cverb!\x0A! and octal: \cverb!\012!.  In addition to the characters mentioned, the singlequote \verb!`''! and doublequote \verb!`"'! often must be escaped, depending on the quotation style used in source code.  This work will not address this issue in further detail.

\subsubsection*{Elements: Character Classes}

\begin{description}  \itemsep -1pt
\item[CCC:] A \emph{custom character class} uses the special characters \verb![! and \verb!]! to enclose a set of characters, any of which can match.  For example \cverb!c[ao]t! matches the both \verb!"cat"! and \verb!"cot"!.  The terminology used in this thesis highlights the difference between a \emph{custom} character class and a \emph{default} character class.  Default character classes are built-in to the language and cannot be changed, whereas custom character classes provide the user of regex with the ability to create their own character classes, customized to fit whatever is needed.  Order does not matter in a CCC, so \cverb![ab]! is equivalent to \cverb![ba]!.
\item[NCCC:] A \emph{negated custom character class} uses the special character \verb!^! as the first character within the brackets of a CCC in order to negate the specified set.  For example the regex \cverb!c[^ao]t! would \emph{not} match \verb!"cat"! or \verb!"cot"!, but would match \verb!"cbt"!, \verb!"c2t"!, \verb!"c$t"! or any string containing a character other than \verb!`a'! or \verb!`o'! between \verb!`c'! and \verb!`t'!.  Notice that the exact set of characters specified by a NCCC depends on what charset is being used. NCCCs in Python's {\tt re} module use the Unicode charset.  In this thesis the 128 characters of traditional ASCII are used for a charset when explaining a concept, because it makes for more compact examples.  For instance the NCCC \cverb![^ao]! excludes 2 characters from a set of 128 characters and will therefore match the remaining 126 characters.

The caret character can be escaped within a CCC, so that \cverb![\^]! represents the set containing only \verb!`^'!.  If a caret appears after some other character, it no longer needs to be escaped, so the CCC \cverb![x^]! represents the set containing \verb!`x'! and \verb!`^'!.
\item[RNG:] A \emph{range} provides shorthand within a CCC for the set of all characters in the charset between two characters (including those two characters).  So \cverb![w-z]! is equivalent to \cverb![wxyz]!.  This feature also works with punctuation or invisible characters, as long as the start of the range occurs before the end of the range.  For example the CCC with range \cverb![:-@]! is equivalent to the CCC with no range \cverb![:;<=>?@]!.  Note that order of ranges and other characters do not matter, so that \cverb![w-z:-@]! is equivalent to \cverb![:-@w-z]!.  The dash character can be included in a CCC, for example \cverb![a-z-]! specifies the lowercase letters and the dash.  The NCCC \cverb![^-]! represents all characters except the dash.
\item[ANY:] The \emph{any} default character class uses the special character \verb!.! to specify any character except the newline character.  For example \cverb!a.b! specifies all strings beginning with an \verb!`a'! and ending with a \verb!`b'! with exactly one non-newline character between the \verb!`a'! and \verb!`b'!, such as \verb!"a2b"!, \verb!"aXb"! or \verb!"a b"!.  In Python, the meaning of this character class can be altered by passing the `DOTALL' flag or using the `s' option so that ANY will also match newlines.  When this flag or option is in effect, ANY will match every character in the charset.
\item[DEC:] The \emph{decimal} default character class uses the special sequence \verb!\d! to specify digits, and so \cverb!\d! is equivalent to \cverb![0-9]!.
\item[NDEC:] The \emph{negated decimal} default character class indicated by the special sequence \verb!\D! is simply the negation of the DEC default character class, so \cverb!\D! is equivalent to \cverb![^0-9]! or \cverb![^\d]!.
\item[WRD:] The \emph{word} default character class uses the special sequence \verb!\w! to specify digits, lowercase letters, uppercase letters and the underscore character.  Therefore \cverb!\w! is equivalent to \cverb![0-9a-zA-Z_]!.
\item[NWRD:] The \emph{negated word} default character class indicated by the special sequence \verb!\W! is simply the negation of the WRD default character class.  Therefore \cverb!\W! is equivalent to \cverb![^0-9a-zA-Z_]! or \cverb![^\w]!.
\item[WSP:] The \emph{whitespace} default character class uses the special sequence \verb!\s! to specify whitespace.  Many characters may be considered whitespace, but the definition for this thesis will be the space, tab, newline, carriage return, form feed and vertical tab.  This set is based on the POSIX \cverb![:space:]! default character class.  Therefore the regexes \cverb!\s! and \cverb![ \t\n\r\f\v]! are considered equivalent.
\item[NWSP:] The \emph{negated whitespace} default character class indicated by the special sequence \verb!\S! is simply the negation of the WSP default character class. Therefore \cverb!\S! is equivalent to \cverb![^ \t\n\r\f\v]]! or \cverb![^\s]!.
\end{description}

\subsubsection*{Elements: Logical groups}
\begin{description}  \itemsep -1pt
\item[CG:] The \emph{capture group} feature uses the special characters \verb!(! and \verb!)! to logically group some regex.  Other operations treat the contents of the logical group as a single unit, so that all operations within the group are performed before operations outside it are considered.  This follows the typical use of parenthesis in algebra as expected.  For example consider \cverb!A(12|98)!, where the regex \cverb!12|98! is treated as one element because it is in a CG.  Therefore \cverb!A(12|98)! matches \verb!"A12"! or \verb!"A98"!.  Without the logical grouping provided by CG, the regex \cverb!A12|98! will match \verb!"A12"! or \verb!"98"! - the concatenation of \cverb!A! is no longer applied to \cverb!98! because it is no longer logically next to the \cverb!A!.

In addition to providing logical grouping, the text matched by the contents of the capture group is stored, or `captured' and can be referred to later in the regex by a back-reference or extracted by a program for any purpose.  The captured content is frequently referred to by the number of the capture group like `group 1' or `group 2'.  For example when \cverb!(x*)(y*)z! matches \verb!"AxxyyzB"!, group 1 contains \verb!"xx"!, and group 2 contains  \verb!"yy"!.  Group 0 contains the entire matched portion of input: \verb!"xxyyz"!.
\item[BKR:] The \emph{back-reference} feature uses the special character \verb!\! followed by a number `n' to refer to the captured contents of the nth capture group, as defined by the order of opening parenthesis.  For example in \cverb!(a.b)\1!, the \cverb!\1! is referring to whatever was captured by \cverb!a.b!.  This regex will match the strings \verb!"aXbaXb"! and \verb!"a2ba2b"! but not \verb!"aXba2b"! because the character matched by ANY in \cverb!a.b! is \verb!`X'! not \verb!`2'!.
\item[PNG:] A \emph{Python-style named capture group} uses the syntax \verb!(?P<name>X)! to name a capture group.  This is known as Python-style because there are other styles of named capture group such as Microsoft's .NET style and other variants.  Python's implementation is noteworthy because it was the first attempt at naming groups.  Names used must be alphanumeric and start with a letter.
\item[BKRN:] The \emph{back-reference; named} feature uses the special syntax \verb!(?P=N)!, and is a back-reference for content captured by PNG with name `N'.  For example, the regex using PNG and BKRN: \cverb!(P<OldGreg>a.b)(?P=OldGreg)! is equivalent to the regex using CG and BKR: \cverb!(a.b)\1!.
\item[NCG:] The \emph{non-capture group} uses the special syntax \verb!(?:E)! to create a NCG containing element `E'.  A NCG can be used in place of a CG to perform logical grouping without affecting capturing logic.  So \cverb!(?:a+)(b+)c\1! will match \verb!"abcb"! because the NCG \cverb!(?:a+)! is ignored by the BKR, so that the first CG is \cverb!(b+)!, which is what is back-referenced by \cverb!\1!.  In contrast, \cverb!(a+)(b+)c\1! would not match \verb!"abcb"! but would match \verb!"abca"! because its first CG is \cverb!"(a+)"!
\end{description}

\subsection*{Options}
\begin{description} \itemsep -1pt
\item[OPT:] The \emph{options} feature allows the user to modify the engine's matching behavior within the regex itself, instead of using flag arguments passed to the regex engine.  For example the regex \cverb!(?i)[a-z]! uses the option \cverb!(?i)! which switches on the ignore-case flag, so that this regex will match \verb!"lower"! and \verb!"UPPER"!.  Other options include \cverb!(?s)! for single-line mode making ANY match all characters, \cverb!(?m)! for multiline mode, making STR and END match the beginning and ending of every line, \cverb!(?l)! may change the meaning of default character classes if a locale has been set, \cverb!(?x)! ignores whitespace between \emph{tokens} and \cverb!(?u)!.  In Python Regular Expressions, options can appear anywhere within the regex and will have the same effect.
\end{description}

\subsection*{Operators}
\subsubsection*{Operators: Repetition modifiers}
\begin{description} \itemsep -1pt
\item[ADD:] \emph{Additional} repetition uses the special character \verb!+! to specify one or more of an element.  For example the regex \cverb!z+! describes the set of strings containing one or more \verb!`z'! characters, such as \verb!"z"!, \verb!"zz"! and \verb!"zzzzz"!.  The regex \cverb!.+! applies additional repetition to the ANY character class, matching one or more non-newline characters such as \verb!"7a"! or \verb!"tulip"!.  In \cverb!(a.b)+!, additional repetition is applied to the CG \cverb!(a.b)!.   By applying additional repetition to this logical group, \cverb!(a.b)+! specifies strings with one or more sequential strings matching the regex in that group, such as \verb!"a2b"!, \verb!"a2baXba*b"! or \verb!"a1ba2ba3ba4b"!.
\item[KLE:] \emph{Kleene star} repetition uses the special character \verb!*! to specify zero-or-more repetition of an element.  For example the regex \cverb!pt*! describes the set of strings that begin with a \verb!`p'! followed by zero or more \verb!`t'! characters, such as \verb!"p"!, \verb!"ptt"! and \verb!"pttttt"!.
\item[QST:] \emph{Questionable} repetition specifies zero-or-one repetition of an element.  For example \cverb!zz(top)?! matches strings \verb!"zz"! and \verb!"zztop"!.
\item[SNG:] \emph{Single-bounded repetition} uses the special characters \verb!{! and \verb!}! containing an integer `n' to specify repetition of some element exactly n times.  For example \cverb!(ab*){3}! will match exactly three sequential occurrences of the regex \cverb!ab*!, such as \verb!"aaa"! or \verb!"abababb"! but not \verb!"aa"! or \verb!"ababb"!.
\item[DBB:] \emph{Double-bounded repetition} uses the special characters \verb!{! and \verb!}! containing integers `m' and `n' separated by a comma to specify repetition of some element at least m times and at most n times.  For example \cverb!(A.X){1,3}! will match one, two or three sequential occurrences of the regex \cverb!A.X!, such as \verb!"A7X"!, \verb!"AaXAnX"! or \verb!"A*XAgXAqX"!.
\item[LWB:] \emph{Lower-bounded repetition} uses the special characters \verb!{! and \verb!}! containing an integer `n' followed by a comma to indicate at least n repetitions of an element.  For example \cverb!(Qt){2,}! will match two or more sequential occurrences of \cverb!Qt!, such as \verb!"QtQt"! or \verb!"QtQtQtQt"! but will not match \verb!"Qt"!.
\item[LZY:] The \emph{lazy} repetition modifier uses the special character \verb!?! \emph{following another repetition operator} to specify lazy repetition.  An example of this syntax is \cverb!(a+?)a*b!, where QST is applied to the ADD in \cverb!a+! to yeild \cverb!a+?!, making ADD lazy instead of greedy.  This regex will match \verb!"aab"!, capturing \verb!"a"! in group 1.  The regex without LZY is \cverb!(a+)a*b! which will also match \verb!"aab"! but will capture \verb!"aa"! in group 1.
\end{description}

\subsubsection*{Operators: Logical OR}
\begin{description} \itemsep -1pt
\item[OR:] An \emph{or} is a disjunction of alternatives, where any of the alternatives is equally acceptable.  This feature is specified by the \verb!|! special character.  Each alternative can be any regex.  A simple example is the regex \cverb!cat|dog! which specifies the two strings \verb!"cat"! and \verb!"dog"!, so either of these will match.
\end{description}

\subsubsection*{Order of operations}
\label{sec:orderOfOperations}
The order of operations is:
\begin{enumerate}\itemsep -1pt
\item repetition features
\item implicit concatenation of elements
\item logical OR
\end{enumerate}
For example, consider the regex \cverb!A|BC+!.  The ADD repetition modifier takes highest importance, so that this regex is equivalent to \cverb!A|B(C+)!.  Then implicit concatenation joins the two regex \cverb!B! and \cverb!(C+)! into \cverb!B(C+)!, so the regex is equivalent to \cverb!A|(B(C+))!.  The last operator to be considered is the logical OR, so that this regex is also equivalent to \cverb!(A|(B(C+)))!.

\subsection*{Positions}

\subsubsection*{Positions: Anchors}

\begin{description} \itemsep -1pt

\item[STR:] The \emph{start anchor} uses the special character \verb!^! to indicate the position before the first character of a string, so \cverb!^B.*! will match every string that starts with \verb!`B'! such as \verb!"Bison"! and \verb!"Bouncy castle"!.  If the `MULTILINE' flag or `m' option is passed to the regex engine, then STR will match the position immediately after every newline.  In this case this regex will match in two separate places for the string \verb!"Big!\gverb!\n!\verb!Bicycle"! - before the \verb!`B'! in \verb!"Big"! and before the \verb!`B'! in \verb!"Bicycle"!.

\item[END:] The \emph{end anchor} uses the special character \verb!$! to indicate either the position between the last newline and the character before it, or between the end of the string and the character before it if the string does not end in a newline.  For example \cverb!R$! will match \verb!"abcR"! and \verb!"xyz!\gverb!\n!\verb!R!\gverb!\n!\verb!"! but not \verb!"R!\gverb!\n!\verb!xyz!\gverb!\n!\verb!"!.  The `MULTILINE' flag or `m' option also affects the END anchor so that if activated, the string \verb!"R!\gverb!\n!\verb!xyz!\gverb!\n!\verb!"! \emph{will} match because there exists a line where \verb!`R'! is at the end of a line.

\item[ENDZ:] The \emph{absolute end anchor} uses the special sequence \verb!\Z! to indicate the absolute end of the.  For example \cverb!R\Z! will match \verb!"abcR"! and  \verb!"xyz!\gverb!\n!\verb!R"! but not \verb!"xyzR!\gverb!\n!\verb!"! or \verb!"Rs"!.

The syntax of this feature may cause much confusion when porting to another language like Java, Perl, JavaScript, etc. where the lowercase z: \cverb!\z! has this meaning, but the uppercase Z: \cverb!\Z! \emph{would} match \verb!"R!\gverb!\n!\verb!"! - it matches the end of string or before the last newline.
\end{description}

\subsubsection*{Positions: Boundaries}

\begin{description} \itemsep -1pt

\item[WNW:] The \emph{word-nonword} anchor uses the special sequence \verb!\b! to indicate the position between a character belonging to the WRD default character class and belonging to NWRD (or no character, such as the beginning or ending of the string).  It doesn't matter if WRD or NWRD comes first, but WNW will only match if the first character is followed by its opposite.  This is useful when trying to isolate words, for example the regex \cverb!\btaco\b! will match \verb!"taco"! or \verb|"My taco!"|  because there is never a word character next to the target word.  But the same regex will not match \verb!"catacomb"!, \verb!"\_taco"! or \verb!"tacos"!.  The escaped backspace character \verb!`\b'! can only be expressed inside a CCC like \cverb![\b]! because this sequence is treated as WNW by default.

\item[NWNW:] The \emph{negated word-nonword} anchor uses the special sequence \verb!\B! to indicate the position between either two WRD characters or two NWRD characters (or no character, such as the beginning or ending of the string).  The regex \cverb!\Btaco\B! matches \verb!"catacomb"! because a word character is found on both sides of wherever the \cverb!\B! is.  The strings \verb!"tacos"! and \verb!"\_taco"! do not match, though, because in both cases some part of \verb!"taco"! is next to the end of the string.
\end{description}

\subsubsection*{Positions: Lookarounds}

\begin{description}  \itemsep -1pt
\item[LKA:] The \emph{lookahead} feature uses the special syntax \verb!(?=R)!  to check if regex R matches immediately after the current position.  The string matched by R not captured, and is also excluded from group 0.  That is why this feature is sometimes called a \emph{zero-width lookahead}.  For example \cverb!ab(?=c)! matches \verb!"abc"! and has \verb!"ab"! in group 0.  Note that a regex like \cverb!ab(?=c)d! is valid but does not make sense, because the lookahead and \cverb!d! can never both match.


\item[LKB:] The \emph{lookback} uses the special syntax \verb!(?<=R)! to check if regex R matches immediately before the current position.  As with LKA, the content matched by the LKB is excluded from group 0.

\item[NLKA:] A \emph{negative lookahead} uses the special syntax \verb|(?!R)| to require that regex R \emph{does not match} immediately after the current position.  Matched content is excluded from group 0.


\item[NLKB:] The \emph{negative lookback} uses the special syntax \verb|(?<!R)| to require that regex R does not match immediately before the current position.  Matched content is excluded from group 0.
\end{description}

