\section*{Artifact Details}
\label{app:nodeCountImplementation}
The implementation is written in Java 8, and depends on antlr 3.5.2 because the PCRE parser used\footurl{https://github.com/bkiers/pcre-parser} uses antlr to identify the features present in a pattern.  Details about the corpus used can be found in Section~\ref{sec:featureUsage}.  For this experiment, the corpus was re-loaded from a text file.  Tests verify that this re-loaded corpus is identical to the corpus built from scratch.

Each regex in the corpus contains the original pattern used to compile the regex, a
\\*{\tt org.antlr.runtime.tree.CommonTree} (parse tree) created by the PCRE parser from the pattern, and a set of integers used to identify the set of Python projects that used this regex in at least one utilization.

\subsection*{Tokenstreams}
A \emph{tokenstream} is a string that can be generated from a regex's parse tree to represent the parsed regex as a string. Unlike the pattern used to compile the regex, all ambiguities due to multiple meanings of characters, balanced parenthesis, etc. have already been resolved by the parser. This sequence of tokens is still a context free language with nested, balanced \verb!DOWN! and \verb!UP! tokens, therefore it cannot be fully described or parsed using Java Regular Expressions (which do not support the recursion feature or similar).  However, the tokenstream can be searched for necessary conditions when constructing filters to identify candidacy for node membership, and regexes that are erroneously added to a node can be removed manually, as described in Section~\ref{sec:nodeCountOverview}.

Each leaf of the parse tree is assigned a text representation based on the token names used by the parser.  The `bullet' character was chosen as a delimiter that is not present in any text representation.  Invisible characters and Unicode characters like the `bullet' are represented using a hex representation of their bytes.  The tokenstream is created by joining the text representation of each leaf with the delimiter, as shown in Figure\todoMid{N}.


\section{Implementation details of determining node membership}

\subsection{Membership based only on feature presence}
The nodes which only require a check for the presence of a feature to determine membership are described here:
\begin{description} \itemsep -1pt
\item[D2] requires QST (zero-or-one repetition using question mark)
\item[S1] requires SNG (curly braces with one number inside specifying the number of repetitions)
\item[L1] requires LWB (curly braces with one number followed by a comma, specifying a lower bound on repetition)
\item[L2] requires KLE (kleene star indicating zero-or-more repetitions)
\item[L3] requires ADD (plus character indicating one-or-more repetitions)
\item[C3] requires NCCC (a negated custom character class, where a \verb!`^'! negates a CCC like \cverb![^X]!)
\end{description}

\subsection{Membership based on a feature presence and search of the pattern}
\begin{description} \itemsep -1pt
\item[S3] requires DBB, and requires the regex's pattern to match \cverb!\{(\d+),\1\}! which guarantees that both bounds of DBB are the same by capturing the first bound in \cverb!(\d+)! and then back-referencing the captured number.
\item[T2] requires the presence of some hex character representation in the pattern, which is verified by searching the regex's pattern with the regex \cverb!\\x[a-f0-9A-F]{2}!.
\item[T4] requires the presence of some octal character representation in the pattern, which is verified by searching the regex's pattern with the regex \cverb!((\\0\d*)|(\\\d{3}))!.  Python-style octals require either exactly three digits after a slash, or a zero and some other digits after a slash.  Only one false positive was identified which was actually the lower end of a hex range using the literal \verb!\0!.
\item[D3] requires OR (alternation using the \verb!`|'!), and requires the regex's pattern to match \cverb!(?<=[|])([^ \\]+)\1+|([^ \\]+)\1+(?=[|])!.  The core of this regex is \cverb!([^ \\]+)\1+! which describes a string that contains a repetition of some sequence at least two times.  The sequence is captured by \cverb!([^ \\]+)! and then back-referenced by \cverb!\1+!, which can appear one or more times as specified by ADD.

In this regex, the lookbehind \cverb!(?<=[|])! and lookahead \cverb!(?=[|])! match when the \verb!`|'! character is found to the left or right of the core regex, respectively.  The regex used as a filter matches either \cverb!(?<=[|])([^ \\]+)\1+! or \cverb!([^ \\]+)\1+(?=[|])!.  All patterns in the corpus that have some repeated sequence as an alternative within an OR match this regex.  For example the pattern \verb!"a|aa"! compiles to the regex \cverb!a|aa! which belongs to D3.  This filter produced a list of 113 candidates which were narrowed down manually to 10 actual members.

The space and slash characters are excluded from the sequence described by \cverb!([^ \\]+)! in order to exclude common false positives like \verb!"some text  |more text"! and \verb!"\\|/"!, which match the regex but do not belong to D3.  Note that these false positives were manually verified as described in Section~\ref{sec:nodeCountOverview}, ensuring that in the corpus used, no valid members of D3 were excluded.  However it is possible for this filter to exclude a regex with a pattern like \verb!"(  | )"! or \verb!"(\\\\|\\)"! which should belong to D3.
\end{description}

\subsection{Membership based on a feature presence and filters}
\begin{description} \itemsep -1pt
\item[D1] requires DBB (curly braces with two numbers separated by a comma inside), where the two numbers are different.  When these two numbers are the same, then the regex cannot be refactored within the DBB group, but instead belongs to the S3 node.  It is likely impossible to directly specify that two numbers are different in Java Regular Expressions, but it is possible to identify when two numbers are the same using back-references, and eliminate only these.  So the same regex used to identify members of S3: \cverb!\{(\d+),\1\}! was used to find all such same-bound parts of a pattern and replace them with \verb!"{$1}"!, where \verb!$1! is referencing the digit captured by \cverb!(\d+)}!.  In effect this step is actually performing a refactoring from S3 to S1.  All patterns that still contain DBB syntax must belong to D1, so the modified pattern is then searched using \cverb!\{\d+,\d+\}! to determine membership in D1.

\item[TODO 7 MORE] \todoMid{finish the remaining 7 filter implementation descriptions}
\item[S2] requires any element to be repeated at least twice. This element could be a character class, a literal, or a collection of things encapsulated in parentheses.
\item[T1] requires that no characters are wrapped in brackets or are hex or octal characters, which matches over 91\% of the total regexes analyzed.
\item[T3] requires that a single literal character is wrapped in a custom character class (a member of T3 is always a member of C2).
\item[C1] requires that a non-negative character class contains a range.
\item[C2] requires that there exists a custom character class that does not use ranges or defaults.
\item[C4] requires the presence of a default character class within a custom character class, specifically, \verb!\d!, \verb!\D!, \verb!\w!, \verb!\W!, \verb!\s!, \verb!\S! and \verb!.!.
\item[C5] requires an OR of length-one sequences (literal characters or any character class).
\end{description}
